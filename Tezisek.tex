\documentclass[a4paper,10pt]{article}
%\usepackage{a4wide}
\usepackage{fullpage}
\usepackage{enumerate}
\usepackage{t1enc}
\usepackage[latin2]{inputenc}
\usepackage[T1]{fontenc}
\usepackage{ae,aecompl}
\usepackage[magyar]{babel}
\usepackage{indentfirst}

\frenchspacing 
\pagestyle{empty}
\title{Theses}
\date{}
\begin{document}
\begin{center}
  \textbf{\Large Theses}\\[0.5cm]
\end{center}

\begin{itemize}
\item General WFS theory. In order to have a more general WFS theory I completed traditional WFS theory and showed a link between traditional and generalized WFS formulation
\begin{itemize}
\item I gave the traditional WFS driving functions fo a plane wave by utilizing the stationary phase approximation. I verified, that similarly for a virtual point source traditional WFS is an approximation of the explicit solution. 
\item Based on the plane wave expansion I gave a more general traditional WFS driving function for an arbitrary 2D sound field. Utilizing this formulation as a demonstration I derived the traditional WFS driving functions for a virtual line source.
\item I pointed out, that generalized WFS formulation suffers from two sources of error: a virtual source dimensionality mismatch
and a secondary source dimensionality mismatch. I showed, that the first will case a constant amplitude error in case of the synthesis of a 3D sound field using a linear set of 3D point sources. This error may be compensated if the virtual sound field is known apriori. Secondary source mismatch will cause amplitude errors in the syntehsized sound field. 
\item I showed, where referencing type, suggested by Ahrens will reference the synthesis of a point source.
\end{itemize}

\item Stochastic Sound Field Synthesis. I created a model for Sound Field Synthesis using extended secondary sources elements arranged into a line array, exhibiting stochastic properties. Stochastic sensitivity, rugosity, mechanical anisotropy and modal behaviour are included in the new model.
	
	\begin{itemize}
	\item I set up and verified a simple far-field approximation for the SFS using extended secondary source elements. Based on the wave-number description of the SFS problem I showed, that not only the directivity characteristics -- which was known from the literature -- but also the spatial extension function of the virtual source and the secondary source elements are inversely interchangeable.
	\item Applying the model, closed analytical formula was given for the spatial distribution characteristics of the noise originating from both stochastic sensitivity and mechanical anisotropy. For the case of a plane wave -- !!! It is proven, that in this case the spatial characteristics of the noise field is determined by the field of secondary source 				elements-. 	
		
	\item For the case of a point source it is proven that in this case noise seems to originate in both cases from a source with a stochastic directivity, located closer to the secondary array than the original deterministic virtual source regarding its amplitude
	\end{itemize}
	
\item Synthesis of uniformly moving sources with the Spectral Division Method
	\begin{itemize}
	\item I gave analytical expression for the wavenumber-frequency content of a sound field, generated by a moving acoustic point source with an arbitrary straight trajectory
	\item I verified, that the derived analytic expression converges weakly to a Dirac-delta distribution for a sound source, moving along the direction of the Fourier-transformation.
	\item Based on the expression for the wavenumber-content I gave the SDM driving functions for the synthesis of a moving point source in the wavenumber domain
	\item For the special case of a source moving parallel to the secondary source distribution I gave analytical, closed form driving function in the spatial-frequency domain
	\end{itemize}
\item Synthesis of uniformly moving sources with Wave Field Synthesis
	\begin{itemize}
	\item Frequency-domain WFS
		\begin{itemize}
		\item I gave the frequency domain 3D WFS driving functions for a planar secondary source distribution for a moving point source
		\item By adapting the stationary phase approximation to the 3D WFS driving functions I derived the 2.5D WFS driving functions for a linear secondary source distribution for a moving point source with an arbitrary straight trajectory
		\item For the case of a source with a trajectory parallel to the SSD I proved, that the WFS is the high-frequency/farfield approximation the SDM.
		\end{itemize}
	\item Time-domain WFS
		\begin{itemize}
		\item I adapted the stationary phase approximation for the 3D time domain WFS driving functions. As a result the analytical 2.5D WFS driving functions are obtained for a source, 	moving at arbitrary direction with an arbitrary source excitation, optimizing the synthesis to a line, parallel to the secondary source distribution
		\item I verified the link between the time-domain and spectral domain WFS driving functions by performing the necessary analytical Fourier-transform.
		\end{itemize}
	\end{itemize}
\end{itemize}
\end{document}