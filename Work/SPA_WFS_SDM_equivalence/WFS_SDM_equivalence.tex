\documentclass[12pt,a4paper]{article}
%

\usepackage{layouts}
\usepackage{amsmath}
\usepackage{a4wide}
\usepackage[T1]{fontenc}
\usepackage[utf8]{inputenc}
\usepackage{xcolor}
\usepackage{listings}
\usepackage{graphicx,overpic,subfigure}
\usepackage{tikz}
\usetikzlibrary{positioning,arrows}
\usepackage{booktabs} 			% Nice tables
\usepackage{csquotes}			% Quotation
\usepackage{multirow} 			% Multirow cells in tables
\usepackage{rotating}
\usepackage{pdflscape}
\usepackage[small,bf]{caption}
\usepackage{ae,aecompl}
\usepackage{url}
\usepackage[american]{babel}
\usepackage{hyperref}
\usepackage{nomencl}
\usepackage[toc,page]{appendix}
\usepackage{amssymb}
\usepackage{steinmetz}
\usepackage{palatino}
\usepackage{array}
\usepackage{booktabs}
\usepackage{footnote}
\usepackage{multicol}
%

\newcommand{\dint}{\int\!\!\!\!\!\int}
\newcommand{\tint}{\int\!\!\!\!\int\!\!\!\!\int}
\newcommand{\qint}{\int\!\!\!\!\int\!\!\!\!\int\!\!\!\!\int}
\newcommand{\td}{\mathrm{d}}
\newcommand{\te}{\mathrm{e}}
\newcommand{\ti}{\mathrm{j}}
\newcommand{\sinfi}{\sin\varphi}
\newcommand{\cosfi}{\cos\varphi}
\newcommand{\sinteta}{\sin\theta}
\newcommand{\costeta}{\cos\theta}
\newcommand{\yref}{y_{\mathrm{ref}}}
\newcommand{\dref}{d_{\mathrm{ref}}}
\newcommand{\vx}{\mathbf{x}}
\newcommand{\vxo}{\mathbf{x}_0}
\newcommand{\vxs}{\mathbf{x}_{\mathrm{s}}}
\newcommand{\vk}{\mathbf{k}}
%
\title{Connection of the explicit solution and unified WFS theory}
\date{\today \\
Budapest University of Technology and Economics, \\ Dept. of Networked Systems and Services, \\ Laboratory of Acoustics and Studio Technologies}
\author{Firtha Gergely}

\begin{document}
\maketitle

\section{Preliminary concepts}
Consider the explicit driving function for a linear SSD consisting of 3D point sources described by $G(\vx,\omega)$. The driving functions are defined only in the synthesis plane $z=0$:
\begin{equation}
\tilde{D}(k_x,y,\omega) = \frac{\tilde{P}(k_x,y,0,\omega)}{\tilde{G}(k_x,y,0,\omega)}. 
\end{equation}
For the sake of simplicity in the followings both $z$ and $\omega$ dependencies are suppressed.
Restricting of both $P$ and $G$ to the synthesis plane will result up in phase correct synthesis of sound fields, that's local propagation direction is restricted to the same plane. This can be formulated using the local wavenumber vector.

An arbitrary complex valued sound field may be written in a polar form as
\begin{equation}
P(\vx,\omega) = A_P(\vx,\omega) \te^{\ti \phi_P (\vx,\omega)},
\end{equation}
with $A_P(\vx,\omega), \phi_P(\vx,\omega) \in \mathbb{R} $.
Let's define the local wavenumber vector by
\begin{equation}
\vk^l_P(\vx) = [k_x^l(\vx),\ k_y^l(\vx),\ k_z^l(\vx)]^{\mathrm{T}} = -\nabla \phi_P(\vx,\omega),
\end{equation}
i.e. as the gradient of the phase function. The wavenumber vector points in the local propagation direction of an arbitrary sound field, with its magnitude equaling $k = \omega/c$ i.e. the acoustic wavenumber. Superscript $l$ is introduced in order to distinguish the here defined local wavenumber from the spectral components in the followings. Throughout the derivation it is a pre-requisition, that $k_{z,P}^l(x,y,0) = 0$. 

Furthermore, for the sake of simplicity we restrict our investigation into non-converging, or non-focused waves, which we define from simple geometric considerations as waves, for 
\begin{eqnarray}
\frac{\partial}{\partial x} k_{x,P}^l(\vx) = -\frac{\partial^2}{\partial x^2} \phi_P (\vx,\omega) \geq 0, \hspace{3mm}\rightarrow \hspace{3mm} \frac{\partial^2}{\partial x^2} \phi_P (\vx,\omega) \leq 0\\
\frac{\partial}{\partial y} k_{y,P}^l(\vx) = -\frac{\partial^2}{\partial y^2} \phi_P (\vx,\omega) \geq 0, \hspace{3mm}\rightarrow \hspace{3mm} \frac{\partial^2}{\partial y^2} \phi_P (\vx,\omega) \leq 0\\
\frac{\partial}{\partial z} k_{z,P}^l(\vx) = -\frac{\partial^2}{\partial z^2} \phi_P (\vx,\omega) \geq 0 \hspace{3mm}\rightarrow \hspace{3mm} \frac{\partial^2}{\partial z^2} \phi_P (\vx,\omega) \leq 0
\end{eqnarray}
holds. The equations trivially hold for an acoustic point source.

\section{Inverse Fourier transform of the driving functions}

\subsection{Approximation of wavenumber spectra}
By definition the wavenumber content of the involved quantities are obtained via a forward Fourier transform
\begin{equation}
\tilde{P}(k_x,y) = \int_{-\infty}^{\infty} P(x,y) \te^{\ti k_x x} \td x = \int_{-\infty}^{\infty} A_P(x,y) \te^{\ti \phi_P(x,y)} \te^{\ti k_x x} \td x,
\end{equation}
\begin{equation}
\tilde{G}(k_x,y) = \int_{-\infty}^{\infty} G(x,y) \te^{\ti k_x x} \td x = \int_{-\infty}^{\infty} A_G(x,y) \te^{\ti \phi_G(x,y)} \te^{\ti k_x x} \td x.
\end{equation}
Here $k_x$ denotes the corresponding wavenumber component of the actual global plane wave, constituting the expansion basis of the transform (i.e. the actual spectral component).

Now consider, that the involved spectra are obtained by using the stationary phase approximation:
Under high-frequency assumptions ($\phi(x,y,\omega)+k_x x \gg 1$) the Fourier integral may be approximated by evaluation around its stationary point (assuming only one stationary point in the integral path). The stationary point $x^*(k_x)$ is found, where the $x$-derivative of the exponent is zero, i.e. where
\begin{equation}
- \left. \frac{\partial}{\partial x} \phi(x,y,\omega) \right|_{x = x^*(k_x)} = k_x^l(x^*(k_x),y,\omega) = k_x,
\end{equation}
holds: \emph{the greatest contribution to the wavenumber spectrum at an arbitrary wavenumber $k_x$ has the point in space, where the local wavenumber component $k_x^l(\vx)$ equals $k_x$}. Note, that it is assumed, that in the sound field each local propagation direction is unique.
Since each $k_x$ wavenumber component assigns a new stationary point, therefore the stationary point for a given $k_x$ is denoted including its dependency by $x^*(k_x)$.

The stationary phase approximation states, that around the stationary point $x^*$ an arbitrary oscillating integral can be approximated as 
\begin{equation}
\int_{-\infty}^{\infty} F(x)\te^{\ti \phi(x)} \td x \approx \sqrt{\frac{2\pi}{| \left. \frac{\partial^2}{\partial x^2} \phi(x)\right|_{x = x^*}|  }}
F(x^*) \te^{\ti \phi(x^*) + 
\ti \frac{\pi}{4} \text{sgn}\left( \left. \frac{\partial^2}{\partial x^2} \phi(x)\right|_{x = x^*} \right)}
\end{equation}

Supposing, that $x^*_P(k_x)$ and $x^*_G(k_x)$ are the stationary positions for the corresponding integrals, i.e. 
\begin{eqnarray}
- \left. \frac{\partial}{\partial x} \phi_P(x,y,\omega) \right|_{x = x^*_P(k_x)} = k_{x,P}^l(x^*_P(k_x),y,\omega) = k_x, \\
- \left. \frac{\partial}{\partial x} \phi_G(x,y,\omega) \right|_{x = x^*_G(k_x)} = k_{x,G}^l(x^*_G(k_x),y,\omega) = k_x,
\end{eqnarray}
holds, and accounting for the negative derivatives---since both $P$ and $G$ are non-converging waves---their spectra can be approximated as
\begin{eqnarray}
\tilde{P}(k_x,y) \approx \sqrt{\frac{2\pi}{\ti |\phi_{P,x,x}''(x^*_P(k_x),y)|}} A_P(x^*_P(k_x),y) \te^{\ti \phi_P(x^*_P(k_x),y)} \te^{\ti k_x x^*_P(k_x)}
\\
\tilde{G}(k_x,y) \approx \sqrt{\frac{2\pi}{\ti |\phi_{G,x,x}''(x^*_G(k_x),y)|}} A_G(x^*_G(k_x),y) \te^{\ti \phi_G(x^*_G(k_x),y)} \te^{\ti k_x x^*_G(k_x)}
.
\end{eqnarray}
For the sake of simplicity the notation $\left. \frac{\partial^2}{\partial x^2} \phi(x)\right|_{x = x^*} = \phi_{x,x}''(x^*)$ is used.

Finally the driving function spectrum can be approximated as
\begin{multline}
\tilde{D}(k_x) \approx 
\sqrt{\frac{|\phi_{G,x,x}''(x^*_G(k_x),y)|}{|\phi_{P,x,x}''(x^*_P(k_x),y)|}}
\frac{A_P(x^*_P(k_x),y)}{A_G(x^*_G(k_x),y)}
\te^{\ti \left( \phi_P(x^*_P(k_x),y) - \phi_G(x^*_G(k_x),y)\right)}
\te^{\ti k_x \left( x^*_P(k_x) - x^*_G(k_x)\right)}=
\\
=\sqrt{\frac{|\phi_{G,x,x}''(x^*_G(k_x),y)|}{|\phi_{P,x,x}''(x^*_P(k_x),y)|}}
\frac{P(x^*_P(k_x),y)}{G(x^*_G(k_x),y)}
\te^{\ti k_x \left( x^*_P(k_x) - x^*_G(k_x)\right)}.
\end{multline}


\subsection{Approximation of the inverse transform}

By definition the inverse Fourier transform of the driving function spectrum is given as
\begin{equation}
D(x_0) = \frac{1}{2\pi} \int_{-\infty}^{\infty} \tilde{D(k_x)} \te^{-\ti k_x x_0} \td k_x.
\end{equation}
As the next 

\end{document}