	\documentclass[10pt,twoside]{article}
\usepackage[a5paper,inner=1.5cm,outer=2.0cm,top=2.0cm,bottom=2.5cm,pdftex]{geometry}
\usepackage[utf8]{inputenc}						% UTF-8 encoding
\usepackage[T1]{fontenc}
\usepackage[magyar,american]{babel}
\usepackage{csquotes}
\usepackage[backend=biber,
			hyperref=true,
			autolang=hyphen,
			style=numeric-comp,
			isbn=false,
			sorting=none,
			defernumbers=true,
			maxnames=10,
			doi=false,
			url=false,
			useprefix=true,
			]{biblatex}
\usepackage{palatino}							% Font
\usepackage{amsmath, amssymb, amsthm} 							% Theorem customization
\usepackage{graphicx,subfigure,overpic}
\usepackage{rotating}
\usepackage{multirow}
\usepackage{xcolor}
\usepackage[bf,small]{caption}
\addbibresource{../LatexProject/dissertation.bib}
\addbibresource{booklet.bib}

\newcount\posveccount
\newcommand*\posvec[1]{
        \global\posveccount#1
        [
        \posvecnext
}
\def\posvecnext#1{
        #1
        \global\advance\posveccount-1
        \ifnum\posveccount>0
                ,\
                \expandafter\posvecnext
        \else
                ]^{\mathrm{T}}
        \fi
} 
\newcommand{\vxs}{\mathbf{x}_{\mathrm{s}}}
\newcommand{\vxref}{\mathbf{x}_{\mathrm{ref}}}
\newcommand{\vk}{\mathbf{k}} 

\usepackage[pdftitle={A hangtérszintézis elmélet általánosítása (Tézisfüzet)},%
  pdfauthor={Firtha Gergely},% 
%  colorlinks=true,% 
%  urlcolor=blue,% 
%  linkcolor=blue,% 
%  citecolor=blue,% 
%  filecolor=green,% 
%  raiselinks=false,% 
  hyperfootnotes=true]{hyperref}

% Citation categories
\DeclareBibliographyCategory{journal} 
\DeclareBibliographyCategory{conference}
\DeclareBibliographyCategory{tech}
\DeclareBibliographyCategory{other}
% Citation commands
\newcommand*{\citeJ}[2][]{\addtocategory{journal}{#2}\cite[#1]{#2}}
\newcommand*{\citeC}[2][]{\addtocategory{conference}{#2}\cite[#1]{#2}}
\newcommand*{\citeO}[2][]{\addtocategory{other}{#2}\cite[#1]{#2}}
\newcommand*{\citeT}[2][]{\addtocategory{tech}{#2}\cite[#1]{#2}}
% No citation commands
\newcommand*{\nociteJ}[1]{\addtocategory{journal}{#1}\nocite{#1}}
\newcommand*{\nociteC}[1]{\addtocategory{conference}{#1}\nocite{#1}}
\newcommand*{\nociteO}[1]{\addtocategory{other}{#1}\nocite{#1}}
\newcommand*{\nociteT}[1]{\addtocategory{tech}{#1}\nocite{#1}}

\nociteJ{firtha2015sound,firtha2016wave_booklet,Firtha2016_booklet,doi:10.1121/1.4996126_booklet,schultz2017wave_booklet,Firtha2018:WFS_vs_SDM_booklet}
\nociteC{Firtha2012:isma_booklet,Firtha2013:daga_booklet,Firtha2013:internoise_booklet,Firtha2014:daga_booklet,Firtha2014:isma_booklet,
Firtha2015:daga_booklet,firtha2016:daga_booklet,Firtha2017:daga_booklet,Firtha2018_daga_a_booklet,Firtha2018_daga_moving_source_booklet}

\renewcommand{\bibfont}{\normalfont\small}
\renewcommand*{\mkbibnamegiven}[1]{%
\ifitemannotation{highlight}
{\textbf{#1}}
{#1}}



\renewcommand*{\mkbibnamefamily}[1]{%
\ifitemannotation{highlight}
{\textbf{#1}}
{#1}}

\newtheoremstyle{thesisgroupstyle}%
	{1em}% 				% Space above
	{5mm}% 				% Space below
	{\slshape}%			% Font type of body
	{0mm}%				% Indentation			
	{\bfseries}%		% Font of title
	{\newline}%			% Punctuation between head and body
	{1em}%				% Space after theorem head
	{}% 				% Manually specify head

\makeatother
\theoremstyle{thesisgroupstyle}
\newtheorem{thesisgroup}{Téziscsoport}
\renewcommand{\thethesisgroup}{\Roman{thesisgroup}}

\makeatletter
\newtheoremstyle{indented}
  {5pt}% space before
  {5pt}% space after
  {\addtolength{\@totalleftmargin}{3.5em}
   \addtolength{\linewidth}{-3.5em}
   \parshape 1 3.5em \linewidth}% body font
  {}% indent
  {\bfseries}% header font
  {.}% punctuation
  {.75em}% after theorem header
  {}% header specification (empty for default)
\makeatother
\theoremstyle{indented}
\newtheorem{thesis}{tézis}[thesisgroup]


\newcommand{\theauthor}{Firtha Gergey}
\newcommand{\authorprof}{M.Sc.E.E.}
\author{\theauthor}
\title{A hangtérszintézis elméletének általánosítása\\[.5ex] \large  és alkalmazása mozgó források szintézisére}

\assignrefcontextkeyws[labelprefix=J]{J}
\assignrefcontextkeyws[labelprefix=C]{C}
\assignrefcontextkeyws[labelprefix=O]{O}

% Add underline for my author name
\def \mylastname {Firtha}
\newbibmacro{name:underlined}{%
    \ifthenelse{\equal{\namepartfamily}{\mylastname}}%
{\underline{\ifblank{\namepartgiven}{}{\namepartgiven\space}\namepartfamily}}%
{\ifblank{\namepartfamily}{}{\namepartgiven\space}\namepartfamily}%
    \ifthenelse{\value{listcount}<\value{liststop}}%
    {\addcomma\space}%
    {}%
}

\DeclareNameFormat{author}{%
    \nameparts{#1}% split the name data, will not be necessary in future versions
    \usebibmacro{name:underlined}%
    \usebibmacro{name:andothers}%
}


\begin{document}
\sloppy
\newcommand{\schoolname}{Budapesti Műszaki és Gazdaságtudományi Egyetem}
\newcommand{\facultyname}{Villamosmérnöki és Informatikai Kar}
\newcommand{\doctoralname}{Villamosmérnöki Tudományok Doktori Iskola}
\newcommand{\bookname}{PhD tézisfüzet}
\newcommand{\authorname}{Author}
\newcommand{\supervisorname}{Konzulens:}
\newcommand{\supervisor}{Dr. Fiala Péter}
\newcommand{\creationdate}{Budapest, 2018.}
\def \deptlogo {logos/hit_logo_en.png}
\def \lablogo {logos/last_logo_eng.png}

\selectlanguage{magyar}
\nonfrenchspacing

%\thispagestyle{empty}
\newcommand{\HRule}{\rule{\linewidth}{1pt}}

\begin{titlepage}
\begin{center}

%%% Header with university logo
\includegraphics[height=1.2cm]{logos/BME1782.pdf}
\HRule\par
{\small \textsc{ \bfseries \schoolname \\
\facultyname \\
\doctoralname}}\par

\vfill

%%% Inserting the title
{
\Large
\makeatletter
\@title
\makeatother
} \par
\bigskip
\bookname \par
\bigskip
{
\normalsize
\theauthor\\[.5ex]
\authorprof
}\par

\vfill
%%% Inserting author, supervisor and co-promotor
\begin{tabular}{@{}p{3cm}l@{}}
	\supervisorname & \supervisor \\
\end{tabular}

\vspace{\fill}

\hspace{1.0cm} \includegraphics[height=1cm]{\deptlogo} \hspace{2cm}
\includegraphics[height=1cm]{\lablogo} \par

\vspace*{\stretch{0.33}}
%%% Inserting date
\normalsize
\creationdate



\end{center}


\end{titlepage}

\thispagestyle{empty}
\cleardoublepage
\setcounter{page}{1}

\section{Bevezetés}

\subsection{A fizikai hullámtérszintézis célja}
%
A térhangzású technikák alapvető célja egy virtuális akusztikai környezet előállítása úgy, hogy a térben elhelyezett audio objektumokat a hallgató a kívánt térbeli jellemzőkkel (pl. virtuális hangforrás pozíciója, szélessége) érzékeli.
A hangtér-reprodukció ezt egy fix elhelyezésű hangszórósokaság megfelelő vezérlésével éri el azáltal, hogy a hangszórók által keltett hullámok szuperpozíciója a hallgató számára a kívánt akusztikai környezet hatását kelti.
Ezzel szemben a binauralizációs módszerek a térhangzást fejhallgatók jelének megfelelő szűrésével érik el.

A mai hangtér-reprodukciós, illetve térhangzású technikák kiinduló pontja a Blumlein által 1931-ben bevezetett kéthangszórós rendszer volt, lefektetve ezzel	 a sztereofónia alapjait.
A sztereofónia manapság is a legelterjedtebb reprodukciós módszer, olyan hangrendszereknek képezi az alapját, mint a Dolby 5.1, 7.1, az NHK vállalat 22.2 rendszere, vagy épp a jelenleg state-of-the-art kereskedelemben is kapható térhatású rendszereknek, a Dolby Atmosnak és a DTS-X-nek.
Általánosan véve---függetlenül a hangrendszerben alkalmazott hangszórók számától---a sztereofónia célja pusztán a térérzetet befolyásoló tényezők (pl. füljelek közötti késleltetés, illetve amplitúdókülönbség) visszaállítása a hallgató pozíciójában.
Ebből kifolyólag azonban a tökéletes térhangzás csak a tér egyetlen pontjának szűk környezetében biztosítható, ez az ún. \emph{sweet spot}.
A fenti megszorítás a sztereofón technikák fontos közös hátrányának tekinthető.

A sztereofóniával ellentétben a fizikai hangtér-reprodukció (sound field synthesis, SFS) célja egy tetszőleges célhangtér fizikai jellemzőinek visszaállítása egy kiterjedt megfigyelési területen.
Amennyiben ezt sikerül elérni, a hallgató a megfigyelési terület tetszőleges pontján eleve a kívánt akusztikai érzeti jellemzőket érzékeli.
Természetesen egy hullámtér kiterjedt területen való befolyásolása bonyolult feladat, elméletileg a megfigyelési terület hangszórókkal való körbezárását igényeli.
A fizikai hangtér-reprodukció alapfeladata ezek után az egyes hangszórók vezérlőjelének, vagy \emph{vezérlő függvényének} meghatározása úgy, hogy a teljes hangszórósokaság eredő hangtere megegyezzen az előírt \emph{virtuális térrel}.
A hangszórósokaság elnevezése a továbbiakban a \emph{másodlagos forráseloszlás (secondary source distribution, SSD)}.
A reprodukció általános elrendezése a \ref{fig:introduction:sfs_aim} ábrán látható.

\begin{figure}  
\small
  \begin{minipage}[c]{0.64\textwidth}
	\begin{overpic}[width = 1\columnwidth ]{figs/sfs_aim.png}
	\small
	\put(58,30){megfigyelési terület}
	\put(0,45){\parbox{.5in}{virtuális forrás}}
	\put(45,7){másodlagos forráseloszlás}
	\end{overpic}   \end{minipage}\hfill
	\begin{minipage}[c]{0.3\textwidth}
    \caption{Általános hangtérszintézis elrendezés: a szintézis célja egy tetszőleges virtuális forráseloszlás terének előállítása a megfigyelési terület határán elhelyzetett hangszórósokaság megfelelő vezérlésével.}
\label{fig:introduction:sfs_aim}  \end{minipage}
\end{figure}

Az elmúlt néhány évtized során számos hangtérszintézis technika megjelent, mind numerikus, mind analitikus megoldásokat kínálva a problémára.
Az analitikus módszerek két fő csoportra oszthatók: a szintézisproblémát leíró integrálegyenletek közvetlen megoldására szolgáló explicit, illetve a Huygens elven alapuló ún. implicit módszerekre.
Az utóbbi, implicit módszer, elterjedt nevén a hangtérszintézis (a következőekben Wave Field Synthesis, WFS) a jelen disszertáció fő témája.

\subsection{A hangtérszintézis története}
%
A hangtérszintézis elméleti alapjai a Delfti Műszaki Egyetemről származnak. 
Itt elsőként ültette át az A.J. Berkhout által vezetett kutatócsoport a szeizmikus hullámelmélet addigra már jól ismert alapkoncepcióit hangszóró- és mikrofonsorokra.
Az ezáltal bevezetett \emph{tradicionális hangtérszintézis} elméletének alapja a Huygens-elv matematikai formájának tekinthető Rayleigh-integrál, amely integrál egy tetszőleges hangteret egy felületi integrál formájában ad meg.
Az integrál magában tartalmazza egy síkmentén elhelyezett hangszórósokaság vezérlőfüggvényét.
Ahhoz, hogy meghatározzák a gyakorlatban is megvalósítható vonalmenti hangszórósorok vezérlőjeleit a Rayleigh-integrált az ún. \emph{stacionárius fázis módszerével} egy kontúrintegrállá redukálták, amely implicite tartalmazza a keresett vezérlőfüggvényeket.

A technika korai megfogalmazása dipól tulajdonságokkal rendelkező hangszórósorokra szolgáltatott vezérlőjeleket,\footfullcite{Berkhout1988}\textsuperscript{,}\footfullcite{Berkhout1993:Acoustic_control_by_WFS} amelyet rövidesen a dinamikus hangszórókat jobban modellező másodlagos monopólusokra is kiterjesztettek.\footfullcite{doi:10.1121/1.404755}\textsuperscript{,}\footfullcite{Start1997:phd}\textsuperscript{,}\footfullcite{Verheijen1997:phd}
A technikával lehetővé vált virtuális 3D pontforrások hullámfrontjának fizikai visszaállítása a hangszórósort és a megfigyelőt tartalmazó horizontális síkban, a \emph{szintézis síkjában}.
Jól ismert volt az a tény is, hogy az ideálisan sík hangszóróeloszlás hangszórósorrá való redukálása azt eredményezi, hogy az amplitúdió-helyes szintézis még a szintézis síkjában is csak egy megadott görbe mentén, a \emph{referenciagörbén} valósítható meg.
A tradicionális elméletben ezt a görbét a hangszórósorral párhuzamos \emph{referenciavonalnak} választották meg.\footfullcite{Start1997:phd}

Már a korai hangtérszintézis elmélet is számos K+F projektnek szolgált alapul. 
Ezek közül a legkiemelkedőbb a nemzetközi CARROUSO projekt volt, amelynek célja a hangtérszintézis elméletének az abban az időben bevetetett MPEG-4 szabványba való integrálása volt.
Ez a törekvés végül nem valósult meg, azonban a projektnek több spin-off cége, így az IOSONO (ma már Barco) és a Sonic Emotion, jelenleg is gyárt és telepít egyéni tervezésű hangtérszintézis rendszereket.

A hangtérszintézis elméletének legutóbbi mérföldköve Spors és Ahrens munkájához köthető, akik a klasszikus WFS elméletet újragondolva általánosították azt tetszőleges virtuális tér és hangszórókontúrok felé.\footfullcite{Spors2008:WFSrevisited}
Az így kapott vezérlőfüggvényekkel egy tetszőleges vertikális irányban invariáns 2D hangtér visszaállítása visszaállíthatóvá vált, az amplitúdóhelyes szintézist a megfigyelői tér egy pontjára, a \emph{referenciapontra} korlátozva.
Ez az általánosítás már lehetővé tette összetettebb hangterek szintézisét is, így pl. egy mozgó forrásét,\footfullcite{Ahrens2008moving} azonban néhány kiindulási feltételezésből származóan (a levezetés a 2D Rayleigh integrálból indult) még a referenciapontban is amplitúdóhibákhoz vezetett 3D virtuális tér reprodukciója esetén.
Ezen felül a két bemutatott módszer pontos kapcsolata sem volt a korábbiakban tisztázva.

\subsection{A kutatás célja}

A jelen disszertáció a kiindulási alapoktól kezdve felülvizsgálja a hangtérszintézis elméleti alapjait. 
A kutatás célja az elmélet teljes általánosítása volt, amely lehetővé teszi egy tetszőleges virtuális tér reprodukcióját tetszőleges hangszórókontúr alkalmazásával és az amplitúdó-helyes szintézis szabadon választott referenciagörbére való optimalizálásával.
Az így kapott elméleti keretrendszer speciális esetekként magában tartalmazza a korábbiakban bemutatott módszereket: 
Egy virtuális pontforrás egyenes hangszórósorral való szintézise egyenesre megválasztott referenciagörbével közvetlenül a tradicionális hangtérszintézis eredményére vezet.
Hasonlóan, egy 2D virtuális teret egy referenciapontban előállítva a Spors által bemutatott vezérlőfüggvényeket kapjuk. 

További kérdéseket vet fel az implicit hangtérszintézis technika és az explicit megoldások kapcsolata is, amelyet eddig a vonatkozó irodalomban csak speciális virtuális forrás-hangszórókontúr geometriákra vizsgáltak:
Egy köralakú másodlagos forráseloszlás esetén bizonyos virtuális terekre már belátták, hogy a hangtérszintéis az explicit, direkt megoldás nagy-frekvenciás közelítésének tekinthető.\footfullcite{Ahrens2012}\textsuperscript{,}\footfullcite{Spors2010:analysis_and_improvement}
A kutatás további célja ennek a kapcsolatnak az általánosítása volt tetszőleges virtuális térre és hangszóró-, illetve referenciakontúrokra.

A bemutatott eredmények egy összetett alkalmazási példájaként a mozgó források terének szintézise is bemutatásra került.
A mozgó hangforrások által keltett hullámok reprodukciójának kérdéses már szinte a hangtérszintézis bevezetésével egy időben megjelent, mint természetes igény, ha időben változó hangterek szintézise a cél.
A korai megoldások a tradicionális vezérlőfüggvényekkel próbálták megoldani a feladatot, a virtuális forrás pozíciójának idővariánssá tételével.
Ez az egyszerű megoldás azonban nem képes a Doppler hatást megfelelőn visszaállítani és így hallható torzulásokhoz vezet a szintetizált térben.
Ahrens az általuk létrehozott újragondolt elméletet terjesztette ki mozgó forrásokra is.\footfullcite{Ahrens2008moving}\textsuperscript{,}\footfullcite{Ahrens2008moving_b}
A módszer alapvetően működőképes volt, azonban a technika alapvető korlátai miatt a visszaállított tér amplitúdóját a referenciapontban sem lehetett kontrollálni.
Épp ezért, a kutatás további célja a bemutatott elméleti keretrendszer kiterjesztése volt dinamikus, idővariáns virtuális terekre.

\section{Módszer}

A következő fejezetben a disszertációban gyakran alkalmazott módszerek és analízis eszközök kerülnek rövid bemutatásra.

Az általánosított hangtérszintézis elmélet, valamint az explicit megoldás tértartománybeli alakja (amely a későbbiekben összehasonlíthatóvá teszi az explicit módszert a hangtérszintézissel) a különböző perem- és spektrális integrálok aszimptotikus közelítésén alapszik.
Munkám során bevezettem a hangterek lokális tulajdonságait leíró mennyiségeket, így a lokális hullámszámvektort és a hullámfront görbületet, amelyek a későbbiekben nagyban elősegítették a fenti integrálközelítések fizikai interpretációját és megértését.
Ezen lokális tulajdonságok ismert mennyiségek a nagyfrekvenciás akusztika, illetve sugárakusztika területein, de ez idáig nem kerültek bevezetésre a hangtér-reprodukció területén.

A disszertáció központi matematikai módszere az ún. \emph{stacionárius fázis módszere (stationary phase approximation, SPA)}.
Az eljárás komplex értékű függvények integráljának közelítő kiértékelésére szolgál az integrálás útjában található krtikus pontok, az \emph{stacionárius pontok} körül.
Gyakorlatilag a disszertáció ezen közelítés alkalmazásáról szól, azt 
\begin{itemize}
\item a virtuális teret leíró peremintegrálokra alkalmazva, amely segítségével a peremintegrálból a vezérlőfüggvények kiolvashatók.
\item különböző hangterek és a vezérlőfüggvények Fourier integráljára alkalmazva, amely alapján meghatározhatóak az itt bevezetett tértartománybeli vezérlőfüggvények, valamint az itt bemutatott átlapolódásgátló szűrési módszer.
\end{itemize}

Az analitikus eredmények a szintetizált tér numerikus szimulációjának segítségével kerültek validációra.
Ezek a szimulációk egyszerűen az egyes hangszóróelemek analitikusan rendelkezésre álló (a Green-függvény alapján) tereinek a vezérlőfüggvénnyel súlyozott összegeként kerültek kiértékelésre. 

\section{Eredmények}

A következő fejezet a disszertáció legfontosabb eredményeit foglalja röviden össze.
A dolgozatban bemutatott eredményekről rövid összefoglaló nemzetközi kollokvium előadás és cikk is született \cite{Firtha2019:daga_booklet}.
Az új tudományos eredmények a következő fejezetben kerülnek részletes felsorolásra.

\subsection{Az általánosított hangtérszintézis elmélet}

Az következőekben alkalmazott geometria a \ref{fig:introduction:sfs_aim} ábrán látható.
Matematikailag a szintetizált hangteret a hangszóró-vezérlőfüggvény és az egyes hangszóróelemek terét leíró Green-függvény a másodlagos forráseloszlás fölött vett konvolúciójaként írhatjuk fel.
A vezérlőfüggvények meghatározása tehát egy dekonvolúciós, inverz probléma.

A hangtérszintézis alapötlete a virtuális, reprodukálandó hangtér egy olyan peremintegrál formájában történő előállítása, amely peremintegrál implicite tartalmazza a keresett vezérlőfüggvényeket.
Mivel általánosan egy 3D hangtér egy zárt felületi integrál formájában adható meg, az alapvető feladat ezen felületi integrál kontúrintegrállá redukálása.
Ez a horizontális síkba való redukálás egy vertikális irányú stacionárius fázis közelítéssel valósítható meg.

A tradicionális elmélet kiinduló pontja a Rayleigh-integrál volt, amely azonban a másodlagos forráseloszlást eleve egyenes vonalra korlátozta.
Nagyfrekvenciás feltételezéssel élve megfelelő választás a virtuális tér Kirchhoff közelítéssel való reprezentációja, amely egy tetszőleges 3D hangteret egy a Green-függvénnyel vett konvolúciós felületi integrálként ír le.
A Kirchhoff közelítés stacionárius fázisú kiértékelésével a felületi integrált kontúrintegrállá redukáltam, amely már magában tartalmazza a keresett vezérlőfüggvényeket.
%
\begin{figure}[t!]
\small
  \begin{minipage}[c]{0.64\textwidth}
	\begin{overpic}[width = 1\columnwidth ]{figs/25D_WFS_general.png}
	\small
	\put(2,53){(a)}
	\put(2,1){(b)}
	\end{overpic}   \end{minipage}\hfill
	\begin{minipage}[c]{0.35\textwidth}
    \caption{3D pontforrás szintézise állandósult állapotban. A virtuális forrás pozíciója $\vxs = \posvec{3}{0.4}{2.5}{0}$, gerjesztőjele harmonikus jel $f_0 = 1.5~\mathrm{kHz}$ frekvenciával.
	Az ábra (a) része a szintetizált tér valós részét, (b) része a célhangtértől vett amplitúdóhibát ábrázolja logaritmikus skálán.
	Az aktív másodlagos forrásokat folytonos, az inaktív forrásokat szaggatott fekete vonal jelzi.
	Az egyes másodlagos források referenciapozíciója a referenciagörbén jelen példában numerikusan van számítva.
%	Obviously, in the present geometry there exist secondary sources for which no unique reference position can be found.
	}
\label{fig:SFS_theory:25D_WFS_generals}   \end{minipage}
\end{figure}  

Ez a dimenzió-csökkentési eljárás azonban az amplitúdóhelyes szintézis helyét egyetlen megfigyelési pontra, az ún. \emph{referenciapontra} korlátozza.
Az Spors féle hangtérszintézis elmélet ebben a referenciapontban tette lehetővé 2D virtuális terek szintézisét.
Disszertációmban bemutatom, hogy a egy további, horizontális síkban történő aszimptotikus közelítés alkalmazásával minden egyes másodlagos forráselemhez (hangszóróhoz) saját referenciapont rendelhető.
Ennek a forráselemenkénti referenciapontnak a helye analitikusan meghatározható az újonnan bevetett lokális hullámszámvektor segítségével, amely vektor definíciószerűen adott pontban a hullám lokális terjedési irányába mutat.
Rámutattam, hogy az egyes forráselemek referenciapontjának a forráselemtől vett irányát a virtuális tér lokális hullámszámvektora határozza meg, míg a forráselemtől vett távolság egy frekvenciafüggetlen erősítési tényező segítségével tetszőlegesen befolyásolható.
Teoretikus, folytonos másodlagos forráseloszlást feltételezve a referenciapontok összessége is folytonos görbét, ún. \emph{referenciagörbét} alkot.
Amennyiben a kívánt referenciagörbe alakja analitikusan ismert, a frekvenciafüggetlen erősítési tényező meghatározható, ami az amplitúdóhelyes szintézist erre a görbére valósítja meg.
Ez alapján tetszőleges virtuális forrás-másodlagos forráseloszlás-referenciagörbe specifikus vezérlőfüggvények vezethetők be.\cite{Firtha2016_booklet}

Egyszerű példaként a \ref{fig:SFS_theory:25D_WFS_generals} ábra egy 3D pontforrás általános alakú másodlagos forráseloszlással történő szintézisét mutatja be.
A referenciagörbe egy koncentrikus kontúr a megfigyelési területen belül.
Az ábra (b) része alátámasztja, hogy a vezérlőfüggvényekkel a teljes megfigyelési területen fázishelyes szintézis érhető el (azaz a hullámfront alakja helyesen visszaállítható), míg az amplitúdóhelyes szintézis a referenciagörbe azon részére korlátozódik, amelyre rálátni az adott virtuális forráspozícióból.

\subsection{A tértartománybeli explicit megoldás}

Az \emph{explicit megoldás} célja a hangtér-reprodukciós inverz probléma közvetlen, direkt megoldása, a konvolúciós integrál spektrális tartományba való transzformációjával.
A spektrális, vagy hullámszámtartományban a konvolúció szorzattá transzformálódik és így a vezérlőfüggvény spektruma a virtuális hangtér és a másodlagos források terének a referenciagörbén mért spektrumainak hányadosaként áll elő.
A másodlagos forráseloszlást és a referenciagörbét és végtelen egyenesnek felvéve a módszerhez szükséges transzformáció egy egyszerű térbeli Fourier transzformáció, és a technikát a \emph{spektrális osztás módszerének (Spectral Division Method, SDM)} nevezzük.
Természetesen a vezérlőfüggvények közvetlenül alkalmazható, tértartománybeli alakja egy térbeli inverz Fourier transzformáció formájában adott, amely transzformáció azonban általában nem végezhető el analitikusan.

A stacionárius fázis módszer alkalmazásával aszimptotikus, nagyfrekvenciás közelítést adtam tetszőleges virtuális hang szintéziséhez az explicit vezérlőfüggvény térbeli alakjára \cite{Firtha2017:daga_booklet}.
A Kirchhoff integrál érvényessége mellett ezek a vezérlőfüggvények tetszőleges konvex hangszórósokaságra érvényesek (azt lokálisan lineárisnak feltételezve).
Fontos különbség a hangtérszintézishez képest, hogy az új explicit vezérlőfüggvény kiértékeléséhez a virtuális hangtér a referenciagörbén vett értékének szükséges az ismerete, míg a hangtérszintézis vezérlőfüggvények a virtuális tér a hangszórósoron mért jellemzői alapján számíthatók.

A Rayleigh-integrál teljes aszimptotikus kiértékelésének segítségével kifejeztem az új explicit vezérlőfüggvényeket a virtuális tér a másodlagos forráseloszláson vett értéke alapján, amely eredmény teljesen egybeesik a hangtérszintézis vezérlőfüggvényekkel.
Így tehát általánosan bizonyítottam, hogy a hangtérszintézis az explicit megoldás aszimptotikus, nagyfrekvenciás közelítését adja tetszőleges virtuális forrás, másodlagos forráseloszlás és referenciagörbe esetén \cite{Firtha2018:WFS_vs_SDM_booklet}.
A két módszer aszimptotikus egyezése a vonatkozó irodalomban eddig csak speciális virtuális forrásmodellek és hangszóróeloszlások esetére volt belátva \cite{Schultz2019:HOA_vs_WFS_booklet}.

\begin{figure}[t!]  
\small
  \begin{minipage}[c]{0.6\textwidth}
	\begin{overpic}[width = 1\columnwidth ]{figs/Antialiased_synth.png}
	\small
	\put(2,53){(a)}
	\put(2,1){(b)}
	\end{overpic}   \end{minipage}\hfill
	\begin{minipage}[c]{0.35\textwidth}
    \caption{
	Sávkorlátozott impulzus gerjesztésű 3D virtuális pontforrás szintézise diszkrét másodlagos forráseloszlással, a másodlagos források közötti távolságot $\Delta x = 10~\mathrm{cm}$-re választva.
	Az (a) ábra a forráseloszlás diszkretizálásának hatását mutatja be, ami átlapolódó visszhangok megjelenéséhez vezet a virtuális hullámfront mögött.
	A (b) ábra az átlapolódásgátló szűrés hatását mutatja.
	A szűrés eredményeképp egy kitüntetett irányban (az ábrán nyíllal jelölve) az átlapolódó hullámfrontok teljesen megszüntethetőek.
   % The arrow originates at the SSD element with no angular bandwidth limitation, i.e. performing full-band synthesis.
   }
\label{fig:SFS_theory:anti-aliased_synthesis}   \end{minipage}
\end{figure} 

\vspace{3mm}
A hangtérszintézis gyakorlati alkalmazása során az ideálisan folytonos másodlagos pontforráseloszlás helyett egymástól véges távolságra, diszkrét pozíciókban elhelyezett hangszórósokaságot alkalmazunk.
Az idealizált fizikai modell megsértéséből fakadóan a szintetizált térben járulékos hatások, ú.n. \emph{térbeli átlapolódási jelenségel} jelennek meg.
A jelenség az időtartományban az eredeti, virtuális hullámfrontot követő másodlagos, az egyes hangszóróelemekből származó gömbi hullámfrontok sorozataként jelentkezik.
Ezek az átlapolódó ,,visszhangok'', amelyek az egyes hangszóró vezérlőfüggvények az \emph{átlapolódási frekvencia} feletti komponenseit tartalmazzák, a \ref{fig:SFS_theory:anti-aliased_synthesis} (a) ábrán figyelhetők meg.

Az explicit hangtér-reprodukciós megoldás nagy előnye, hogy a fenti átlapolódási jelenségek a spektrális tartományban egyszerűen modellezhetőek, amely modell a módszerek ekvivalenciája miatt alkalmazható a hangtérszintézisre is.
Az explicit vezérlőfüggvények aszimptotikus vizsgálatával azonosítottam, mely másodlagos forráselemek járulnak hozzá a térbeli átlapolódáshoz egy adott frekvencián.
Adott frekvencián ezeket a forrásokat elnémítva a térbeli átlapolódás elkerülhető, a megfigyelési terület csökkenése árán.
Széles-sávú gerjesztés esetén a fenti átlapolódásgátló szűrés a vezérlőfüggvények egyszerű alul-áteresztő szűrésével valósítható meg.
Végeredményben tehát egyszerű alul-áteresztő szűrő alapú átlapolódásgátló szűrési stratégiát vezettem be, ahol a szűrő vágási frekvenciája a virtuális hangtér lokális hullámszámvektorán keresztül analitikusan adott.
A módszer alkalmazásával a megfigyelési terület egy kitüntetett iránya mentén átlapolódásmentes, teljes-sávú szintézis valósítható meg, ahogy az a \ref{fig:SFS_theory:anti-aliased_synthesis} (b) ábra mutatja.
A tér többi irányában a virtuális hullámfront sávkorlátozott a szűrés miatt, valamint oldalirányba terjedő átlapolódó hullámfrontok maradnak jelen a térben, amelyek már kizárólag irányított másodlagos forrásokkal nyomhatóak el.

A bevezetett módszer alternatívájaként a probléma megközelíthető pusztán a tértartományban, szintén a lokális hullámszámvektor alkalmazásával.
Ez a megközelítés alkalmazza a \cite{8611109_booklet}-ban leírt módszer, amely folyóiratcikknek társszerzője vagyok.

\subsection{Mozgó források szintézise}

A bevezetett elméleti keretrendszer alkalmazására szolgáló összetett példaként megvizsgáltam a mozgó hangforrások keltett tér reprodukciójának lehetőségeit.
Ebben az esetben az elsődleges feladat a \emph{Doppler-hatás} pontos visszaállítása, amely azonban eleve biztosítva van megfelelő analitikus forrásmodell alkalmazása esetén.

Munkám során kiterjesztettem a lokális hullámszámvektort fogalmát idő-variáns hangterekre is és a stacionárius fázis módszerrel mozgó források szintézisére alkalmas hangtérszintézis vezérlőfüggvényt vezettem be.
A vezérlőfüggvény lehetővé teszi egy tetszőleges (a megfigyelő síkjában) mozgó forrás szintézisét tetszőleges konvex másodlagos forráseloszlás alkalmazásával, az amplitúdóhelyes szintézist tetszőleges referenciagörbén megvalósítva \cite{firtha2016wave_booklet, doi:10.1121/1.4996126_booklet, Firtha2015:daga_booklet}.
A módszer alkalmazását a \ref{fig:SFS_theory:25D_WFS_moving_source} ábra szemlélteti, egy egyenletesen mozgó harmonikus forrás szintézisének példáján keresztül.
Látható, hogy a vezérlőfüggvények a reprodukált tér amplitúdóját az előírt referenciagörbére optimalizálják. 

\begin{figure}[t!]
\small
  \begin{minipage}[c]{0.64\textwidth}
	\begin{overpic}[width = 1\columnwidth ]{figs/25D_WFS_ms.png}
	\small
	\put(2,53){(a)}
	\put(2,1){(b)}
	\end{overpic}   \end{minipage}\hfill
	\begin{minipage}[c]{0.35\textwidth}
    \caption{
	Egyenletesen mozgó harmonikus, $f_0 = 1.5~\mathrm{kHz}$ frekvenciájú pontforrás szintézise általános hangszórógörbe alkalmazásával.
	A forrás sebessége $|\mathbf{v}| = 150~\mathrm{m/s}$.
	A referenciagörbe egy koncentrikus kontúr a megfigyelési területen.
	Az (a) ábra a szintetizált tér valós részét, a (b) ábra a szintézis abszolút hibáját mutatja be.
	A másodlagos forráseloszlás aktív részét a pillanatkép rögzítésének pillanatában folytonos, az inaktív részét szaggatott fekete vonal jelzi.
    }
\label{fig:SFS_theory:25D_WFS_moving_source}  \end{minipage}
\end{figure}

Az egyenes pályán egyenletesen mozgó források speciális esetében a forrás által lesugárzott tér spektrális, hullámszámtartomány-beli leírása analitikusan is a rendelkezésre áll.
Ez a leírás lehetővé tette explicit vezérlőfüggvények meghatározását mozgó források szintéziséhez \cite{Firtha2014:daga_booklet, Firtha2014:isma_booklet}.
Az így bevezetett vezérlőfüggvény referenciamegoldást szolgáltat a problémára, mivel az alkalmazásával közelítések nélküli, tökéletes reprodukció érhető el a referenciavonal mentén.
Ezen felül a hullámszámtartomány-beli leírás lehetővé teszi a térbeli átlapolódási jelenségek tárgyalását.
Dolgozatomban megmutatom, hogy a mozgó források szintézisét megvalósító hangtérszintézis vezérlőfüggvények---hasonlóan a statikus esethez---nagyfrekvenciás közelítését adják az explicit megoldásnak.

\begin{figure}
	\centering
	\begin{overpic}[width = 1\columnwidth]{figs/antialiased_synth_moving_source.png}
	\footnotesize
	\put(0, 0){(a)}
	\put(50,0){(b)}
	\end{overpic}
\caption{Általános, görbe pálya mentén mozgó forrás szintézise, $\Delta x = 10~\mathrm{cm}$ osztásközű, $R_{\mathrm{SSD}} = 2~\mathrm{m}$ sugarú kör alakú másodlagos forráseloszlással.
	A forrás sebessége állandó, $|\mathbf{v}| = \frac{3}{4}c$.
	Az (a) ábra a forrás diszkretizálás hatását szemlélteti, jól látható átlapolódási jelenségekkel.
	Az legnagyobb energiájú átlapolódó hullámok a mozgó forrás előtti másodlagos forrásokból származnak: itt az lokális, érzékelt frekvencia megnő a Doppler-hatás miatt.
	A (b) ábra a bevezetett átlapolódásgátló szűrés hatását mutatja be.}
	\label{Fig:Moving_sources:anti-aliased_synth_moving_source}
\end{figure}

Végül kutatásom célja a mozgó források szintézise esetén jelentkező átlapolódási jelenségek vizsgálata, valamint a bevezetett átlapolódásgátló szűrés erre az esetre való kiterjesztése volt.
Ebben az esetben az átlapolódási jelenségek a szintetizált tér fokozottan hallható torzulásához vezet, mivel az átlapolódó másodlagos hullámfrontok más Doppler-tolást szenvednek, mint az ideális, virtuális hullámfront.
Ennek eredményeképp a statikus esetben is fellépő \emph{hangszíneződésen} felül nemkívánatos frekvenciatorzítás is megjelennek a szintetizált térben \cite{firtha2016:daga_booklet}.

Harmonikus forrásgerjesztés esetén ez a frekvenciatorzítás új, nemkívánatos frekvenciakomponensekként jelenik meg, ahogy az már korábban is ismert volt a vonatkozó irodalomban.\footfullcite{Franck2007}
A szintetizált tér spektrális vizsgálata alapján analitikus formulát adtam ezekre az átlapolódó frekvenciakomponensek frekvenciájára, és rámutattam, hogy a másodlagos forráseloszlás alakjának megfelelő megválasztásával a frekvenciatorzítás hatása jelentősen csökkenthető \cite{firtha2016:daga_booklet}.
Az átlapolódás elkerülése szempontjából az optimális választás köralakú hangszórósokaság alkalmazása.

Végezetül kiterjesztettem a korábban bevezetett átlapolódás elleni stratégiát mozgó források szintézisére is.
A javasolt módszerrel az átlapolódás a hangszóró vezérlőfüggvények egyszerű aluláteresztő szűrésével megvalósítható.
A szűrő idő-variáns vágási frekvenciája analitikusan ismert, a mozgó forrás lokális hullámszámvektorának függvényében \cite{Firtha2018_daga_moving_source_booklet}.
Hasonlóan a statikus esethez, az átlapolódásmentes, teljes-sávú szintézis csupán egy kitüntetett irányba valósítható meg.
Kör alakú hangszóróeloszlás esetén ez mindig a radiális irány, így a hangszórótömb közepén tetszőeges időpillanatban tetszőleges forráspálya mellett optimális szintézis érhető el
Ezt az esetet szemlélteti a \ref{Fig:Moving_sources:anti-aliased_synth_moving_source} ábra egy általános görbe pálya mentén mozgó impulzussorozat-gerjesztésű forrás szintézise esetén.

\clearpage	
\section{Tézisek}
\label{sec:theses}

\begin{thesisgroup}[A hangtérszintézis elméletének általánosítása]
Bevezettem egy általánosított hangtérszintézis elméletet, amely lehetővé teszi egy tetszőleges virtuális hangtér szintézisét tetszőleges alakú konvex hangszóróeloszlás alkalmazásával és a szintézis amplitúdója tetszőleges alakú referenciagörbére optimalizálható.
Az általánosított elmélet speciális esetekként magában tartalmazza  a korábbi hangtérszintézis módszereket \cite{Firtha2016_booklet}.
\begin{thesis}
Szemléletes fizikai interpretációt adtam peremintegrálok stacionárius fázisú közelítésének:
Bevezettem a tetszőleges harmonikus hangtérre értelmezett lokális hullámszámvektort, amely felhasználásával megmutattam, hogy peremintegrál stacionárius fázis közelítése a megfigyelési pozícióban virtuális hangtér és a perem mentén elhelyezett másodlagos források terének hullámfront-illesztéseként értelmezhető.
\end{thesis}
\begin{thesis}
Hangtérszintézis vezérlőfüggvényeket vezettem be tetszőleges konvex másodlagos forráseloszlásra a fenti fizikai magyarázat alapján nagyfrekvenciás feltételezések mellett.
\end{thesis}
\begin{thesis}
Analitikus összefüggést adtam a referenciagörbe általános alakjára, amely görbe azon pontok összessége, amelyekben az szintézis amplitúdóhibája minimális.
Bemutattam, hogyan befolyásolható ezen referenciagörbe alakja egy egyszerű, frekvenciafüggetlen amplitúdó tényezővel a vezérlőfüggvényben.
Bemutattam, hogy a korábbi hangtérszintézis módszerek milyen speciális esetei az általánosított elméletnek virtuális forrás-másodlagos eloszlás-referenciagörbe szempontjából.
\end{thesis}
\end{thesisgroup}

\begin{thesisgroup}[Térbeli explicit megoldás és ekvivalencia a hangtérszintézissel]
A hangtérszintézis mellett---amely célja egy olyan kontúrintegrál megfogalmazása amely a keresett hangszóró vezérlőfüggvényeket implicite tartalmazza---egyszerű geometriák esetén explicit vezérlőfüggvények számítása is lehetséges a probléma térbeli frekvenciatartományba transzformációja alapján.
Egyenes hangszóróeloszlás esetén az explicit vezérlőfüggvény a virtuális tér és a másodlagos forráselemek fix referenciavonalon mért spektrális hányadosának térbeli inverz Fourier transzformáltjaként áll elő.
Az implicit és explicit módszerek közötti kapcsolatot a korábbiakban csak speciális virtuális hangterekre vizsgálták.
Az explicit megoldás aszimptotikus közelítésével nagyfrekvenciás térbeli explicit vezérlőfüggvényeket vezettem be, és rámutattam az implicit és explicit megoldások általános nagyfrekvenciás ekvivalenciájára \cite{Firtha2017:daga_booklet, Firtha2018:WFS_vs_SDM_booklet}
\begin{thesis}
Tértartománybeli explicit vezérlőfüggvényeket vezettem be a térbeli Fourier transzformáció stacionárius fázisú közelítésével.
Az új vezérlőfüggvények kiértékeléséhez a virtuális hangtér jellemzőit a tetszőlegesen előírt referenciagörbén szükséges ismerni \cite{Firtha2017:daga_booklet}.\end{thesis}
\begin{thesis}
Bebizonyítottam, hogy nagyfrekvenciás feltételek mellett a explicit megoldás és a hangtérszintézis (az implicit megoldás) aszimptotikusan megegyezik tetszőleges virtuális hangtér esetére \cite{Firtha2018:WFS_vs_SDM_booklet}.
A bizonyítás alapja a térbeli explicit vezérlőfüggvények kifejezése a virtuális forrás a másodlagos forráseloszláson mért jellemzői alapján.
\end{thesis}
\begin{thesis}
Egyszerű átlapolódásgátló stratégiát mutattam be a másodlagos forráseloszlás diszkretizálásából származó átlapolódó hullámfrontok csillapítása érdekében.
A módszer a térbeli Fourier transzformáció aszimptotikus közelítésén alapszik.
A bemutatott stratégia a vezérlőfüggvények egyszerű aluláteresztő szűrésével implementálható \cite{Firtha2018_daga_a_booklet}.
\end{thesis}
\end{thesisgroup}

\begin{thesisgroup}[Mozgó források hangtérszintézise]
Dinamikusan változó akusztikai környezetek szintézise szempontjából a mozgó források terének megfelelő szintézise alapvető feladat.
Kiterjesztettem az itt bemutatott hangtérszintézis elméletet mozgó források szintézisére is.
\begin{thesis}
Kiterjesztettem a 3D hangtérszintézis elméletet tetszőleges pályán, tetszőleges sebességprofillal mozgó 3D pontforrások szintézisére.
A megoldás a Doppler-hatást analitikusan figyelembe veszi \cite{Firtha2015:daga, firtha2016wave_booklet, doi:10.1121/1.4996126_booklet}.\end{thesis}
\begin{thesis}
2.5D hangtérszintézis vezérlőfüggvényeket határoztam meg egy tetszőleges alakú hangszórógörbére, amellyel egy tetszőleges---a hangszórók síkjában---mozgó 3D pontforrás tere szintetizálható, az amplitúdóhelyes szintézist tetszőleges referenciagörbére megvalósítva \cite{doi:10.1121/1.4996126_booklet}.
A megoldás alapja a stacionárius fázis módszer kiterjesztése a jelen dinamikus esetre.
Rámutattam, hogy lineáris hangszórósor és ezzel párhuzamos referenciavonal esetén az általános megoldás a tradicionális pontforrás vezérlőfüggvényeket eredményezi a statikus forráspozíciót az emissziós időben felvett pozícióra cserélve \cite{doi:10.1121/1.4996126_booklet}.
\end{thesis}
\begin{thesis}
Zárt analitikus vezérlőfüggvény adtam egyenes vonalon egyenletesen mozgó pontforrás szintézisére, amely esetben a vezérlőfüggvényben jelen lévő emissziós idő, illetve terjedési késleltetések explicite kifejezhetőek \cite{firtha2016wave_booklet}.
\end{thesis}
\begin{thesis}
2.5D hangtérszintézis vezérlőfüggvényeket határoztam meg egyenes hangszórósorra a frekvenciatartományban mozgó forrás spektrális leírásának stacionárius fázisú közelítésével \cite{firtha2015sound_booklet}.
\end{thesis}
\end{thesisgroup}

\begin{thesisgroup}[Explicit megoldás mozgó források szintézisére]
Analitikus kifejezést adtam egyenes vonal mentén egyenletesen mozgó pontforrás terének térbeli Fourier transzformáltjára.
Mivel az explicit megoldás nincs időinvariáns terekre korlátozva, ezért a kifejezés felhasználható explicit vezérlőfüggvények számítására mozgó forrásokhoz.
\begin{thesis} 
Explicit vezérlőfüggvényeket határoztam meg egyenletesen mozgó források szintézisére a hullámszámtartományban.
A másodlagos forráseloszlással párhuzamosan mozogó források speciális esetében az explicit vezérlőfüggvényt analitikusan meghatároztam a tér-frekvenciatartományban is \cite{Firtha2014:daga_booklet, Firtha2014:isma_booklet, firtha2015sound_booklet}.
A frekvenciatartománybeli hangtérszintézis vezérlőfüggvények alapján rámutattam, hogy a statikus esethez hasonlóan, az implicit megoldás az explicit megoldásnak a nagyfrekvenciás, aszimptotikus közelítését adja \cite{firtha2015sound_booklet}. 
\end{thesis}
\begin{thesis}
Analitikus leírását adtam a térbeli átlapolódási jelenségeknek mozgó források szintézise esetére.
Rámutattam, hogy a nemkívánatos frekvenciakomponensek a Green-függvény spektrumában jelenlévő szinguláris pontokhoz köthetők, és ez alapján analitikus összefüggést adtam ezen nemkívánatos összetevők frekvenciájának meghatározására.
Bemutattam, hogy ezek a hatások csökkenthetőek megfelelően megválasztott másodlagos forráseloszlás alkalmazásával \cite{firtha2016:daga_booklet}.
\end{thesis}
\begin{thesis} 
Kiterjesztettem a térbeli átlapolódásgátló-szűrési stratégiát tetszőleges trajektórián mozgó források esetére is.
A javasolt módszerrel az átlapolódó hullámfrontok egyszerű idővariáns aluláteresztő szűréssel kiküszöbölhetőek \cite{Firtha2018_daga_moving_source_booklet}.
\end{thesis}
\end{thesisgroup} 
\clearpage

\section*{Publikációs lista}
\addcontentsline{toc}{sec}{Publikációk}

\nocite{*}
\begin{refcontext}[labelprefix=J]
\printbibliography[title={Folyóiratcikkek}, keyword=J, heading=subbibliography] 
\end{refcontext}
\begin{refcontext}[labelprefix=C]
\printbibliography[title={Konferencia közlemények}, keyword=C, heading=subbibliography] 
\end{refcontext}
\begin{refcontext}[labelprefix=O]
\printbibliography[title={Egyéb publikációk}, keyword=O, heading=subbibliography] 
\end{refcontext}

\end{document}
