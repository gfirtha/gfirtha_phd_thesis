	\documentclass[10pt,twoside]{article}
\usepackage[a5paper,inner=1.5cm,outer=2.0cm,top=2.0cm,bottom=2.5cm,pdftex]{geometry}
\usepackage[utf8]{inputenc}						% UTF-8 encoding
\usepackage[T1]{fontenc}
\usepackage[magyar,american]{babel}
\usepackage{csquotes}
\usepackage[backend=biber,
			hyperref=true,
			autolang=hyphen,
			style=numeric-comp,
			isbn=false,
			sorting=none,
			defernumbers=true,
			maxnames=10,
			doi=false,
			url=false,
			useprefix=true,
			]{biblatex}
\usepackage{palatino}							% Font
\usepackage{amsmath, amssymb, amsthm} 							% Theorem customization
\usepackage{graphicx,subfigure,overpic}
\usepackage{rotating}
\usepackage{multirow}
\usepackage{xcolor}
\usepackage[bf,small]{caption}
\addbibresource{../LatexProject/dissertation.bib}
\addbibresource{booklet.bib}

\newcount\posveccount
\newcommand*\posvec[1]{
        \global\posveccount#1
        [
        \posvecnext
}
\def\posvecnext#1{
        #1
        \global\advance\posveccount-1
        \ifnum\posveccount>0
                ,\
                \expandafter\posvecnext
        \else
                ]^{\mathrm{T}}
        \fi
} 
\newcommand{\vxs}{\mathbf{x}_{\mathrm{s}}}
\newcommand{\vxref}{\mathbf{x}_{\mathrm{ref}}}
\newcommand{\vk}{\mathbf{k}} 

\usepackage[pdftitle={A Generalized Wave Field Synthesis Theory (Ph.D. Thesis Booklet)},%
  pdfauthor={Gergely Firtha},% 
%  colorlinks=true,% 
%  urlcolor=blue,% 
%  linkcolor=blue,% 
%  citecolor=blue,% 
%  filecolor=green,% 
%  raiselinks=false,% 
  hyperfootnotes=true]{hyperref}

% Citation categories
\DeclareBibliographyCategory{journal} 
\DeclareBibliographyCategory{conference}
\DeclareBibliographyCategory{tech}
\DeclareBibliographyCategory{other}
% Citation commands
\newcommand*{\citeJ}[2][]{\addtocategory{journal}{#2}\cite[#1]{#2}}
\newcommand*{\citeC}[2][]{\addtocategory{conference}{#2}\cite[#1]{#2}}
\newcommand*{\citeO}[2][]{\addtocategory{other}{#2}\cite[#1]{#2}}
\newcommand*{\citeT}[2][]{\addtocategory{tech}{#2}\cite[#1]{#2}}
% No citation commands
\newcommand*{\nociteJ}[1]{\addtocategory{journal}{#1}\nocite{#1}}
\newcommand*{\nociteC}[1]{\addtocategory{conference}{#1}\nocite{#1}}
\newcommand*{\nociteO}[1]{\addtocategory{other}{#1}\nocite{#1}}
\newcommand*{\nociteT}[1]{\addtocategory{tech}{#1}\nocite{#1}}

\nociteJ{firtha2015sound,firtha2016wave_booklet,Firtha2016_booklet,doi:10.1121/1.4996126_booklet,schultz2017wave_booklet,Firtha2018:WFS_vs_SDM_booklet}
\nociteC{Firtha2012:isma_booklet,Firtha2013:daga_booklet,Firtha2013:internoise_booklet,Firtha2014:daga_booklet,Firtha2014:isma_booklet,
Firtha2015:daga_booklet,firtha2016:daga_booklet,Firtha2017:daga_booklet,Firtha2018_daga_a_booklet,Firtha2018_daga_moving_source_booklet}

\renewcommand{\bibfont}{\normalfont\small}
\renewcommand*{\mkbibnamegiven}[1]{%
\ifitemannotation{highlight}
{\textbf{#1}}
{#1}}



\renewcommand*{\mkbibnamefamily}[1]{%
\ifitemannotation{highlight}
{\textbf{#1}}
{#1}}

\newtheoremstyle{thesisgroupstyle}%
	{1em}% 				% Space above
	{5mm}% 				% Space below
	{\slshape}%			% Font type of body
	{0mm}%				% Indentation			
	{\bfseries}%		% Font of title
	{\newline}%			% Punctuation between head and body
	{1em}%				% Space after theorem head
	{}% 				% Manually specify head

\makeatother
\theoremstyle{thesisgroupstyle}
\newtheorem{thesisgroup}{Thesis group}
\renewcommand{\thethesisgroup}{\Roman{thesisgroup}}

\makeatletter
\newtheoremstyle{indented}
  {5pt}% space before
  {5pt}% space after
  {\addtolength{\@totalleftmargin}{3.5em}
   \addtolength{\linewidth}{-3.5em}
   \parshape 1 3.5em \linewidth}% body font
  {}% indent
  {\bfseries}% header font
  {.}% punctuation
  {.75em}% after theorem header
  {}% header specification (empty for default)
\makeatother
\theoremstyle{indented}
\newtheorem{thesis}{Thesis}[thesisgroup]


\newcommand{\theauthor}{Gergely Firtha}
\newcommand{\authorprof}{M.Sc.E.E.}
\author{\theauthor}
\title{A Generalized Wave Field Synthesis Theory\\[.5ex]with Application for Moving Virtual Sources}

\assignrefcontextkeyws[labelprefix=J]{J}
\assignrefcontextkeyws[labelprefix=C]{C}
\assignrefcontextkeyws[labelprefix=O]{O}

% Add underline for my author name
\def \mylastname {Firtha}
\newbibmacro{name:underlined}{%
    \ifthenelse{\equal{\namepartfamily}{\mylastname}}%
{\underline{\ifblank{\namepartgiven}{}{\namepartgiven\space}\namepartfamily}}%
{\ifblank{\namepartfamily}{}{\namepartgiven\space}\namepartfamily}%
    \ifthenelse{\value{listcount}<\value{liststop}}%
    {\addcomma\space}%
    {}%
}

\DeclareNameFormat{author}{%
    \nameparts{#1}% split the name data, will not be necessary in future versions
    \usebibmacro{name:underlined}%
    \usebibmacro{name:andothers}%
}


\begin{document}
\sloppy
\newcommand{\schoolname}{Budapest University of Technology and Economics}
\newcommand{\facultyname}{Faculty of Electrical Engineering and Informatics}
\newcommand{\doctoralname}{Doctoral School of Electrical Engineering}
\newcommand{\bookname}{Ph.D.\ Thesis Booklet}
\newcommand{\authorname}{Author}
\newcommand{\supervisorname}{Supervisor}
\newcommand{\supervisor}{Péter Fiala, PhD.}
\newcommand{\creationdate}{Budapest, 2018.}
\def \deptlogo {logos/hit_logo_en.png}
\def \lablogo {logos/last_logo_eng.png}

\selectlanguage{american}
\nonfrenchspacing

%\thispagestyle{empty}
\newcommand{\HRule}{\rule{\linewidth}{1pt}}

\begin{titlepage}
\begin{center}

%%% Header with university logo
\includegraphics[height=1.2cm]{logos/BME1782.pdf}
\HRule\par
{\small \textsc{ \bfseries \schoolname \\
\facultyname \\
\doctoralname}}\par

\vfill

%%% Inserting the title
{
\Large
\makeatletter
\@title
\makeatother
} \par
\bigskip
\bookname \par
\bigskip
{
\normalsize
\theauthor\\[.5ex]
\authorprof
}\par

\vfill
%%% Inserting author, supervisor and co-promotor
\begin{tabular}{@{}p{3cm}l@{}}
	\supervisorname & \supervisor \\
\end{tabular}

\vspace{\fill}

\hspace{1.0cm} \includegraphics[height=1cm]{\deptlogo} \hspace{2cm}
\includegraphics[height=1cm]{\lablogo} \par

\vspace*{\stretch{0.33}}
%%% Inserting date
\normalsize
\creationdate



\end{center}


\end{titlepage}

\thispagestyle{empty}
\cleardoublepage
\setcounter{page}{1}

\section{Introduction}

\subsection{Aim of sound field synthesis}
%
Spatial audio aims at the recreation of a sound scene containing sources of sound, termed as audio objects, in a sense that the human listener perceives the spatial characteristics of the desired acoustic environment.
Sound field reproduction achieves this by driving an arrangement of fix positioned loudspeakers so that the superposition of the sound waves emerging from the individual loudspeakers generate the impression of the desired virtual audio object present in the sound scene.
Sound field reproduction has been the subject of excessive study and development over the second half of the XX. century, starting with the work of Blumlein, who introduced the first two-loudspeaker system in 1931 and thereby created the basics of stereophony.
Modern stereophonic systems include the well-known Dolby stereo, 5.1, 7.1 systems, the 22.2 system of the NHK or the currently state-of-the-art commercial spatial audio systems, the Dolby Atmos and the DTS-X.
Generally speaking---independently from the number of the speakers applied---stereophony generates the desired spatial impression by the recreation of localization cues at the listener position.
Correct sound localization, therefore, can be ensured only over a limited listening area termed as the \emph{sweet spot}, being a central limitation of these techniques.

Opposed to stereophony, the aim of sound field synthesis (SFS) is the reproduction of the physical properties of a desired sound field over an extended listening area. 
Once its is achieved, it is inherently ensured that the listener perceives the desired perceptual properties in an arbitrary position within the listening region.
Obviously, controlling the sound field over an extended region requires numerous loudspeakers, positioned on the boundary of the control region.
Hence, these techniques are often referred to as \emph{massive multichannel sound reproduction methods}.\footfullcite{Spors2013:Survey}
The central question of SFS is the derivation of the loudspeaker driving signals, termed the \emph{driving function}, so that the resultant field of the loudspeaker arrangement coincides with the desired \emph{virtual field}.
The loudspeaker array applied for reproduction is termed as the \emph{secondary source distribution} in the followings.
The general geometry is depicted in Figure \ref{fig:introduction:sfs_aim}.

\begin{figure}  
\small
  \begin{minipage}[c]{0.64\textwidth}
	\begin{overpic}[width = 1\columnwidth ]{figs/sfs_aim.png}
	\small
	\put(60,30){listening region}
	\put(0,45){\parbox{.5in}{virtual sound source}}
	\put(45,7){secondary source distribution}
	\end{overpic}   \end{minipage}\hfill
	\begin{minipage}[c]{0.3\textwidth}
    \caption{The general geometry for sound field synthesis: the goal of synthesis is to reproduce the physical properties of a virtual sound object or primary source of sound inside a control region bounded by a densely spaced loudspeaker ensemble.}
\label{fig:introduction:sfs_aim}  \end{minipage}
\end{figure}

Over the last several decades numerous SFS techniques emerged, approaching the problem both numerically and analytically.
Analytical approaches include the direct solution of the involved integral formulations termed as the explicit solution, and the most traditional SFS technique deriving driving function based on the Huygens principle, termed as Wave Field Synthesis.

\subsection{Wave Field Synthesis history}
%
The original theory of Wave Field Synthesis---often referred to as \emph{traditional WFS}---evolved from the works of Berkhout et al. at the Technical University of Delft by utilizing concepts well-known in the field of seismic migration.
%, applied for sound field control.
The basis of WFS theory were the Rayleigh integrals, the mathematical form of the Huygens' principle, representing a sound field as the sum of spherical waves emerging from an infinite plane.
Berkhout applied the stationary phase approximation (SPA) to the Rayleigh integrals in order to arrive at loudspeaker driving signals for a linear array of loudspeakers instead of the practically infeasible planar array.
The original formulation provided driving signals for loudspeakers with dipole characteristics\footfullcite{Berkhout1988}\textsuperscript{,}\footfullcite{Berkhout1993:Acoustic_control_by_WFS}---soon extended for monopole loudspeakers as well \footfullcite{doi:10.1121/1.404755}\textsuperscript{,}\footfullcite{Start1997:phd}\textsuperscript{,}\footfullcite{Verheijen1997:phd}---reproducing the wavefront of a virtual spherical wave in the horizontal plane containing the loudspeaker array.
It was discussed that reducing the secondary source distribution from an enclosing surface to a more practical contour of loudspeakers results that amplitude correct reproduction is restricted to a control curve in the plane of synthesis, termed here as the reference curve.
For traditional WFS this reference curve was usually chosen to be a reference line parallel with the loudspeaker arrangement.\footfullcite{Start1997:phd}
Furthermore, the traditional approach considered exclusively virtual point sources as the virtual field model.

Traditional WFS was also the subject of various research projects, most notably the CARROUSO project, aiming at the integration of the technique into the MPEG-4 standard. This endeavor was not realized eventually, but two "spin-off" companies of the project, the IOSONO and Sonic Emotion are still offering commercially available WFS systems nowadays.

The latest milestone in Wave Field Synthesis theory were the works of Spors et al., generalizing WFS towards the synthesis of an arbitrary analytically available sound field by applying an arbitrary shaped loudspeaker contour.\footfullcite{Spors2008:WFSrevisited}
The presented loudspeaker driving signals allowed the synthesis of general 2-dimensional sound fields, ensuring amplitude correct synthesis at a single reference point.
The approach allows the reproduction of complex virtual sound scenes, e.g. the field generated by a moving sound source.\footfullcite{Ahrens2008moving}
The method, however---since it derived driving signals from the 2D Rayleigh integral---failed to control the amplitude of general 3D sound fields.
Furthermore, the exact connection between the traditional and the latter \emph{revisited WFS} formulations has not been known so far.

\subsection{Motivation of the present research}
The present dissertation revisits the theoretical basics of Wave Field Synthesis.
The motivation behind the research was to give a generalized theoretical framework for WFS that allows the reproduction of an arbitrary analytically available virtual sound field by applying an arbitrary secondary source contour and optimizing the synthesis to an arbitrary reference curve.
Hence, the presented framework includes the previous WFS approaches as special cases: choosing the virtual source to a point source, applying a linear SSD and referencing the synthesis to a line, parallel with the SSD returns the well-known traditional WFS driving function.
Choosing the virtual source to be a general 2D sound field and referencing the synthesis to a single reference point results in the revisited WFS formulation.

Besides the connection between traditional and revisited WFS, also the exact link between WFS and the direct, explicit solution has not been investigated in the related literature for general virtual sound fields.
For the special case of circular SSDs and specific virtual field models it has been shown that WFS constitutes a high frequency approximation for the explicit method.\footfullcite{Ahrens2012}\textsuperscript{,}\footfullcite{Spors2010:analysis_and_improvement}	
A further motivation of the present research was to establish this connection for arbitrary virtual field models and arbitrary SSD and reference contours.

As a complex application example for the presented framework, the reproduction of moving virtual sources was investigated in details.
The reproduction of moving sources has been the subject of studies since the early age of WFS theory as an obvious need when dynamic sound scenes are to be synthesized.
Early formulations attempted to synthesize the field of a moving point source by applying the traditional WFS driving signals with changing the virtual source position as the function of time.
This approach, however, failed to properly recreate the Doppler effect, leading to serious artifacts in the synthesized field.
Ahrens et al. used the revisited WFS formulation in order to recreate the field of a moving source.\footfullcite{Ahrens2008moving}\textsuperscript{,}\footfullcite{Ahrens2008moving_b}
However, due to physical constrains of revisited WFS theory, it failed to control the amplitude of the synthesized source.
Hence, a further research goal was to extend the presented framework to include the synthesis of dynamic sound scenes.

\section{Methodology}

This section briefly summarizes the methodology and analysis tools applied throughout the thesis.

The derivation of novel, generalized Wave Field Synthesis and spatial domain explicit driving functions (allowing the comparison with WFS) requires the asymptotic evaluation of convolutional and spectral integrals.
First, I introduced local wavefield properties---the local wavenumber vector and the local wavefront curvature---later serving as powerful tools in order to give a meaningful physical interpretation for the results concerning integral approximations.
These quantities were well-known in the field of ray acoustics and high frequency acoustics, but have not been introduced in the context of sound field synthesis so far.

The central mathematical method, forming the backbone of the present thesis, is the \emph{stationary phase approximation} (SPA). 
The method is applied frequently in asymptotical analysis in order to approximate integrals around their critical points, termed as the \emph{stationary point}.
Loosely speaking, the main topic of the present dissertation is the discussion how the SPA can be applied 
\begin{itemize}
\item to boundary integral representations of the virtual sound field, allowing the extraction of WFS driving function 
\item to spectral integral representations, in order to derive novel, spatial domain explicit driving functions and to connect spectral coefficients to particular positions in space.
\end{itemize}
The validity of the analytical results were verified via the numerical simulation of the synthesized field.
Throughout the thesis these simulations were performed by the simple summation of the analytically available field of the individual SSD elements (given by the Green's function), weighted by the driving function and assuming free field conditions.

\section{Results}
This section briefly presents the most important results discussed in the thesis. 
A short overview of the results is also given in \cite{Firtha2019:daga_booklet}.
The scientific achievements behind the results presented in this section are summarized as theses in Section \ref{sec:theses}.

\subsection{A generalized Wave Field Synthesis framework}
The general SFS geometry is depicted in Figure \ref{fig:introduction:sfs_aim}.
The synthesized field can be given mathematically as a convolutional integral over the SSD contour in which the weighting factor is the \emph{driving function} to be expressed and the kernel is the Green's function describing the SSD elements' sound field.
The general SFS problem is, therefore, an inverse problem.

Wave Field Synthesis approaches the problem by finding an appropriate boundary integral representation of the virtual sound field, from which the required driving function can be extracted.
This representation can be found by reducing the surface integral describing an arbitrary 3D sound field into a contour integral by a vertical stationary phase approximation.

Traditional WFS started out from the Rayleigh integral, which, however, inherently restricted the SSD to be a linear one.
A suitable choice for the boundary integral representation is the Kirchhoff approximation, representing arbitrary two- or three-dimensional sound fields in terms of a single layer potential under high frequency conditions.
I reduced the Kirchhoff approximation of an arbitrary 3D sound field by the SPA into a contour integral.

\begin{figure}[t!]
\small
  \begin{minipage}[c]{0.64\textwidth}
	\begin{overpic}[width = 1\columnwidth ]{figs/25D_WFS_general.png}
	\small
	\put(2,53){(a)}
	\put(2,1){(b)}
	\end{overpic}   \end{minipage}\hfill
	\begin{minipage}[c]{0.35\textwidth}
    \caption{2.5D synthesis of a 3D point source located at $\vxs = \posvec{3}{0.4}{2.5}{0}$, radiating at $f_0 = 1.5~\mathrm{kHz}$.
    Figure (a) depicts the real part of the synthesized field, (b) presents the absolute error of synthesis in a logarithmic scale.
	The active arc of the SSD is denoted by solid black line and the inactive part with dotted by black line.
	The reference position on the reference curve for each active SSD element is evaluated numerically.
	Obviously, in the present geometry there exist secondary sources for which no unique reference position can be found.
	}
\label{fig:SFS_theory:25D_WFS_generals}   \end{minipage}
\end{figure}  

The dimensionality reduction restricts the amplitude correct synthesis to a single receiver position, termed as the \emph{reference position}.
Revisited WFS allowed the synthesis of 2D virtual fields, ensuring amplitude correct synthesis only in this reference position.
I showed that by a further, horizontal application of the SPA a unique reference position can be assigned to each infinitesimal SSD element.
By applying the local wavenumber concept---being a vector pointing into the local propagation direction of the sound field---a simple analytical formulation can be given for these reference positions for each secondary source.
I showed that the direction at which the reference position lies from the actual SSD element is determined by the local wavenumber vector of the virtual sound field.
The distance between the SSD element and its reference position, however, can be manipulated by applying a simple frequency independent correction term.
The set of all SSD elements reference position forms the \emph{reference curve}.
By prescribing a desired reference curve, the corresponding amplitude correction factor can be derived analytically and a virtual source-SSD shape-reference curve specific driving function can be expressed that ensures approximately amplitude correct synthesis over the reference curve.\cite{Firtha2016_booklet}

As an example, the synthesis of a 3D point source applying an arbitrary SSD contour is presented in Figure.\ref{fig:SFS_theory:25D_WFS_generals}
The reference curve is chosen to be a concentric contour inside the SSD.
As Figure (b) depicting the error of synthesis verifies, inside the entire listening region phase correct synthesis is achieved, while amplitude correct synthesis is ensured over that part of the reference curve to which an active SSD element can be found.

\subsection{Explicit solution in the spatial domain}
The explicit solution for the general sound field synthesis problem aims at the direct solution of the inverse problem by transforming the convolution integral describing the synthesized field into the spectral (wavenumber) domain.
In the wavenumber domain, the driving function can be expressed as the ratio of the virtual field and the SSD elements' field measured over a reference contour.
By choosing the SSD and the reference contour to be infinite parallel lines, the required transform is given by a simple spatial Fourier transform and the method is referred to as the \emph{Spectral Division Method } (SDM).
The spatial driving function is then obtained by a corresponding inverse Fourier transform, that's result is, however, barely available analytically.

I applied the stationary phase approximation in order to approximate the SDM driving functions asymptotically, resulting in novel, merely spatial explicit driving function \cite{Firtha2017:daga_booklet}.
Under the validity of the Kirchhoff approximation the derived driving function is valid for arbitrary SSD and reference contours.
Unlike WFS driving functions, which require the virtual field's properties measured over the SSD, the explicit driving functions can be evaluated as the ratio of the virtual field and the stationary SSD element's field \emph{measured over the reference curve}.

By approximating the virtual field on the reference curve by the asymptotical Rayleigh integral, I expressed the novel driving functions in terms of the virtual field measured on the SSD.
The derivation resulted in the generalized WFS driving function.
Hence, I proved that the introduced WFS framework constitutes a high frequency approximation of the explicit solution for arbitrary virtual fields \cite{Firtha2018:WFS_vs_SDM_booklet}.
This fact has already been proven for specific SSD shapes and virtual source models in the related literature even by the present author \cite{Schultz2019:HOA_vs_WFS_booklet}.

\begin{figure}[t!]  
\small
  \begin{minipage}[c]{0.6\textwidth}
	\begin{overpic}[width = 1\columnwidth ]{figs/Antialiased_synth.png}
	\small
	\put(2,53){(a)}
	\put(2,1){(b)}
	\end{overpic}   \end{minipage}\hfill
	\begin{minipage}[c]{0.35\textwidth}
    \caption{2.5D synthesis of a 3D point source emitting a bandlimited impulse, applying an arbitrary shaped discrete SSD with the secondary source spacing being $\Delta x = 10~\mathrm{cm}$.
    Figure (a) shows the effect of the discrete SSD resulting in aliasing echoes following the intended wavefront.
    Figure (b) shows the result of the proposed anti-aliasing filtering.
    As a result, anti-aliased synthesis may be achieved behind the virtual wavefront into the particular direction denoted by dashed arrow.
    The arrow originates at the SSD element with no angular bandwidth limitation, i.e. performing full-band synthesis.}
\label{fig:SFS_theory:anti-aliased_synthesis}   \end{minipage}
\end{figure} 

\vspace{3mm}
In practical applications the SSD is realized by a densely spaced loudspeaker ensemble with the source elements positioned at discrete locations.
The violation of the continuous SSD assumption leads to severe artifacts in the synthesized field, commonly referred to as \emph{spatial aliasing phenomena}. 
Spatial aliasing manifests in a series of echoes---each produced by one individual secondary source element---following the intended virtual wavefront carrying the driving function of the individual secondary sources high-pass filtered above the \emph{aliasing frequency}.

An advantage of the explicit solution is that it allows the analytical description of aliasing artifacts which can be directly applied to the results of WFS as well, due to the presented asymptotic equivalence of the two methods.
By the asymptotic evaluation of the SDM driving functions I identified those SSD elements that contribute to spatial aliasing artifacts on a given angular frequency.
By muting these secondary sources spatial aliasing can be avoided at the cost of reducing the listening region.
For wideband excitation signals the above anti-aliasing strategy can be implemented by temporal low-pass filtering of the loudspeaker driving function.
As a result of my research, I gave the analytical, virtual source model dependent cut-off frequency of this ideal anti-aliasing low-pass filter.
Applying my anti-aliasing strategy allows a perfect, full-band synthesis into a particular direction.
In other directions the intended virtual wavefront is low-pass filtered and aliasing wavefronts are still present which can be avoided only by applying directive secondary sources \cite{Firtha2018_daga_a_booklet, Firtha2012:isma_booklet}.

Alternatively to my solution, my theoretical framework allows the direct spatial investigation of the spatial aliasing artifacts.
This possibility is discussed in \cite{8611109_booklet} for which I serve as a co-author.

\subsection{Synthesis of moving sources}
\begin{figure}[t!]
\small
  \begin{minipage}[c]{0.64\textwidth}
	\begin{overpic}[width = 1\columnwidth ]{figs/25D_WFS_ms.png}
	\small
	\put(2,53){(a)}
	\put(2,1){(b)}
	\end{overpic}   \end{minipage}\hfill
	\begin{minipage}[c]{0.35\textwidth}
    \caption{2.5D synthesis of a moving 3D point source radiating at $f_0 = 1.5~\mathrm{kHz}$ applying an arbitrary shaped SSD.
    The source is under uniform motion with $|\mathbf{v}| = 150~\mathrm{m/s}$. 
    The synthesis is referenced on a concentric reference contour denoted by white dots.
    Figure (a) depicts the real part of the synthesized field and part (b) presents the absolute error of synthesis in a logarithmic scale.
	The active arc of the SSD \emph{at the time instant $t = 0~\mathrm{s}$} is denoted by solid black line, and the inactive part with dotted by black line.
    }
\label{fig:SFS_theory:25D_WFS_moving_source}  \end{minipage}
\end{figure}

As a complex application example for my theoretical framework, I investigated the possibilities for the synthesis of a moving virtual source.
For this dynamic case the primary challenge is the proper reconstruction of the \emph{Doppler effect}, occurring due to the constant, finite wave propagation velocity in a homogeneous medium.

By adapting the stationary phase approximation and the local wanumber concept to time-variant harmonic fields I derived WFS driving functions capable of the synthesis of a source moving along an arbitrary trajectory with an arbitrary velocity profile and by applying an arbitrary SSD-reference curve geometry \cite{firtha2016wave_booklet, doi:10.1121/1.4996126_booklet, Firtha2015:daga_booklet}.
The example of synthesizing a moving point source along a straight trajectory is depicted in Figure \ref{fig:SFS_theory:25D_WFS_moving_source}.
The simulation verifies that amplitude correct synthesis may be achieved along the reference contour

For sources under uniform motion the spectral description of the generated field is available analytically.
This representation allowed me to derive explicit driving function for this dynamic case as well \cite{Firtha2014:daga_booklet, Firtha2014:isma_booklet}.
This driving function serves as a reference solution, since it results in perfect synthesis over the reference line. 
Furthermore, it allows the description of spatial aliasing artifacts due to the discrete SSD in practical applications.
I showed that similarly to the stationary case, WFS constitutes a high frequency approximation of the SDM solution \cite{firtha2015sound_booklet}.

\begin{figure}
	\centering
	\begin{overpic}[width = 1\columnwidth]{figs/antialiased_synth_moving_source.png}
	\footnotesize
	\put(0, 0){(a)}
	\put(50,0){(b)}
	\end{overpic}
\caption{Synthesis of a moving source on an arbitrary trajectory applying a circular SSD with the radius of $R_{\mathrm{SSD}} = 2~\mathrm{m}$, sampled at $\Delta x = 10~\mathrm{cm}$.
	The source is traveling with a constant velocity of $|\mathbf{v}| = \frac{3}{4}c$.
    Figure (a) shows the effects of source discretization with clearly visible aliasing echoes. 
    The strongest contribution emerges from in front of the virtual source where the local perceived frequency is increased by the Doppler effect.
    Figure (b) shows the effect of ideal anti-aliasing filtering.}
	\label{Fig:Moving_sources:anti-aliased_synth_moving_source}
\end{figure}

Finally, I investigated the description and possibilities for the elimination of the spatial aliasing artifacts also in this dynamic case.
In this scenario, aliasing phenomena are even more enhanced than in the stationary case, since the aliasing wavefronts suffer a different Doppler shift than the intended virtual wavefront.
This results in strong coloration and frequency distortion of the virtual field \cite{firtha2016:daga_booklet}.

In case of a harmonic source excitation, aliasing manifests in undesired frequency components, as it has been already reported in the related literature.\footfullcite{Franck2007}
I utilized the spectral, SDM based analysis of the synthesized field in order to give an analytical formulation for these undesired frequency components and showed that with the proper choice of the SSD shape frequency distortion can be minimized \cite{firtha2016:daga_booklet}.
The optimal choice in the aspect of aliasing is the application of a circular SSD.

I extended my anti-aliasing strategy in order to include the case of moving sources.
By applying the proposed approach, anti-aliasing can be achieved by simple temporal low-pass filtering of the loudspeaker signals with a time-variant cut-off frequency \cite{Firtha2018_daga_moving_source_booklet}.
In case of a circular SSD in the center perfect, full-band, anti-aliased synthesis can be performed.
This scenario is presented in Figure \ref{Fig:Moving_sources:anti-aliased_synth_moving_source} in case of moving source emitting a train of band limited impulses.

\clearpage	
\section{Theses}
\label{sec:theses}

\begin{thesisgroup}[Generalization of WFS theory]
I introduced a generalized WFS framework allowing one to synthesize 3D sound fields with arbitrary shaped convex loudspeaker ensembles (secondary source distribution (SSD)) and to optimize the synthesis on an arbitrary convex reference curve. 
The generalized framework inherently contains the existing WFS approaches as special cases \cite{Firtha2016_booklet}.
\begin{thesis}
I established a physical interpretation of the stationary phase approximation (SPA) of boundary integrals.
By defining the local wavenumber vector of a time-harmonic sound field, I showed that the SPA ensures wave front matching of the virtual field and the secondary sound fields at the receiver position.
\end{thesis}
\begin{thesis}
I derived WFS driving function for an arbitrary convex SSD contour based on the above physical interpretation, within the validity of the physical optics approximation of the Kirchhoff-Helmholtz integral.
\end{thesis}
\begin{thesis}
I derived analytical expression for the general \emph{reference curve} that connects the points in the synthesis plane where the amplitude error is minimal.
I presented how the shape of the reference curve can be controlled by applying a frequency independent amplitude correction term to the driving function.
I critically revised existing WFS solutions by the analytical characterization of their reference curves. 
\end{thesis}
\end{thesisgroup}

\begin{thesisgroup}[Spatial explicit driving functions and WFS equivalence]
Besides the implicit WFS technique---yielding the required driving functions as an implicit integral kernel in a reduced surface integral---explicit solutions exist obtaining the driving functions as a spectral integral.
For a linear SSD the explicit solution is termed as the \emph{Spectral Division Method (SDM)} yielding the linear driving functions in terms of an inverse spatial Fourier transform of the ratio of the target field spectrum and the Green's function spectrum measured along a reference line.
So far the connection between the implicit and explicit solutions has only been investigated for special target sound fields.
By applying the SPA to the SDM driving function, I derived it's asymptotical spatial approximation, and I highlighted the general equivalence of the explicit and implicit solutions in the high frequency region. \cite{Firtha2017:daga_booklet, Firtha2018:WFS_vs_SDM_booklet}
\begin{thesis}
I derived analytical SDM driving functions in the spatial domain by applying the SPA to the Fourier integral with establishing a physical interpretation of the stationary phase approximation of Fourier integrals.
Unlike WFS the new expicit driving functions express the SSD driving signals in terms of the target sound field measured along the convex reference curve \cite{Firtha2017:daga_booklet}.\end{thesis}
\begin{thesis}
I proved that under high frequency assumptions the explicit SDM and the implicit WFS driving functions are completely equivalent for an arbitrary target sound field \cite{Firtha2018:WFS_vs_SDM_booklet}.
The proof is performed by expressing the newly introduced driving function in terms of the target field's gradient measured on the SSD.
\end{thesis}
\begin{thesis}
I gave a simple asymptotic anti-aliasing criterion in order to suppress aliasing waves emerging due to the application of a discrete SSD in practical scenarios.
The derivation is based on the above equivalence of the WFS and SDM driving function.
The proposed approach can be implemented in practice by the temporal low-pass filtering of the loudspeaker driving signals \cite{Firtha2018_daga_a_booklet}.
\end{thesis}
\end{thesisgroup}

\begin{thesisgroup}[Wave Field Synthesis of moving point sources]
In the aspect of synthesizing dynamic sound scenes, the synthesis of moving sources is of primary importance.
I adapted the introduced WFS framework to the synthesis of sound fields generated by moving point sources.
\begin{thesis}
I adapted the generalized 3D WFS theory to the synthesis of the field of a point source moving along an a-priori known trajectory, and I defined driving function for an arbitrary convex SSD surface.
The solution takes the Doppler-effect inherently into account \cite{Firtha2015:daga, firtha2016wave_booklet, doi:10.1121/1.4996126_booklet}.\end{thesis}
\begin{thesis}
I derived 2.5D WFS driving functions for a 2D SSD contour in order to synthesize 3D point sources moving along an arbitrary trajectory in the plane of the SSD \cite{doi:10.1121/1.4996126_booklet}.
The derivation relies on the adaptation of the SPA to this dynamic scenario, allowing to optimize the amplitude correct synthesis to a convex reference curve. 
I verified that for the special case of a linear SSD and a parallel reference line the presented driving functions coincide with the traditional WFS driving functions with the stationary source position replaced by the source position at the emission time \cite{doi:10.1121/1.4996126_booklet}.
\end{thesis}
\begin{thesis}
I gave closed form WFS driving function for sources in uniform motion for which particular case the propagation time delay can be expressed explicitly \cite{firtha2016wave_booklet}.
\end{thesis}
\begin{thesis}
I derived frequency domain 2.5D WFS driving function for a linear SSD by applying the SPA directly to the frequency content of a point source under uniform motion \cite{firtha2015sound_booklet}.
\end{thesis}
\end{thesisgroup}

\begin{thesisgroup}[Synthesis of moving sources in the wavenumber domain]

I gave analytical expressions for the spatial Fourier transform of a source moving uniformly along an arbitrary directed straight trajectory. 
Since the SDM is not restricted to stationary sound fields therefore the obtained formulation can be used in order to derive explicit driving function for the synthesis of a moving source.
\begin{thesis} 
I gave the SDM driving functions for the synthesis of a source under uniform motion in the wavenumber domain.
For the special case of a source moving parallel to the secondary source distribution I derived analytical, closed form driving function in the spatial-frequency domain \cite{Firtha2014:daga_booklet, Firtha2014:isma_booklet, firtha2015sound_booklet}.
I showed that similarly to the stationary case, the WFS solution is the high-frequency/farfield approximation of the presented explicit driving function for moving sources \cite{firtha2015sound_booklet}. 
\end{thesis}
\begin{thesis} 
I presented an analytical investigation of the spatial aliasing artifacts emerging from the discretization of the SSD based on the wavenumber description.
I connected the phenomena of frequency distortion with the poles in the secondary sources wavenumber representation and I analytically expressed the aliasing frequency components.
I showed that the artifact can be avoided by applying an SSD that does not exhibit poles on the receiver curve, satisfied optimally by a circular SSD \cite{firtha2016:daga_booklet}.
\end{thesis}
\begin{thesis} 
I extended the spatial anti-aliasing criterion in order to include the synthesis of dynamic sound fields.
By using the introduced formulation spatial aliasing may be eliminated by simple low-pass filtering of the loudspeaker driving signals \cite{Firtha2018_daga_moving_source_booklet}.
\end{thesis}
\end{thesisgroup} 
\clearpage
\section*{Publications}
\addcontentsline{toc}{sec}{Publications}

\nocite{*}
\begin{refcontext}[labelprefix=J]
\printbibliography[title={Journal papers}, keyword=J, heading=subbibliography] 
\end{refcontext}
\begin{refcontext}[labelprefix=C]
\printbibliography[title={Conference papers}, keyword=C, heading=subbibliography] 
\end{refcontext}
\begin{refcontext}[labelprefix=O]
\printbibliography[title={Other publications}, keyword=O, heading=subbibliography] 
\end{refcontext}

\end{document}
