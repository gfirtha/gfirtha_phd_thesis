\documentclass[a4paper,10pt]{article}
%\usepackage{a4wide}
\usepackage{fullpage}
\usepackage{enumerate}
\usepackage{t1enc}
\usepackage[latin2]{inputenc}
\usepackage[T1]{fontenc}
\usepackage{ae,aecompl}
\usepackage[magyar]{babel}
\usepackage{indentfirst}

\frenchspacing 
\pagestyle{empty}
\title{Theses}
\date{}
\begin{document}
\begin{center}
  \textbf{\normalsize Theses for the dissertation \\
   		  \Large Reproduction of moving sound sources applying a unified WFS framework}\\[0.5cm]
\end{center}

\begin{itemize}
\item \emph{Generalization of WFS theory} \\ 
In order to have a unified WFS formulation I created a comprehensive framework for the general WFS theory. The framework inherently contains the existing WFS approaches as special cases, allowing one to synthesize arbitrary sound fields with arbitrary shaped loudspeaker ensembles (secondary source distribution (SSD)), and to optimize the synthesis on an arbitrary receiver curve. 
\begin{itemize}
\item By defining the local wavenumber vector of a harmonic sound field---which points into the direction of the local propagation direction of an arbitrary sound field, being perpendicular to the wave front--- I gave a physical interpretation of the stationary phase approximation (SPA). The approximation ensures a wave front matching of the virtual sound field, and the wave fronts emerging from the SSD elements at an arbitrary receiver position
\item Using the local wavenumber vector I gave simple 3D WFS driving function for an arbitrary shaped SSD curve, emerging from the Physical Optics approximation of the Kirchhoff-Helmholtz integral equation
\item It had been an already known fact, that according to the SPA the synthesized field at an arbitrary position is dominated by its stationary SSD element, i.e. the SSD element, that's emitted wavefront matches with the target sound field at this receiver position.
By utilizing this concept, and the local wavenumber vector I gave the analytical expression, describing the \emph{reference curve}, which are the points, where ampltide error between the synthesized field and the target field is minimal, for an arbitrary shaped SSD contour.
\item I defined the \emph{referencing function}, which is simply the distance between the receiver curve and the corresponding stationary SSD element for a 2D virtual sound field, while it modifies for the case of a 3D virtual sound field due to the \emph{virtual source dimensionality mismatch}
I verified, that by controlling the referencing function one may reference the synthesis on an arbitrary reference curve. As a special case it was shown, that using a linear SSD with a straight reference curve, parallel with the SSD, teremed as the \emph{reference line} the presented framework provides the traditional WFS driving functions.
\item Using the framework I proved, that 2.5D WFS is the high-frequency approximation of the explicit spectral solution, termed the \emph{Spectral Division Method} for an arbitrary 2D virtual sound field.
\item I showed, how the framework may be used for the analysis of the existing referencing approaches, by providing the reference curve for previous WFS techniques.
\end{itemize}
%
\item \emph{Wave Field Synthesis of moving point sources}\\
In the aspect of synthesizing dynamic sound scenes the synthesis of moving sources is of primary importance. In order to give the analytically correct driving function I adapted the unified WFS framework to the description of moving point sources.
\begin{itemize}
\item f
\end{itemize}

%\item Synthesis of uniformly moving sources with the Spectral Division Method
%	\begin{itemize}
%	\item I gave analytical expression for the wavenumber-frequency content of a sound field, generated by a moving acoustic point source with an arbitrary straight trajectory
%	\item I verified, that the derived analytic expression converges weakly to a Dirac-delta distribution for a sound source, moving along the direction of the Fourier-transformation.
%	\item Based on the expression for the wavenumber-content I gave the SDM driving functions for the synthesis of a moving point source in the wavenumber domain
%	\item For the special case of a source moving parallel to the secondary source distribution I gave analytical, closed form driving function in the spatial-frequency domain
%	\end{itemize}
%\item Synthesis of uniformly moving sources with Wave Field Synthesis
%	\begin{itemize}
%	\item Frequency-domain WFS
%		\begin{itemize}
%		\item I gave the frequency domain 3D WFS driving functions for a planar secondary source distribution for a moving point source
%		\item By adapting the stationary phase approximation to the 3D WFS driving functions I derived the 2.5D WFS driving functions for a linear secondary source distribution for a moving point source with an arbitrary straight trajectory
%		\item For the case of a source with a trajectory parallel to the SSD I proved, that the WFS is the high-frequency/farfield approximation the SDM.
%		\end{itemize}
%	\item Time-domain WFS
%		\begin{itemize}
%		\item I adapted the stationary phase approximation for the 3D time domain WFS driving functions. As a result the analytical 2.5D WFS driving functions are obtained for a source, 	moving at arbitrary direction with an arbitrary source excitation, optimizing the synthesis to a line, parallel to the secondary source distribution
%		\item I verified the link between the time-domain and spectral domain WFS driving functions by performing the necessary analytical Fourier-transform.
%		\end{itemize}
%	\end{itemize}
\end{itemize}
\end{document}