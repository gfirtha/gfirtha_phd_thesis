The present thesis discussed a generalized Wave Field Synthesis theory giving an asymptotic, ray-based solution for the general sound field synthesis problems.
As the central result of the present treatise, general 2.5-dimensional WFS loudspeaker driving function was introduced allowing the synthesis of arbitrary virtual sound fields by applying arbitrary shaped convex SSD contours with ensuring an optimal synthesis along a prescribed reference curve.

The central concept of the presented work is the stationary phase approximation (SPA), allowing the asymptotic evaluation of integrals of complex valued functions.
Adapted from classic ray tracing theory, the concept of the local wavenumber vector and the local wavefront curvature was introduced, giving a local description of sound fields.
These quantities gave a simple, elegant physical interpretation for the asymptotic evaluation of integrals describing radiation problems: 
It was demonstrated via numerous examples how boundary integrals and spatial Fourier transforms can be evaluated around their stationary points and how the stationary points for these problems can be found in a simple geometric manner.
All of the involved concepts give a fair approximation for radiation problems under high frequency conditions.

It was demonstrated that the stationary phase approximation of boundary integrals realizes wavefront matching between the target field and the field of the boundary sources regarding both propagation direction and wavefront curvature.
From the stationary phase evaluation of the Kirchhoff integral the generalized 2.5D Wave Field Synthesis driving function was extracted.
It was shown that for a 2D boundary contour the secondary and virtual wavefronts can be matched in amplitude only at a single receiver position per secondary source element.
The ensemble of these individual receiver positions forms the reference curve and its shape can be controlled with a frequency independent amplitude term in the driving function.
The introduced generalized theoretical framework contains the previous WFS approaches as special cases, as it was demonstrated through simple examples.

Alternatively to WFS an explicit solution exists for the general sound field synthesis scenario, solving the problem by mode-matching in the spectral domain.
For an infinite linear secondary source distribution the approach yields the driving function in the form of a Fourier integral, while for enclosing arrays (e.g. circular and spherical distributions) the solution is an infinite spectral sum.
By applying the SPA in order to evaluate the explicit spectral driving function for a linear SSD, a novel spatial domain explicit driving function was introduced: 
The new driving function requires the description of the target sound field measured along the reference curve, opposed to the WFS solution which requires the target field properties on the secondary source array.
It was verified that this new solution also realizes the wavefront matching of the target wavefront and the secondary sources' wavefronts on the prescribed reference curve.
Hence, the global mode-matching solution was transformed asymptotically into a local wavefront matching approach.
This solution may be generalized towards the synthesis with arbitrary shaped secondary source distributions as long as the contour of the loudspeaker array can be considered locally linear.
Finally, it was also demonstrated that the explicit spatial driving function is equivalent with the generalized WFS formulation.

The high frequency equivalence of the explicit and implicit solutions can be exploited in order to discuss phenomena concerning WFS in a unified manner.
As an example, the effects of the secondary source discretization---modeling real-life loudspeaker arrays---can be described analytically in the wavenumber domain in terms of the explicit solution.
By utilizing the introduced local wavenumber concept, a simple anti-aliasing strategy was presented in order to suppress the aliasing wavefronts.
The presented approach can be realized by simple low-pass filtering of the driving signals. 
As a result anti-aliased synthesis may be achieved along certain directions over the listening region.

Finally, as a complex application example for the foregoing, the synthesis of a moving point source was discussed within the context of the introduced WFS framework.
In this case the proper reproduction of the Doppler effect is of central importance which is inherently ensured once an appropriate analytical model is applied.
It was presented how the local attributes of wavefronts can be extended for moving sources and it was demonstrated how the introduced WFS framework can be adapted to this dynamic scenario.
For the special case of sources under uniform motion, planar and linear explicit driving functions were derived in both the wavenumber and in the spatial domain.
Similarly to the stationary case, the explicit driving function was found to coincide with the WFS solution under high frequency assumptions.
Based on the wavenumber domain representation, the effects of secondary source discretization was discussed and the introduced anti-aliasing strategy was extended for the synthesis of moving sources.

\vspace{3mm}
The aim of the present dissertation was to give a complete, self-contained discussion on the questions concerning Wave Field Synthesis.
However, several aspects were out of the scope of the present treatise: as an example, here, only diverging fields were discussed, i.e. the target wavefronts would originate from a sound source outside of the listening region.
Under several restrictions Wave Field Synthesis is capable of synthesizing focused sources in which case the synthesized wavefront converges towards a focal point inside the listening area.
The synthesis of such a converging sound field needs the proper manipulation of the stationary phase approximation.
Although hints were given how to adapt the presented WFS framework to focused sources, the exact study of this focused case---even involving focused moving sources---is the subject of future work.