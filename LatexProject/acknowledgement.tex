This dissertation concludes the work of 9 years which I spent in the Laboratory of Acoustics and Studio Technologies starting with my MSc studies.
To being from that point, I'm thankful for my BSc supervisor Krisztián Gulyás to draw my attention to the technology ,,which is should take a look at, called Wave Field Synthesis'', and to advise me to seek the help of a co-supervisor, who could better help me out with all the underlying math:

My greatest gratitude goes to my MSc and PhD supervisor, Péter Fiala for supporting me with his friendship and strong mathematical basics throughout the years.
His contribution was indispensable for the birth and the quality of the present writing.\footnote{ as well as in the founding of the first hungarian Winnie the Pooh club.}
Also, my thanks goes to all my colleagues for welcoming me in the Laboratory of Acoustics and for the years we spent together: Péter Rucz, Attila Balázs Nagy, Fülöp Augusztinovicz, Ferenc Márki, Dóra Jenei-Kiss, all of whose company and support I greatly appreciated throughout the years.

Over the years I had the luck to connect my work with fellow researchers.
I am very grateful to meet and work together with Fiete Winter, Nara Hahn, Franz Zotter, Jens Ahrens and Sascha Spors, whose work was a starting point for my research.
I am especially thankful for Frank Schultz for the friendly discussions, all the collaboration and for his careful proof-reading with the present dissertation.

Also, from the last years I am grateful for Csaba Huszty for the co-operation in which I could employ my knowledge from the field of sound field reproduction.

Last, but certainly not least, I would like to thank my family---my parents, my brother and my sisters---all my friends and my girlfriend to not only accept but also to support me throughout the years I spent with my research.


%\fscom{check if/how 'application examples' could be more emphasized within TOC or marked/boxed for easier finding within the manuscript}
%
%\fscom{since for plane waves the SPA and SDM not holds in strictly sense, you might show the limiting case for a point source very far away, reducing to no curvarture?? this should end in plane wave drinfing functions, or? I've checked this at once numerically once with the parallel / ref point referencing }
