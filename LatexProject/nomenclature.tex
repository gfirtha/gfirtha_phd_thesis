\nomenclature[01]{$\vx$}{Position vector in Descartes coordinates: $\vx = \posvec{3}{x}{y}{z}$}
%
\nomenclature[02]{$\vv$}{General 3D vector in Descartes coordinates: $\vv = \posvec{3}{v_x}{v_y}{v_z}$}
%
\nomenclature[03]{$\lvert \vv \rvert$}{Length, or $l^2$ norm of vector: $\lvert \vv \rvert = \sqrt{\sum_i v_i^2}$}
%
\nomenclature[04]{$\hat{\vv}$}{Normalized vector, with unit length pointing into the direction of $\vv$: $\hat{\vv} = \frac{\vv}{\lvert \vv \rvert}$.}
%
\nomenclature[05]{$\left< \mathbf{u} \cdot \vv \right>$}{Inner product of vectors $\mathbf{u}$ and $\vv$: $\left< \mathbf{u} \cdot \vv \right>= \sum_i u_i \cdot v_i =\mathbf{u}^{\mathrm{T}} \vv$}
%
\nomenclature[07]{$\vn$}{Normal vector of a surface, or contour}
%
\nomenclature[08]{$\frac{\partial}{\partial x_i}$}{Partial derivative of a function with respect variable $x_i$}
%
\nomenclature[09]{$\frac{\td}{\td x_i}$}{Total derivative of a function with respect argument $x_i$}
%
\nomenclature[10]{$\Dx$}{Gradient operator. In Descartes-coordinates: $\Dx f = \frac{\partial f}{\partial x} \mathbf{e}_x + \frac{\partial f}{\partial y} \cdot \mathbf{e}_y + \frac{\partial f}{\partial z} \cdot \mathbf{e}_z$, with $\mathbf{e}_{x,y,z}$ being unit vectors into $x,y,z$ directions}
%
\nomenclature[11]{$\Dx \cdot$}{Divergence operator. In Descartes-coordinates: $\Dx \cdot f =  \frac{\partial f}{\partial x} + \frac{\partial f}{\partial y} + \frac{\partial f}{\partial z}$}
%
\nomenclature[12]{$\Lx$}{Laplacian operator. 
In Descartes-coordinates: $\Lx f = \frac{\partial^2 f}{\partial x^2} + \frac{\partial^2 f}{\partial y^2} +  \frac{\partial^2 f}{\partial z^2}$}%
%
\nomenclature[13]{$\frac{\partial}{\partial \vv}$}{Directional derivative of a function into the direction of vector $\vv$: $\frac{\partial}{\partial \vv} f = \left< \Dx f \cdot \vv \right>$}
%
\nomenclature[14]{$f'_{x_i}(\vxo)$}{Shorthand for the partial derivative of function $f(\vx)$ with respect to variable $x_i$, taken at the evaluation point $\vxo$: $f'_{x_i}(\vxo)= \frac{\partial}{\partial x_i} f(\vx) \bigg \rvert_{\vx = \vxo}$}
%
\nomenclature[15]{$f''_{x_i,x_j}(\vxo)$}{Shorthand for the second partial derivative of function $f(\vx)$ with respect to variables $x_i$, $x_j$, taken at the evaluation point $\vxo$: $f''_{x_i,x_j}(\vxo)= \frac{\partial^2}{\partial x_i \partial x_j} f(\vx) \bigg \rvert_{\vx = \vxo}$}
%
\nomenclature[16]{$\mathcal{F}_x\left\{\right\}$}{Fourier transform of a function with respect to variable $x$}
%
\nomenclature[17]{$\mathcal{F}^{-1}_{k_x}\left\{\right\}$}{Inverse Fourier transform of a function with respect to variable $k_x$}
%
\nomenclature[18]{$\ast_x$}{Convolution of functions w.r.t. variable $x$: $f \ast_x g = \int_{-\infty}^{\infty} f(x-x_0) g(x_0) \td x_0$.}
\printnomenclature
%
\vspace{1cm}
\paragraph{Temporal Fourier transform:}\mbox{} \\
Given a four dimensional function $f(\vx,t)$, depending on both time and the spatial position.
The forward and inverse temporal Fourier transform is defined as 
\begin{equation}
\label{eq:temporal_fourier_transform_def}
F(\vx,\omega) = \mathcal{F}_t \left\{ f(\vx,t) \right\} = \int\limits_ {-\infty}^{\infty} f(\vx,t) \te^{-\ti \omega t} \td t,
\end{equation}
\begin{equation}
\label{eq:temporal_inverse_fourier_transform_def}
f(\vx,t) = \mathcal{F}_{\omega}^{-1} \left\{ F(\vx,\omega) \right\} = \frac{1}{2\pi} \int_ {-\infty}^{\infty} F(\vx,\omega) \te^{ \ti \omega t} \td \omega.
\end{equation}
Note that capital letter indicates that the function is taken in the angular frequency domain.
%
\paragraph{Spatial Fourier transforms:}\mbox{} \\
Following the convention, given in e.g. \cite{Ahrens2012} the spatial Fourier transform is defined as follows:
\begin{itemize}
\item in one dimension:
\begin{equation}
\label{eq:spatial_fourier_transform_def}
\hat{F}(k_x,y,z,\omega) = \mathcal{F}_x \left\{ F(\vx,\omega) \right\} = \int_ {-\infty}^{\infty} F(\vx,\omega) \, \te^{\ti k_x x} \, \td x,
\end{equation}
\begin{equation}
\label{eq:spatial_inverse_fourier_transform_def}
F(\vx,\omega) = \mathcal{F}_{k_x}^{-1} \left\{ \hat{F}(k_x,y,z,\omega) \right\} = \frac{1}{2\pi} \int_ {-\infty}^{\infty} \hat{F}(k_x,y,z,\omega) \, \te^{ -\ti k_x x} \td k_x,
\end{equation}
\item in two dimensions:
\begin{equation}
\label{eq:spatial_fourier_transform_def_2d}
\hat{F}(k_x,y,k_z,\omega) = \iint_ {-\infty}^{\infty} F(\vx,\omega) \te^{\ti \left( k_x x + k_z z \right } \td x \, \td z,
\end{equation}
\begin{equation}
\label{eq:spatial_inverse_fourier_transform_def_2d}
F(\vx,\omega) = \frac{1}{\left( 2\pi \right)^2} \iint_ {-\infty}^{\infty} \hat{F}(k_x,y,k_z,\omega) \, \te^{ -\ti \left( k_x x + k_z z \right)} \td k_x \, \td k_z,
\end{equation}
\item in three dimensions:
\begin{equation}
\label{eq:spatial_fourier_transform_def_3d}
\hat{F}(\vk,\omega)= \iiint_ {-\infty}^{\infty} F(\vx,\omega) \te^{ \ti \left< \vk \cdot \vx \right>} \td x \,\td y\,\td z,
\end{equation}
\begin{equation}
\label{eq:spatial_inverse_fourier_transform_def_3d}
F(\vx,\omega) = \frac{1}{\left( 2\pi \right)^3} \iiint_ {-\infty}^{\infty} \hat{F}(\vk,\omega) \te^{ -\ti \left< \vk \cdot \vx \right>} \td k_x \, \td k_y \x \td k_z.
\end{equation}
\end{itemize}
Hence, hat over the function symbol indicates that the function is taken in the wavenumber domain.
Note that the exponent of the spatial Fourier transform is taken with a reversed sign, compared to the temporal transform.
Writing an arbitrary function in the form of a spatio-temporal inverse Fourier transform
\begin{equation}
f(\vx,t) = \frac{1}{\left( 2\pi \right)^4} \iiiint_ {-\infty}^{\infty} \hat{F}(\vk,\omega) \te^{ \ti \left( \omega t - \left< \vk \cdot \vx \right> \right)} \td k_x \, \td k_y \x \td k_z
\end{equation}
basically describes the expansion of an arbitrary function into the linear combination of plane waves, propagating to direction $\vk$.
The reversed sign therefore ensures that this simple physical interpretation can be assigned to the Fourier transform.

\paragraph{Fourier transform properties:}\mbox{} \\
Several important properties of the Fourier transform, applied frequently in the present thesis are the following.
\begin{itemize}
\item Shift theorem:
\begin{equation}
\int_{-\infty}^{\infty} f(x-x_0) \te^{\ti k_x x} \td x = \mathcal{F}_x \left\{ f(x-x_0) \right\} = \hat{F}(k_x) \te^{\ti k_x x_0}.
\end{equation}
In case of temporal Fourier transform the right side is with reversed exponent.
\item Convolution theorem:
\begin{equation}
\int_{-\infty}^{\infty} \int_{-\infty}^{\infty} f(x-x_0) g(x_0) \td x_0 \, \te^{\ti k_x x} \td x = \mathcal{F}_x \left\{ f(x) \ast_x g(x) \right\} = \hat{F}(k_x) \cdot \hat{G}(k_x).
\end{equation}
\item Differentiation property:
\begin{equation}
\int_{-\infty}^{\infty} \frac{\partial}{\partial x} f(x) \te^{\ti k_x x} = \mathcal{F}_x \left\{ \frac{\partial}{\partial x} f(x) \right\} = 
-\ti k_x \hat{F}(k_x).
\end{equation}
In case of temporal Fourier transform the right side is with reversed sign.
\end{itemize}


\paragraph{Properties of Dirac delta:}
\paragraph{Basic differentiation properties:}


