In this chapter the theoretical basis of sound radiation is introduced. The section starts with discussing the physics of sound propagation and radiation by deriving the formulation and solution of the governing homogeneous and inhomogeneous wave equations. Various integral representations of sound fields are presented including wave-number domain representation and the Kirchhoff-Helmholtz integral.
%

\vspace{3mm}
Sound is a mechanical disturbance propagating in an elastic fluid, causing an alternation in the pressure (along with density) and in the displacement of the medium's particles. The propagation of the disturbance is described fully by the acoustic wave equation. First the homogeneous wave equation is introduced briefly, which is valid for \emph{source-free} domains. For a detailed treatise on the derivation refer to \cite{Beranek1993, Morse1968, Williams1999, Blackstock2000}.

\section{The homogeneous wave equation}

Consider a homogeneous, elastic fluid, modeled as an ideal gas with no viscosity. In the aspect of the present thesis it is feasible to investigate sound propagation solely in air at room temperature.

%
The domain of investigation i.e. where sound waves propagate is termed as \emph{sound field} hereinafter.
The acoustical quantities of the the sound field is described by \emph{dynamic field variables} in each point $\vx$, at each time instant $t$: the vector variable \emph{particle velocity} $\mathbf{v}(\vx,t)$ and the scalar \emph{instantaneous sound pressure} $p(\vx,t)$ superimposed onto the static pressure $P_0 \approx 10^5~\mathrm{Pa}$.
The medium is quiescent, meaning on average each particle is at rest with zero particle displacement (thus zero particle velocity) at the static pressure $P_0$. 
The presence of a sound wave causes incremental change in the instantaneous pressure and the particle velocity.
%

In order to apply a linear model for sound propagation two assumptions are made.
Since the traveling speed of thermal diffusion is small compared to the speed of sound and the acoustical wavelength in the audible frequency range, it is feasible to assume, that heat exchange in the wave due to compression and expansion is negligible: the state changes are modeled as adiabatic changes.
Furthermore the alternation of the instantaneous sound pressure must be small compared to the static pressure, so the non-linear adiabatic state-change characteristics can be linearized around $P_0$. This later assumption is also fulfilled for pressure magnitudes below the threshold of pain of the human auditory system, as it was pointed out in \cite{Gumerov2004, Ahrens2012}.
%

The linear wave equation may be derived by utilizing two fundamental physical principles.
\begin{itemize}
\item \emph{The equation of motion:} By applying Newton's second law for an infinitesimal small volume of gas we obtain the connection between the particle velocity vector and the pressure field at each point at each time instant. The resulting \emph{Euler's equation} states that the force acting on the volume, which is proportional to the pressure on the surface causes an acceleration, which is given by the time derivative of the particle velocity:
\begin{equation}
\nabla p(\vx,t) = -\rho_0 \frac{\partial}{\partial t} \mathbf{v}(\vx,t),
\label{Eq:Theory:Eulers_equation}
\end{equation}
\nomenclature[2]{$\nabla$}{Gradient operator. In Descartes-coordinates it is given by $\nabla = \frac{\partial}{\partial x} \mathbf{e}_x + \frac{\partial}{\partial y} \mathbf{e}_y + \frac{\partial}{\partial z} \mathbf{e}_z$}
where $\nabla$ is the gradient operator and $\rho_0$ is the fluid ambient density. In room temperature for the above given static pressure it is given as $\rho_0 = 1.18~\mathrm{kg}/\mathrm{m}^3$.
Equation \eqref{Eq:Theory:Eulers_equation} may be transformed with respect to time into the angular frequency domain, supposing steady-state harmonic wave solutions. Using the Fourier differentiation theorem the frequency domain Euler's equation reads
\begin{equation}
\nabla P(\vx,\omega) = -\ti \omega \rho_0 \mathbf{V}(\vx,\omega).
\label{Eq:Theory:Freq_Eulers_equation}
\end{equation}

\item \emph{The gas law:} For adiabatic processes the change of state is governed by the following equation
\begin{equation}
P V^{\gamma} = \mathrm{constant},
\label{Eq:Theory:Adiabatic_change}
\end{equation}
where $\gamma = c_P/c_V$ is the ratio of specific heat of the fluid in constant pressure and with constant volume. For air $\gamma = 1.4$. Linerization of \eqref{Eq:Theory:Adiabatic_change} and expressing the change in pressure --equaling the instantaneous pressure-- one obtains
\begin{equation}
\Delta P = p(\vx,t) = -\gamma P_0 \frac{\Delta V}{V_0},
\end{equation}
where $V_0$ is the undisturbed volume. The relative change of volume may be expressed as the sum of particle displacement over the boundary surface. Applying the definition of divergence and expressing the equation in terms of particle velocity yields
\begin{equation}
\frac{\partial}{\partial t} p(\vx,t) = -\gamma P_0 \nabla \cdot \mathbf{v}(\vx,t),
\label{Eq:Theory:Second_eq}
\end{equation}
where $\nabla \cdot$ is the \emph{divergence operator}.
\end{itemize}

%
Taking the time derivative of equation \ref{Eq:Theory:Second_eq} and the divergence of equation \ref{Eq:Theory:Eulers_equation} the particle velocity may be eliminated. By using the \emph{Laplacian-operator} $\nabla \cdot \nabla = \nabla^2$ the scalar linear homogeneous wave equation is obtained for the sound pressure
\begin{equation}
\nabla^2 p(\vx,t) - \frac{1}{c^2} \frac{\partial^2}{\partial t^2} p(\vx,t) = 0,
\label{Eq:Theory:Scalar_wave_equation}
\end{equation}
\nomenclature[1]{$c$}{Speed of sound}%
\nomenclature[3]{$\nabla^2$}{Laplacian operator. In Descartes-coordinates: $\nabla^2 = \frac{\partial^2}{\partial x^2} + \frac{\partial^2}{\partial y^2} +  \frac{\partial^2}{\partial z^2}$}%
where $c \equiv \sqrt{ \frac{\gamma P_0}{\rho_0} }$ is the speed of the sound wave in the medium. For air in room temperature it is given as $c = 343.1 ~ \mathrm{m}/\mathrm{s}$.

The instantaneous pressure may be also eliminated in a similar way. In this case the vector wave equation is obtained for the particle velocity, valid for curl-free media
\begin{equation}
\nabla^2 \mathbf{v}(\vx,t) - \frac{1}{c^2} \frac{\partial^2}{\partial t^2} \mathbf{v}(\vx,t) = 0,
\label{Eq:Theory:Vector_wave_equation}
\end{equation}
where $\nabla^2 = \nabla \left( \nabla \cdot \right)$.
%
The wave equations fully describe the properties of acoustic wave propagation as long as the above made assumptions are fulfilled.
Besides the pressure and the velocity, acoustic fields are often expressed via the scalar \emph{velocity potential} $\varphi(\vx,t)$, for which the acoustic wave equation also holds, and is related to the other field variables as 
\begin{equation}
\mathbf{v}(\vx,t) = \nabla \varphi(\vx,t), \hspace{7mm} p(\vx,t) = -\rho_0 \frac{\partial}{\partial t} \varphi(\vx,t).
\label{eq:theory:velocity_potential_definition}
\end{equation}
%

\vspace{3mm}
%
Equation \eqref{Eq:Theory:Scalar_wave_equation} can be transformed into the frequency domain by performing a temporal Fourier-transform. By using the differentiation property of Fourier-transform the \emph{homogeneous Helmholtz-equation is obtained}:
\begin{equation}
\nabla^2 P(\vx,\omega) + k^2 P(\vx,\omega) = 0,
\label{Eq:Theory:Homog_Helmholtz}
\end{equation}
where $k$ is the \emph{acoustic wavenumber}, which is related to the temporal frequency trough the \emph{dispersion relation}:
\begin{equation}
k = \frac{\omega}{c}.
\end{equation}
%
Equation \eqref{Eq:Theory:Homog_Helmholtz} must hold for every physically possible \emph{steady-state} wave form with harmonic time-dependence for a source-free volume (which latter is indicated with the zero load term on the right side). In the aspect of the present thesis the time-domain wave equation is rarely solved, therefore in the followings the general solution for the Helmholtz-equation is presented.

\subsection{Solution of the homogeneous wave equation}

First the solution of the homogeneous wave equation is considered in Cartesian coordinate system, leading to the plane wave theory.

The Descartes coordinate form of the Laplace-operator is given by
\begin{equation}
\nabla^2 = \frac{\partial^2}{\partial x^2} + \frac{\partial^2}{\partial y^2} +  \frac{\partial^2}{\partial z^2}.
\end{equation}
The general solution for the Helmholtz-equation is obtained by the separation of variables \cite{Devaney2012}: let's try to find the solution in the form
\begin{equation}
P(\vx,\omega) = \hat{P}(\omega) X(x)Y(y)Z(z).
\label{Eq:Theory:Seperated_variables}
\end{equation}
Substituting it into \eqref{Eq:Theory:Homog_Helmholtz} and dividing both sides by $\hat{P}(\omega) X(x)Y(y)Z(z)$ yields
\begin{equation}
\underbrace{\frac{\td^2 X(x)}{\td x^2}\frac{1}{X(x)}}_{-k_x^2} + 
\underbrace{\frac{\td^2 Y(y)}{\td y^2}\frac{1}{Y(y)}}_{-k_y^2} + 
\underbrace{\frac{\td^2 Z(z)}{\td z^2}\frac{1}{Z(z)}}_{-k_z^2}
= - k^2.
\label{Eq:Theory:Seperated_variables_expanded}
\end{equation}
Since each term contains a total derivative---independent from any other variable-- equality may hold only if each term is constant. These constant are denoted by $k_x-k_y-k_z$. Consequently each part of the equation leads to a simple eigenvalue problem, for which the eigenfunction solution is well-known. Given e.g. for $x$-variable:
\begin{equation}
\frac{\partial^2 X(x)}{\partial x^2} = -k_x^2 X(x) \hspace{5mm} \rightarrow \hspace{5mm} X(x) = A_1 \te^{-\ti k_x x} + A_2 \te^{\ti k_x x}.
\end{equation}
The solutions may substituted back to equation \eqref{Eq:Theory:Seperated_variables}. In order to include every possible solution the general solution for the free-field homogeneous Helmholtz-equation is yielded by summation over all possible values of $k_x-k_y-k_z$ weighted by arbitrary constants. However, the variables are not independent, since for a fixed temporal frequency they are related according the dispersion relation
(resulting from \eqref{Eq:Theory:Seperated_variables_expanded}):
\begin{equation}
k^2 = \left( \frac{\omega}{c} \right)^2 = k_x^2 + k_y^2 + k_z^2.
\end{equation}
As a dependent variable we will use $k_y$ trough this treatise so that
\begin{equation}
k_y = \sqrt{ k^2 - k_x^2 - k_z^2 }.
\end{equation}
Using this and by denoting the arbitrary constant by $\tilde{P}(k_x,k_z, \omega)$ the general solution reads
\begin{equation}
P(\vx,\omega) = \frac{1}{(2\pi)^2}\iint_{-\infty}^{\infty} \tilde{P}(k_x,k_z, \omega)  \te^{- \ti \left( k_x x + k_y y + k_z z \right) }
\td k_x\td k_z.
\label{Eq:Theory:Helmholtz_Inverse_Fourier}
\end{equation}
Constant $\frac{1}{(2\pi)^2}$ is introduced as a Fourier-transform normalization term\footnote{The same result can be obtained by performing a forward Fourier transform to the Helmholtz equation along all spatial dimensions resulting in $\left(-k_x^2 -k_y^2 -k_z^2 + \left(\frac{\omega}{c}\right)^2 \right)\tilde{P}(k_x,k_y,k_z,\omega) = 0$, which is satisfied for any $\tilde{P}$ as long as the dispersion relation holds. In the inverse transform for $P(\vx,\omega)$ the latter can be taken into consideration by multiplication with $\delta(k_y - \sqrt{k^2-k_x^2-k_z^2})$, and by exploiting the sifting property of the Dirac function integration with respect to $k_y$ can be carried out.}.

\vspace{3mm}
One separated solution from the integral is in the form of \cite{Williams1999}
\begin{equation}
P(\vx,\omega) = \hat{P}(\omega) \te^{-\ti \left( k_x x + k_y y + k_z z \right) } =  \hat{P}(\omega) \te^{-\ti \mathbf{k}^{\mathrm{T}} \vx },
\end{equation}
where $\mathbf{k} = [k_x,\ k_y,\ k_z]^{\mathrm{T}}$ is the \emph{wavenumber vector}, with its length equaling the acoustic wavenumber $k = | \mathbf{k}|$ and pointing into the direction of the maximum phase advance, given by the gradient of the phase function.
The solution represents a \emph{plane wave} component traveling in the direction $\mathbf{k}$ with the acoustic wavelength of $\lambda = 2\pi/k$. The terminology indicates that the surface of constant phase points are lying along an infinite plane, perpendicular to $\mathbf{k}$. Refer to figure \ref{Fig:Theory:plane_wave} (a) for the illustration of a traveling plane wave.

\begin{figure}[!h]
	\centering
	\begin{overpic}[width = 1\columnwidth ]{Figures/Wave_theory/plane_wave_illustration.png}
	\put(2,1){(a)}
	\put(52,1){(b)}
	\end{overpic}
\caption{Illustration of a traveling plane wave (a) and an evanescent wave (b) with $\omega = 2\pi \cdot 1000 ~\mathrm{rad/s}$. In the present case the plane wave travels along the $xy$-plane, with $k_z = 0$. Variables $k_x = \cos \varphi, \hspace{2mm} k_y = \sin \varphi$ give the wavenumber components along the $x$ and $y$ directions. For the case of the evanescent wave $k_y<\frac{\omega}{c}$, resulting in exponential decay along the $y$-coordinate.}
	\label{Fig:Theory:plane_wave}
\end{figure}
\vspace{2mm}
%As it is indicated in the figure $k_x-k_y-k_z$ variables are the $x-y-z$ directional components of the wavenumber vector. For the sake of simplicity assume that $k_z = 0$, thus the propagation direction of the plane wave is parallel with the $z=0$ plane. In this case the wavenumber components are expressed as
%\begin{eqnarray}
%k_x = k \sin \theta , \\
%k_y = k \cos \theta .
%\end{eqnarray}
%  
%\subsubsection{Evanescent waves}
Since there is no constraint on the values of $k_x$ and $k_z$, the plane wave equation is satisfied also when $k_x>k$ or $k_z>k$. Resulting from the dispersion relation in these cases $k_y$ becomes complex.
In order to ignore the non-physical exponentially increasing solution in the followings we define $k_y$ as 
\begin{equation}
k_y = \begin{cases}
                        \sqrt{\left(\frac{\omega}{c}\right)^2 - k_x^2 - k_z^2}  & \text{if} \hspace{3mm} k_x^2 + k_z^2 \leq \left(\frac{\omega}{c}\right)^2\\
                        -\ti \sqrt{\left(k_x^2+k_z^2 - \frac{\omega}{c}\right)^2 } = -\ti k_y' &  \text{if} \hspace{3mm} k_x^2 + k_z^2 > \left(\frac{\omega}{c}\right)^2,
                    \end{cases}
\label{eq:theory:k_y_definition}
\end{equation}
Solutions with $k_y'$ describe plane waves, propagating perpendicular to the $y$-axis, exhibiting an exponentially decaying amplitude along $y$-direction (see Figure \ref{Fig:Theory:plane_wave} (b)):
\begin{equation}
P(\vx,\omega) = \hat{P}(\omega) \te^{-k_y' y} \te^{-\ti \left( k_x x + k_z z \right) }.
\end{equation}
In those cases when one wavelength component is shorter, then the acoustic wavelength the wave can not propagate from the $y = 0$ surface, but an exponentially decaying radiation phenomena occurs. These type of waves are termed \emph{evanescent waves}, opposed to \emph{propagating waves}, when all wavenumber components are real valued.

Evanescent waves are often the results of the difference between the speed of sound in different materials: in solids the speed of sound is significantly higher, than in air. As a consequence in case of e.g. a vibrating solid surface higher-order modes will not be radiated into the free-space, since the wavelength on the surface of the object is shorter, than the acoustic wavelength would be in air. In these cases air above the surface acts as a hydrodynamic short-circuit.

The evanescent contribution is of central importance in the field of \emph{Nearfield Acoustic Holography}---when one needs a high-resolution image from the velocity distribution on the vibrating object's surface---, however their contribution is often neglected in the field of Sound Field Synthesis, when the listener is relatively far from the secondary loudspeaker array, and loudspeaker spacing is higher than the evanescent wavelengths.


\subsubsection{The Angular Spectrum and Wave Field Extrapolation}
We could see, that any source-free sound field may be expressed in terms of a double inverse Fourier-transform, given by \eqref{Eq:Theory:Helmholtz_Inverse_Fourier}.
This formulation is termed as the \emph{angular spectrum representation} \cite{Ahrens2010phd, Ahrens2012, Williams1999} or the \emph{plane wave expansion} \cite{Spors2005} of the sound field.
Expressing the pressure at the infinite plane $y=0$ reveals, that $P(x,0,z,\omega) = \mathcal{F}_x^{-1}\mathcal{F}_z^{-1} \left\{\tilde{P}(k_x,k_y, \omega)\right\}$,
and the \emph{angular spectrum}, or \emph{plane wave expansion coefficients} $\tilde{P}(k_x,k_z, \omega)$ can be therefore expressed as the corresponding forward Fourier-transform of the pressure distribution at $y=0$.%: $\hat{P}(k_x,k_y, \omega) = \mathcal{F}_x\mathcal{F}_z \left\{  P(x,0,z,\omega) \right\}$.
In the followings, the domain characterized by $k_x$,$k_z$ is termed as the \emph{wavenumber domain}.

Equation \eqref{Eq:Theory:Helmholtz_Inverse_Fourier} therefore constitutes a connection between the pressure distribution of an arbitrary sound field measured an arbitrary point and at the plane $y=0$. In the wavenumber domain the equation reads:
\begin{equation}
\mathcal{F}_x\mathcal{F}_z \left\{ P(\vx,\omega) \right\} = \tilde{P}(k_x,y,k_z,\omega) = \tilde{P}(k_x,0,k_z,\omega) \te^{-\ti k_y y}.
\label{Eq:Theory:Wave_field_extrapolation}
\end{equation}
Note, that wave propagation is determined by the phase change of the plane wave expansion's $y$-component, therefore generally speaking the following equation holds:
\begin{equation}
\tilde{P}(k_x,y,k_z,\omega) = \tilde{P}(k_x,y',k_z,\omega) \te^{-\ti k_y ( y - y' ) }.
\end{equation}

\vspace{3mm}
This statement leads to two important formulations:
the above equation written in the spatial domain yields
\begin{equation}
P(\vx,\omega) = \frac{1}{4\pi^2}\iint_{-\infty}^{\infty} \tilde{P}(k_x,y',k_z,\omega) \te^{-\ti k_y ( y - y' ) }  \te^{- \ti \left( k_x x + k_y y + k_z z \right) }
\td k_x\td k_z,
\label{Eq:Theory:Pressure_propagated}
\end{equation}
Expressing $\tilde{P}(k_x,y',k_z,\omega)$ in terms of the normal velocity $\tilde{V}_{\mathrm{n}}(k_x,y',k_z,\omega)$ using the Euler's equation  \eqref{Eq:Theory:Freq_Eulers_equation}, with the normal ---i.e. $y$---derivative at $y = y'$ calculated by applying the differentiation theorem to \eqref{Eq:Theory:Wave_field_extrapolation} one obtains
\begin{equation}
P(\vx,\omega) = \frac{1}{4\pi^2}\iint_{-\infty}^{\infty} \rho_0 c k \tilde{V}_{\mathrm{n}}(k_x,y',k_z,\omega) \frac{\te^{-\ti k_y ( y - y' ) } }{k_y} \te^{- \ti \left( k_x x + k_y y + k_z z \right) }
\td k_x\td k_z.
\label{Eq:Theory:Velocity_propagated}
\end{equation}
These formulations are of central importance in the field of Fourier-acoustics. They state that an arbitrary sound field is completely determined by either the pressure, or by the normal velocity component, measured along an infinite plane. Wave propagation is calculated by multiplying the measured spectra with an exponential term, referred as the propagators: from \eqref{Eq:Theory:Pressure_propagated} term $\te^{-\ti k_y ( y - y' ) }$ is referred as the \emph{pressure propagator} and from \eqref{Eq:Theory:Velocity_propagated} term $\frac{\te^{-\ti k_y ( y - y' ) } }{k_y}$ is called the \emph{velocity propagator}.
The importance of these statements---which are the wavenumber expression of the Rayleigh I. and II. integrals---will be further investigated in the latter sections, dealing with Sound Field Synthesis using a planar secondary source distribution.

\vspace{3mm}
Similarly to the presented Cartesian-solution, the general solution for the free-field homogeneous Helmholtz equation can be found for spherical and cylindrical coordinate systems. The representations are given in the form of an infinite series of spherical and cylindrical harmonics respectively, connecting the radiated sound at an arbitrary point, and the sound field measured on a spherical or an infinite cylindrical surface. These solutions are of great importance when spherical, or circular secondary source distributions are applied for sound field reconstruction. Since the present thesis does not include the spectral solution of the reconstruction problem for these geometries, the presentation of the spherical and cylindrical solutions are omitted. For a detailed investigation refer to \cite{Williams1999, Zotter2009phd, Ahrens2012}.

\subsubsection{Boundary conditions}
\label{Section:Theory:Boundary_conditions}

So far we considered wave propagation in free-field, i.e. no boundaries were present.% In order to obtain a particular solution for the wave equation boundary conditions must be known.
%As initial conditions through the present thesis we suppose \emph{homogeneous Cauchy initial conditions} by setting $p(\vx,0) = 0$, $\frac{\partial}{\partial t}p(\vx,t)|_{t=0} = 0$, ensuring, that no sound waves are present in the domain of investigation, that would case the non-uniqueness of the particular solution.
In the presence of boundaries the wave field must satisfy prescribed boundary conditions.
If the domain of interest is the exterior of the enclosing boundary, while the sources of radiation are inside the volume---or it is the vibrating boundary surface itself--- the problem to be solved is termed as an \emph{exterior radiation problem}. On the other hand, if the aim is to determine the sound field inside a source-free volume an \emph{interior problem} must be solved.

The boundary conditions are typically continuous pressure or particle velocity. By assuming zero pressure or velocity on the boundary surface \emph{homogeneous boundary conditions} are considered. Non-zero field variables on the other hand represent a vibrating surface and are termed \emph{inhomogeneous bondary conditions}.

In the aspect of this thesis two important types of boundary conditions are of interest:
\begin{itemize}
\item \emph{Dirichlet boundary condition} prescribes the pressure, measured on the boundary surface. The homogeneous Dirichlet boundary conditions are thus
\begin{equation}
P(\vx,\omega) = 0, \hspace{3mm} \forall \hspace{3mm} \vx \in S.
\end{equation}
These types of boundaries are called \emph{sound-soft}, or \emph{pressure release} boundaries. These types of boundary conditions are used to model e.g. the surface of the ocean for a wave, propagating in the water \cite{Blackstock2000, Ziomek1995}.

The inhomogeneous Dirichlet boundary condition assumes a prescribed pressure value on the boundary surface:
\begin{equation}
P(\vx,\omega) = f_D(\vx,\omega), \hspace{3mm} \forall \hspace{3mm} \vx \in S.
\end{equation}
\begin{figure}
	\centering
	\begin{overpic}[width = .5\columnwidth]{Figures/Wave_theory/boundary_conditions.png}
	\small
	\put(27,37){$\mathbf{n}_{\mathrm{in}}$}
	\put(40,47){$V_{\mathrm{i}}$}
	\put(50,82){$V_{\mathrm{e}}$}
	\put(11,48){$S$}	
	\put(79,71){$r$}	
	\put(84,85){$S_{\infty}$}
	\end{overpic}
	\caption{Geometry for the boundary conditions in general interior and exterior radiation problems, and the infinite boundary for the Sommerfeld radiation condition}
	\label{Fig:Theory:bounday_condition}
\end{figure}

\item \emph{Neumann boundary condition} gives the normal derivative of the pressure on the boundary surface, i.e. prescribes the normal velocity of the surface. %For the sake of simplicity the normal derivative taken on the surface uses the following notation and definition
%\begin{equation}
%\frac{\partial}{\partial n} f(\vx)\equiv \left. \frac{\partial}{\partial \mathbf{n}(\vx)} f(\vx) \right|_{\partial \Omega} \equiv \left. \langle \nabla f(\vx), %\mathbf{n}(\vx) \rangle \right|_{\partial \Omega},
%\end{equation}
%where $ \mathbf{n}(\vx) $ is the normal vector of the boundary surface. For interior problems the inward pointing normal is used.%
%
Homogeneous Neumann boundary condition are
\begin{equation}
\left. \frac{\partial}{\partial \mathbf{n}(\vx)} f(\vx) \right|_{S}= 0,
\end{equation}
where $ \mathbf{n}(\vx) $ is the normal vector of the boundary surface.
These type of boundaries are termed as \emph{sound hard}, or \emph{rigid} boundaries, ensuring that no incident wave can move the boundary surface.

Inhomogeneous Neumann boundary conditions are given by
\begin{equation}
\left. \frac{\partial}{\partial \mathbf{n}(\vx)} f(\vx) \right|_{S}= f_N(\vx,\omega), \hspace{3mm} \forall \hspace{3mm} \vx \in S.
\end{equation}
Vibrating surfaces ---e.g. mounted loudspeakers, or baffled pistons--- are most often modeled with these type of boundary conditions.
\end{itemize}

For radiation problems it is feasible to assume free field conditions, i.e. only outgoing waves are present in the sound field. This is ensured mathematically by the \emph{Sommerfeld radiation condition}, excluding the non-physical solutions of the wave equation emerging from infinity, for which the desired sources of radiation behave as acoustic sinks.
Mathematically it can be formulated by implying boundary condition on $S_{\infty}$, with $r$ is increased to infinity, as shown in Figure \ref{Fig:Theory:bounday_condition}:
\begin{equation}
\lim_{r \rightarrow \infty} r \left( \left. \frac{\partial}{\partial r}P(\vx,\omega)\right|_{S_{\infty}} +\ti \frac{\omega}{c}P(\vx,\omega) \right) = 0, \hspace{3mm} \forall \hspace{3mm} \vx \in S_{\infty}.
\label{Eq:Theory:Sommerfeld_radiation_condition}
\end{equation}
The condition stems from the inclusion of the surface $S_{\infty}$ to the Kirchhoff-Helmholtz integral for the general exterior problem (see in the following section), and by deriving boundary conditions, that ensure zero contribution of this surface in its limiting value\cite{Schot1992:Eighty_years, Williams1999}.

\newpage
\section{The inhomogeneous wave equation and the Green's function}

So far we investigated wave propagation in source-free volumes. Sources of sound may be included into the wave equation resulting in the time domain inhomogeneous wave equation
\begin{equation}
\nabla^2 p(\vx,t) -\frac{1}{c^2}\frac{\partial^2}{\partial t^2}p(\vx,t) = -q(\vx,t),
\label{Eq:Theory:Inhomogene_wave_eq_time_domain}
\end{equation}
and by transforming wrt. time in the inhomogeneous Helmholtz equation:
\begin{equation}
(\nabla^2 + k^2 ) P(\vx,\omega ) = -Q(\vx,\omega).
\end{equation}
Term $q(\vx,t)$ is referred as the \emph{load term}, describing the spatial extension and time history of the excitation.
%It should be noted, that the solution of the inhomogeneous wave equation is not unique, since any solution for the homogeneous wave equation may be added to the solution, the inhomogeneous wave equation is still satisfied. In order to obtain a unique solution again, we impose \emph{Cauchy initial conditions}.

\vspace{3mm}
A common way to obtain the solution for the inhomogeneous wave equation is using the \emph{Green's function}. We define the n-dimensional \emph{Green's function} as the solution for the following equation \cite{Gumerov2004, Williams1999}
\begin{equation}
\nabla^2 g(\vx|\vxo,t) -\frac{1}{c^2}\frac{\partial^2}{\partial t^2} g(\vx|\vxo,t) = -\delta\left( \vx - \vxo \right)\delta\left( t - t_0 \right),
\label{Eq:Theory:Green_function_def}
\end{equation}
with $\vx, \vxo \in \mathbb{R}^{n}$ and where $\delta()$ is the Dirac-delta distribution. The Green's function therefore describes the propagation of an impulsive disturbance in the pressure field at the time instant $t_0$, located at $\vxo$, measured at $\vx$. The Green's function is often referred as the \emph{spatio-temporal impulse response} of the domain of interest and its temporal Fourier-transform $G(\vx|\vxo,\omega)$ as the \emph{spatio-temporal transfer function} of a point source at $\vxo$. In the followings we assume free-field conditions by implying the Sommerfeld-radiation condition. Under these assumptions the \emph{free field Green's function} is translation invariant, denoted by $g(\vx-\vxo,t)$.

The motivation behind the use of the Green's function is that assuming an arbitrary linear differential operator $\mathcal{L}\left\{ \right\}$ acting on a distribution $p(\vx)$ with an arbitrary excitation $-f(\vx)$, then the solution of the inhomogeneous differential equation $\mathcal{L}\left(p(\vx)\right) = -f(\vx)$ may be expressed by the convolution of the Green's function and the load term:
\begin{equation}
\mathcal{L}\left\{ g(\vx-\vxo) \right\} = -\delta( \vx-\vxo ) \hspace{3mm} \rightarrow \hspace{3mm}
p(\vx) = \int_{\vxo }  g(\vx-\vxo) f(\vxo) \td \vxo.
\label{Eq:Theory:Basic_Green_function_eq}
\end{equation}

The Green's function is usually obtained by eigenfunction expansion of the operator in a given geometry with specified boundary conditions. Under free-space assumptions, where harmonic functions give a full orthogonal basis a straightforward method is to perform a Fourier transform to equation \eqref{Eq:Theory:Green_function_def} with respect to space and time, yielding in $\vx \in \mathbb{R}^{3}$
\begin{equation}
(-(k_x^2 + k_y^2 + k_z^2) + \left(\frac{\omega}{c} \right)^2)\tilde{G}(\mathbf{k},\omega) = -1,
\end{equation}
with $\mathbf{k} = [k_x,\ k_y,\ k_z]^{\mathrm{T}}$.
The Green's function in the wavenumber space reads \cite{Devaney2012, Watanabe2015}
\begin{equation}
\tilde{G}(\mathbf{k},\omega) = -\frac{1}{\left( \frac{\omega}{c}\right)^2 - \mathbf{k}^{\mathrm{T}} \mathbf{k}}.
\label{Eq:Theory:3D_kxkykzw_Green}
\end{equation}
Applying the convolution theorem the solution of \eqref{Eq:Theory:Inhomogene_wave_eq_time_domain} in the wavenumber domain reads
\begin{equation}
\tilde{P}(\mathbf{k},\omega)  = \tilde{Q}(\mathbf{k},\omega) \tilde{G}(\mathbf{k},\omega) = -\frac{Q	(\mathbf{k},\omega)}{\left( \frac{\omega}{c}\right)^2 -  \mathbf{k}^{\mathrm{T}} \mathbf{k} },
\end{equation}
and the solution in the spatio-temporal domain is yielded by the inverse Fourier-transform:
\begin{equation}
p(\vx,t) =\frac{1}{(2\pi)^4} \iiiint^{\infty}_{-\infty} - \frac{\tilde{Q}(\mathbf{k},\omega)}{\left( \frac{\omega}{c}\right)^2 -  \mathbf{k}^{\mathrm{T}} \mathbf{k} } \te^{-\ti \left( \mathbf{k}^{\mathrm{T}}\vx - \omega t \right) } \td k_x \td k_y \td k_z \td \omega.
\end{equation}

%\begin{figure}
%	\centering
%	\begin{overpic}[width = 1\columnwidth]{Figures/Theory/point_source.png}
%	\end{overpic}
%	\caption{3D point source}
%	\label{Fig:Theory:point_source}
%\end{figure}

\vspace{3mm}
The different representations of the free-field Green's function may be obtained by the corresponding inverse Fourier-transform of \eqref{Eq:Theory:3D_kxkykzw_Green} (setting $\mathbf{x_0} = 0$, i.e. the point source is placed at the origin) \cite{Devaney2012, Duffy2001:Greens, Ahrens2010a, Ahrens2012, Gibson2008}
\begin{center}
    \begin{tabular}{  c | | l |	 l }%\toprule
      & 3-dimensional & 2-dimensional \\ \hline
    $\tilde{G}(k_x,k_y,k_z,\omega)$ & $-\frac{1}{ \left(\frac{\omega}{c}\right)^2 - k_x^2-k_y^2-k_z^2} $ &  $-\frac{1}{\left(\frac{\omega}{c}\right)^2 - k_x^2-k_y^2}\delta(k_z)$ \\ 
    $\tilde{G}(k_x,y,k_z,\omega)$   &  
    \scriptsize	$\begin{aligned}[t]
	-\frac{\ti}{2}\frac{\te^{-\ti\sqrt{(\frac{\omega}{c})^2 - k_x^2 - k_z^2}|y|}}{\sqrt{(\frac{\omega}{c})^2 - k_x^2 - k_z^2}},\hspace{3mm} \text{for} \hspace{1mm}
	\sqrt{k_x^2+k_z^2}	\leq\left| \frac{\omega}{c} \right| \\
	\frac{1}{2}\frac{\te^{-\sqrt{k_x^2 + k_z^2-(\frac{\omega}{c})^2}|y|}}{\sqrt{k_x^2 + k_z^2-(\frac{\omega}{c})^2}},\hspace{3mm}  \text{for} \hspace{1mm}				\sqrt{k_x^2+k_z^2}>\left| 	\frac{\omega}{c} \right| 
	\end{aligned}$ \normalsize    
    & $-\frac{1}{\left(\frac{\omega}{c}\right)^2 - k_x^2-k_y^2}$  \\
    $\tilde{G}(k_x,y,z,\omega)$
    \footnote{$\tilde{G}(k_x,k_y,z,\omega)$ and $\tilde{G}(k_x,y,z,\omega)$ may be written in a less expressive, but briefer form as 
    $\tilde{G}(k_x,y,k_z,\omega) = -\frac{\ti}{2}\frac{\te^{-\sqrt{(\frac{\omega}{c})^2 - k_x^2 - k_z^2}|y|}}{\sqrt{(\frac{\omega}{c})^2 - k_x^2 - k_z^2}}$ and
    $\tilde{G}(k_x,y,z,\omega) = -\frac{\ti}{4}H_0^{(2)}\left( -\ti\sqrt{k_x^2-(\frac{\omega}{c})^2  } \sqrt{y^2+z^2} \right)$ valid for arbitrary $k_x$.
    }     
    &      
    \scriptsize
    $\begin{aligned}[t] % placement: default is "center", options are "top" and "bottom"
	-\frac{\ti}{4} H_0^{(2)}\left( \sqrt{(\frac{\omega}{c})^2 - k_x^2 } \sqrt{y^2+z^2} \right),\hspace{3mm} \text{for} \hspace{1mm}|k_x|<\left| \frac{\omega}{c} \right| \\ \frac{1}{2\pi} K_0\left( \sqrt{k_x^2 - (\frac{\omega}{c})^2 } \sqrt{y^2+z^2} \right),\hspace{3mm}  \text{for} \hspace{1mm}|k_x|>\left| \frac{\omega}{c} \right| 
	\end{aligned}$ \normalsize
     &     
     \scriptsize	$\begin{aligned}[t]
	-\frac{\ti}{2}\frac{\te^{-\ti\sqrt{(\frac{\omega}{c})^2 - k_x^2 }|y|}}{\sqrt{(\frac{\omega}{c})^2 - k_x^2 }},\hspace{3mm} \text{for} \hspace{1mm}|k_x|		\leq			\left| \frac{\omega}{c} \right| \\
	\frac{1}{2}\frac{\te^{- \sqrt{k_x^2 -(\frac{\omega}{c})^2}|y|}}{\sqrt{k_x^2 -(\frac{\omega}{c})^2}},\hspace{3mm}  \text{for} \hspace{1mm}|k_x|>\left| 					\frac{\omega}{c} \right| 
	\end{aligned}$ \normalsize      \\ 
    $G(x,y,z,\omega)$ 				 &  $\frac{1}{4\pi}\frac{\te^{-\ti\frac{\omega}{c}\sqrt{x^2+y^2+z^2}}}{\sqrt{x^2+y^2+z^2}}$ & \scriptsize$-\frac{\ti}{4} H_0^{(2)}\left( \frac{\omega}{c} \sqrt{x^2+y^2} \right) $\normalsize  \\ 
    $g(x,y,z,t)$ 					 &  $\frac{1}{4\pi}\frac{\delta\left( t - \sqrt{x^2+y^2+z^2}/c \right)}{\sqrt{x^2+y^2+z^2}}$  & $\frac{1}{2\pi}\frac{\theta(t - \sqrt{x^2+y^2}/c)}{\sqrt{t^2 - \left(\frac{\sqrt{x^2+y^2}}{c}\right)^2}}$
%\bottomrule 
\label{eq:theory:Greens_fun_representations} 
    \end{tabular}
\end{center}
where $\theta\left( \right)$ denotes the Heaviside step function, $H_0^{(2)}\left( \right)$ is the zeroth order Hankel function of the second kind and $K_0\left( \right)$ is the modified Bessel function of the second kind.
The conditional expressions are introduced in order ensure, that the evanescent contribution increases exponentially. In the followings for the sake of brevity the Green's function is expressed only in the propagation region, however it must be kept in mind, that the wavenumber components are defined in the evanescent region as it was given by \eqref{eq:theory:k_y_definition}.

\vspace{3mm}
In 3-dimensions the 2-dimensional Green's function represents an infinite line source along the $z$-axis, that can be described as a continuous linear distribution of 3D point sources---explaining directly the infinite tail of the 2D impulse response: it is given by the sum of 3D impulse responses, delayed and attenuated depending on how far the actual impulse arrives from, and multiplied by 2, since along the vertical line source two point elements contribute to the total field at each time instant--, thus the relation between the 3D and 2D Green's function is given as
\begin{equation}
G_{2\mathrm{D}}(x,y,\omega) = \int_{-\infty}^{\infty} G_{3D}(x,y,z,\omega) \td z = \left. \mathcal{F}_{z}\left\{ G_{3D}(x,y,z,\omega) \right\}\right|_{k_z = 0},
\end{equation} 
obviously holding for any other representation, as it is reflected by the table above.

\paragraph{Monopole and dipole sources:\\} 
A frequently used point source model for far-field radiation is the acoustic monopole, which is defined as a pulsating sphere, with its radius decreased to infinitesimal, with the total velocity over the surface held constant \cite{Howe2007}. If the surface vibrates with a velocity $q_{\mathrm{m}}(t)$, the field of a monopole is a point excitation in the velocity potential, i.e. solves
\begin{equation}
\nabla^2 p_{\mathrm{m}}(\vx|\vxo,t) - 	\frac{1}{c^2}\frac{\partial^2}{\partial t^2}p_{\mathrm{m}}(\vx|\vxo,t) = \rho_0 \frac{\partial}{\partial t} q_{\mathrm{m}}(t)\delta(\vx-\vxo),
\end{equation}
and using the temporal derivative of the Green's function the field of a harmonic monopole reads in the frequency domain
\begin{equation}
P_{\mathrm{m}}(\vx|\vxo,\omega) = -\frac{\ti \omega \rho_0 Q_{\mathrm{m}}(\omega)}{4\pi}\frac{\te^{-\ti \frac{\omega}{c}|\vx-\vxo|}}{|\vx-\vxo|}.
\end{equation}
$Q_{\mathrm{m}}(\omega)$ is often referred as \emph{monopole strength}. Monopoles constitute a good far-field approximation of sources in the velocity field, e.g. a dynamical loudspeaker. It should be noted, that in sound field synthesis literature virtual sources are often assumed to be monopoles, but described by a pressure point source, i.e.  by the Green's function. This slight misconcept has a little practical effect, since only the frequency response differ of the two point source models. Also, if dynamical loudspeakers are applied to reproduce the field of an acoustic point source, due to the secondary source-virtual source interchangeability (see later) inherently monopoles are reproduced.

\vspace{3mm}
Besides a single omnidirectional point source, the distribution of monopoles are also frequently used models, termed as \emph{multipoles}. In the aspect of the present treatise \emph{dipoles} are of special interest, consisting of two anti-phase monopoles, placed infinitesimally close to each other. The spatial extension of such a source can be described by the derivative of a monopole along the axis of dipole $\mathbf{v}_{\mathrm{d}}$--as the derivative of a Dirac-delta is two anti-phase Dirac-deltas--, and the field of a dipole is yielded as
\begin{equation}
P_{\mathrm{d}}(\vx|\vxo,\omega) = \left< \mathbf{v}_{\mathrm{d}} \cdot \nabla P_{\mathrm{m}}(\vx,\omega) \right>= -\frac{\ti \omega \rho_0 Q_\mathrm{d}(\omega)}{4\pi}\frac{\left<\mathbf{v}_{\mathrm{d}}\cdot (\vx-\vxo) \right> } {|\vx-\vxo|}\left(  \frac{1}{|\vx-\vxo|} + \ti \frac{\omega}{c} \right)\frac{\te^{-\ti \frac{\omega}{c}|\vx-\vxo|}}{|\vx-\vxo|}
\end{equation}
and by denoting $\cos \theta = \frac{\left< \mathbf{v}_{\mathrm{d}}\cdot (\vx-\vxo)  \right> } {|\vx-\vxo|}$, where $\theta$ is the angle measured from the dipole axis 
\begin{equation}
P_{\mathrm{d}}(\vx|\vxo,\omega) = \cos \theta \left( \frac{1 + \ti \frac{\omega}{c}|\vx-\vxo|}{|\vx-\vxo|} \right) P_{\mathrm{m}} (\vx,\omega).
\end{equation}
A dipole therefore radiates with a directivity characteristics of $\cos \theta$, with its maximum pointing in the direction of the dipole axis, and attenuating in the far-field approximately according to $\sim 1/|\vx-\vxo|$ ( since in the far-field $\frac{1 + \ti \frac{\omega}{c}|\vx-\vxo|}{|\vx-\vxo|} \approx \ti \frac{\omega}{c}$). $Q_{\mathrm{d}}$ is termed as the \emph{dipole moment}.
 
%\begin{figure}
%	\centering
%	\begin{overpic}[width = .5\columnwidth]{Figures/Theory/dipole_source.png}
%	\end{overpic}
%	\caption{3D dipole}
%	\label{Fig:Theory:dipole_source}
%\end{figure}
%
%\vspace{3mm}
%For the solution of the wave equation in enclosures the Green's function must satisfy the imposed boundary conditions. In these cases the Green's function --except for special cases-- varies with $\vxo$, therefore it is translation variant.
%For an more general treatment please refer to \cite{Spors2005}. 
%When the Green's function satisfies Neumann-boundary conditions (ie. $G(\vx_{S},\omega) = 0$) it is called \emph{Neumann Green's function}, while if it satisfies Dirichlet boundary conditions it is termed as \emph{Dirichlet Green's function}.
%For special geometries the Neumann and Dirichlet Green's functions may be expressed analytically, as we will see in the following section.
%
%\vspace{3mm}
%Besides simple monopoles the superposition of several point sources is of importance in this thesis. The superposition of two anti-phase monopoles placed infinitesimally close to each other forms an acoustic dipole. Mathematically its field is derivad as the directional gradient of the field of a monopole, eg. in the $y$-direction:
%\begin{equation}
%G_d(\vx|\vxo,\omega) =
%\frac{\partial}{\partial y} G(\vx|\vxo,\omega)|= 
%- \frac{y}{4\pi}\left( \frac{1}{| \vx - \vxo |} + \ti \frac{\omega}{c} \right) 
%\frac{\te^{-\ti \frac{\omega}{c}  | \vx - \vxo | }}{| \vx - \vxo |^2}.
%\end{equation}
%The sound field of this dipole is depicted in figure \ref{Fig:Theory:dipole_source}.

\newpage
\section{Boundary integral representation of sound fields}

\subsection{The Kirchhoff-Helmholtz integral equation}
In the foregoing of this chapter we were dealing with the free-space solutions of the wave equation. In the following the presence of enclosures will be investigated.

Any sound field obeying the homogeneous Helmholtz-equation may be written in the form of a surface integral above an enclosing surface, termed as the \emph{Kirchhoff-Helmholtz integral equation}. This integral formulation, which solves the homogeneous wave equation with inhomogeneous boundary conditions is of central importance in the field of acoustics, eg. forms the backbone of the Boundary Element Method, and SVD-based Conformal Nearfield Acoustic Holography.

In this chapter we investigate the integral formulation of interior problems in source-free volumes.
Note, that the effect of direct sources may be straightforwardly included in the following results by the proper addition of the solution of the inhomogeneous Helmholtz equation. See \cite{Spors2005} for the examination of this case.
\begin{figure}[!h]
	\centering
	\begin{overpic}[width = .65\columnwidth]{Figures/Theory/Kirchhoff-Helmholtz.png}
	\end{overpic}
\caption{Kirchhoff-Helmholtz integral geometry}
	\label{Fig:Theory:HIE_geometry}
\end{figure}

Let $\Omega$ be a 3D volume, bounded by the surface $\partial \Omega$ with arbitrary position vectors $\vxo$ and $\vx$. Refer to figure xy for the geometry. For two continuous, differentiable scalar valued functions $\Phi(\vxo)$, $\Psi(\vxo)$ the Green's theorem reads (see \ref{App:Green_theorem} for the derivation)
\begin{equation}
\iiint_{\Omega}          \left(  \Phi(\vxo) \nabla^2 \Psi(\vxo) - \Psi(\vxo) \nabla^2 \Phi(\vxo)   \right)   \td \Omega = 
\iint_{\partial \Omega}  \left(  \Psi(\vxo) \frac{\partial \Phi(\vxo)}{\partial n}  - \Phi(\vxo) \frac{\partial \Psi(\vxo)}{\partial n}  \right)   \td \partial \Omega,
\label{Eq:Theory:Greens-theorem}
\end{equation}
with $\frac{\partial}{\partial n}$ denoting the inward normal derivative. Let these functions satisfy the homogeneous and inhomogeneous Helmholtz-equations:
\begin{equation}
(\nabla^2 + k^2)\Phi(\mathbf{x_0}) = 0,
\label{Eq:Theory:Phi}
\end{equation}
\begin{equation}
(\nabla^2 + k^2)\Psi(\vx|\vxo) = -\delta(\vx - \vxo).
\label{Eq:Theory:Psi}
\end{equation}
Although the continuity requirement is not fulfilled for $\Psi$, with proper mathematical workaround the singularity may be excluded from the volume integral\cite{Williams1999}. By setting the homogeneous solution of \eqref{Eq:Theory:Psi} to 0 and assuming acoustically transparent boundary the particular solution is given by the free-field Green's function
\begin{equation}
\Psi(\vx | \vxo) = G(\vx| \vxo,\omega) = \frac{1}{4\pi} \frac{\te^{-\ti \frac{\omega}{c} |\vx-\vxo|}}{|\vx-\vxo|},
\end{equation}
describing the field of a point source located at $\vx$ measured at $\vxo$. For acoustic problems the scalar valued function satisfying the Helmholtz equation is generally the pressure field, thus $\Phi(\vxo) = P(\vxo,\omega) $
 
Let's combine equations \eqref{Eq:Theory:Greens-theorem}-\eqref{Eq:Theory:Phi}-\eqref{Eq:Theory:Psi}!
\begin{equation}
\iiint_{\Omega} - P(\vxo) \delta(\vx - \vxo)
  \td \Omega(\vxo) = 
\iint_{\partial \Omega}  \left(  G(\vx|\vxo) \frac{\partial P(\vxo)}{\partial n}  - P(\vxo)  \frac{\partial G(\vx|\vxo)}{\partial n}  \right)   \td \partial \Omega ( \vxo),
\end{equation}

The sifting property of the Dirac-delta may be exploited, by taking into account, that the singularity is located in the enclosure: if $\vxo$ lies outside the volume the integral is identically zero, while if it is on the surface it is assumed, that "only half of the Dirac-impulse is in the volume". As a result the \emph{Kirchhoff-Helmholtz integral}, or \emph{Helmholtz integral equation (HIE)} is obtained:
\begin{equation}
\alpha P(\vx,\omega) = 
\iint_{\partial \Omega}  \left( 
P(\vxo,\omega)  \frac{\partial G(\vx|\vxo,\omega)}{\partial n}  -  
G(\vx|\vxo,\omega) \frac{\partial P(\vxo,\omega)}{\partial n} 
\right)   \td \partial \Omega ( \vxo),
\label{Eq:Theory:Kirchhoff-Helmholtz}
\end{equation}
with
\begin{equation*}
\alpha = \begin{cases} 
1           & \hspace{1mm} \forall \hspace{5mm}  \vx \in \Omega_i  	   \\
\frac{1}{2} & \hspace{1mm} \forall \hspace{5mm}  \vx \in \partial \Omega  \\
0 			& \hspace{1mm} \forall \hspace{5mm}  \vx \in \Omega_e.
\end{cases}
\end{equation*}
Point $\vx$ is termed as \emph{evaluation point}, while $\vxo$ is termed the \emph{field point}. Utilizing the Euler's equation \eqref{Eq:Theory:Eulers_equation}
\begin{equation}
\frac{\partial P(\vxo,\omega)}{\partial n} = \ti \rho_0 c k V_{\mathrm{n}}(\vxo,\omega),
\end{equation}
ie. the normal component of the velocity on the surface:
\begin{equation}
\alpha P(\vx,\omega) = 
\iint_{\partial \Omega}  \left(  
P(\vxo,\omega)  \frac{\partial G(\vx|\vxo,\omega)}{\partial n}  -
\ti \rho_0 c k V_{\mathrm{n}}(\vxo,\omega) G(\vx|\vxo,\omega) 
\right)   \td \partial \Omega ( \vxo),
\label{Eq:Theory:Kirchhoff}
\end{equation}

The equation states that the pressure field inside an enclosure is completely determined by the boundary conditions for the pressure and normal velocity on the boundary surface.
The interior HIE describes the pressure field only inside the volume of investigation, outside the volume the pressure field is identically zero. For exterior radiation problems the exterior HIE can be derived in a very similar manner --By calculating with the outward normal velocity on the surface and positioning the Dirac-delta outside the enclosure--, and the formulation will result in correct pressure field outside the volume and zero pressure inside \cite{Williams1999}.
In both cases the method is capable of dealing only with forward propagation problems.
\vspace{3mm}

It should be noted, that HIE is consist of two integral components:
\begin{equation}
S_{\mathrm{monopole}}(\vx,\omega) = 
\iint_{\partial \Omega}  \frac{\partial P(\vxo)}{\partial n} G(\vxo|\vx) \td \partial \Omega (\vxo),
\label{Eq:Theory:Single_Layer_Potential}
\end{equation}
\begin{equation}
S_{\mathrm{dipole}}(\vx,\omega) = 
\iint_{\partial \Omega}  P(\vxo,\omega)  \frac{\partial G(\vx|\vxo,\omega)}{\partial n}   \td \partial \Omega (\vxo).
\label{Eq:Theory:Double_Layer_Potential}
\end{equation}
The components are termed \emph{single layer potential} and \emph{double layer potential} respectively in the field of potential theory. The terminology represents the fact, that single layer potential describes the field as the weighted sum of a single layer of monopoles, characterized by $ G(\vxo|\vx) $. On the other hand double layer potential describes the field of an ensemble of dipoles whose field is described by 
$\frac{\partial G(\vx|\vxo,\omega)}{\partial n}$ and which can be physically realized by two anti-phase monopoles, thus by a double layer.

\vspace{3mm}
One drawback of interior HIE is that it overspecifies the problem in order to ensure zero pressure and velocity outside the domain of interest. In the aspect of Sound Field Synthesis the presence of both single and double layer potentials is infeasible. By letting the sound field non-zero outside the enclosure it is possible to completely describe the sound field in the region of interest in terms of only single or double layer potentials.
This can be done by modifying the Green's function in order to satisfy Dirichlet or Neumann boundary conditions, or to impose these boundary conditions to the sound field $ P(\vxo,\omega) $ itself in an equivalent scattering problem (leading to the simple source formulation).
In the following section the former approach is applied for planar boundaries resulting in the Rayleigh integral theorem.

\subsection{The Rayleigh-integrals}
\label{Section:Theory:Rayleigh}

The Rayleigh integrals formulate the sound field with merely the pressure field or the normal velocity measured on an infinite plane. The derivation utilizes the Neumann and Dirichlet Green's functions for the geometry, that can be seen in figure \ref{Fig:Theory:Rayleigh_geometry}.

In this scenario we investigate an interior problem with the volume bounded by a plane and a hemisphere: the HIE is written into this two surfaces. As we increase the radius of the hemisphere to infinity ($r_s \rightarrow \infty$) the Sommerfeld-radiation condition is invoked and the integral on the sphere vanishes: the radiated field is described by a surface integral written on the infinite plane. Also by noticing that $\frac{\partial}{\partial n} = \frac{\partial}{\partial y_0}$:
\begin{multline}
P(\vx,\omega) = \lim_{r\rightarrow \infty} \left( \int_{\partial \Omega_P} + \int_{\partial \Omega_S} \td \partial \Omega \right) = \\
\iint_{\partial \Omega_S}  \left( 
P(\vxo,\omega)  \frac{\partial G(\vx|\vxo,\omega)}{\partial y_0}  -
\ti \rho_0 c k V_{\mathrm{n}}(\vxo,\omega)  G(\vx|\vxo,\omega) 
\right)   \td \partial \Omega_S ( \vxo).
\end{multline}

It can be easily proven, that any homogeneous solution of the Helmholtz equation -- satisfying free field boundary conditions -- may be added to the Green's function, the Kirchhoff-Helmholtz integral still holds. In order to eliminate either the single or the double layer potential in the HIE we set the homogeneous solution in the Green's function to non-zero.\begin{itemize}
\item \emph{Neumann Green's function} eliminate the double layer potential, by describing Neumann boundary conditions for the Green's function on the bounding infinite plane:
\begin{equation}
G_N = G + g_N,
\end{equation}
\begin{equation}
\frac{\partial G_N}{\partial n}|_{\partial \Omega_P} = 0.
\label{Eq:Theory:Neumann_Greenfun_def}
\end{equation}
\item \emph{Dirichlet Green's function} eliminate the double layer potential in the same manner by prescribing
\begin{equation}
G_D(\vxo) = G + g_D = 0, \hspace{3mm} \forall \hspace{3mm} \vxo \in \partial \Omega_S.
\end{equation}
\end{itemize}

\begin{figure}
	\centering
	\begin{overpic}[width = .5\columnwidth]{Figures/Theory/Rayleigh_integral.png}
	\end{overpic}
	\caption{Rayleigh geometry}
	\label{Fig:Theory:Rayleigh_geometry}
\end{figure}
\vspace{3mm}
First we are looking for the Neumann's Green function for the planar geometry under discussion:
For the free field Green's function the partial derivative on the plane $y_0 = 0$ is given by
\begin{equation}
\frac{\partial}{\partial y} G(\vx|\vxo,\omega)|_{y_0 = 0} = 
- \frac{y}{4\pi}\left( \frac{1}{| \vx - \vxo |} + \ti \frac{\omega}{c} \right) 
\frac{\te^{-\ti \frac{\omega}{c}  | \vx - \vxo | }}{| \vx - \vxo |^2}
\label{Eq:Theory:monopole_y_derivative}
\end{equation}

The construction of Neumann's Green function is now straightforward. In order ensure that \eqref{Eq:Theory:Neumann_Greenfun_def} is fulfilled the solution must have the form
\begin{equation}
\frac{\partial}{\partial y} g_N(\vx|\vxo,\omega)|_{y_0 = 0} = \frac{\partial}{\partial y} G(\vx|\vxo,\omega)|_{y_0 = 0},
\end{equation}
which is the normal derivative of a point source, positioned at $-y$, thus the image of $G(\vx|\vxo,\omega)$ mirrored on the infinite plane. Since this mirror singularity lies outside the domain of investigation, therefore in the $y > 0$ volume it satisfies the homogeneous Helmholtz equation. The Neumann Green's function on the plane then takes the form
\begin{equation}
G_N(\vx|\vxo, \omega)|_{y = 0} = 
2G(\vx|\vxo, \omega)|_{y = 0} = 
\frac{1}{2\pi} \frac{\te^{-\ti \frac{\omega}{c} |\vx-\vxo|}}{ |\vx-\vxo| }
\end{equation}
and the HIE is simplfied into the \emph{Rayleigh's first integral} \cite{Berkhout1984}
\begin{multline}
P(\vx,\omega) =
- 2 \iint_{\partial \Omega_S} \frac{\partial P(\vxo)}{\partial n} G(\vxo|\vx) \td \vxo 
=
 - \frac{\ti \rho_0 c k}{2\pi} \iint_{\partial \Omega_S} V_{\mathrm{n}}(\vxo,\omega)   \frac{\te^{-\ti \frac{\omega}{c} |\vx-\vxo|}}{ |\vx-\vxo| } \td \vxo.
\label{Eq:Theory:RayleighI}
\end{multline}

\vspace{3mm}
The construction of Dirichlet Green's function is simple: by driving the mirror source anti-phase $G_D = G + g_D = 0$ is obtained on the surface, while for the derivative
\begin{equation}
\frac{\partial}{\partial y} G_N(\vx|\vxo,\omega)|_{y_0 = 0} = 2 \frac{\partial}{\partial y} G(\vx|\vxo,\omega)|_{y_0 = 0}
\end{equation}
holds. Substituting that into the HIE the double-layer potential is eliminated and \emph{Rayleigh's second integral} is yielded:
\begin{equation}
P(\vx,\omega) = 
2 \iint_{\partial \Omega_S}  P(\vxo,\omega)  \frac{\partial G(\vx|\vxo,\omega)}{\partial y_0}     \td \vxo.
\label{Eq:Theory:RayleighII}
\end{equation}

The Rayleigh I integral is of major importance in the aspect of Sound Field Synthesis, and in the theory of diffraction from finite aperture. It is also extensively used in the calculation of radiated fields from finite radiators, mounted in infinite walls, eg. field of loudspeakers. It states that the radiated field from a rigid vibrating plane can be calculated by summing the field of monopoles, driven by the normal velocity distribution, or mathematically speaking: by convolving the Green's function with the velocity distribution over the infinite surface.

\vspace{3mm}
Finally inverse Fourier-transform with respect to time leads us to the time domain version of the Rayleigh I integral\cite{Pierce1991}:
\begin{equation}
p(\vx,t) = \frac{\rho_0}{2\pi} \iint_{\partial \Omega_S} \frac{\partial}{\partial t} \frac{v_{\mathrm{n}}(\vxo,t-\frac{ | \vx-\vxo | }{c})}{| \vx-\vxo |} \td \vx_0,
\end{equation}
Note, that since the Rayleigh integral describes the field of an ensemble of point source on the plane $y = 0$, therefore the pressure field is the solution for the following equation \cite{Pierce1991}
\begin{equation}
\nabla^2 p(\mathbf{x},t) - \frac{1}{c^2}\frac{\partial^2 p(\mathbf{x},t)}{\partial t^2} = -2\rho_0 \frac{\partial}{\partial t} v_{\mathrm{n}}(x,z,t)\delta(y).
\end{equation}

\newpage
