A hangtér-reprodukció célja egy tetszőleges virtuális akusztikai környezet, a virtuális hangtér fizikai jellemzőinek visszaállítása kiterjedt megfigyelői terület mentén, a megfigyelési területet határoló hangszórósokaság, az ún. másodlagos forráseloszlás megfelelő vezérlésével. 
A reprodukció alapfeladata ezen hangszóró-vezérlőfüggvények meghatározása úgy, hogy az egyes hangszóróelemekből származó hullámok összege megegyezzen az elérendő virtuális térrel.
Két alapvető módszer létezik a probléma megoldására:
Az explicit megoldás célja a szintetizált teret leíró integrálegyenlet közvetlen megoldása a vezérlőfüggvényre vonatkozóan, amely egy megfelelő spektrális dekompozícióval érhető el.
Ezzel szemben az implicit megoldás, vagy elterjedtebb nevén a hangtérszintézis (Wave Field Synthesis) alapja a virtuális hangteret peremfeltételek alapján leíró kontúrintegrál meghatározása, amely implicite magában tartalmazza a keresett vezérlőfüggvényeket.

A jelen disszertáció a hangtérszintézis elméletét tárgyalja újra a kiindulási alapoktól kezdve a technika általánosításának érdekében.
Fontos eredményként hangszóró-vezérlőfüggvényeket kerülnek bemutatásra, amelyek alkalmazásával egy tetszőleges virtuális hangtér előállítható szabadon választott másodlagos forráskontúrral és a szintézis tetszőleges referenciagörbére optimalizálható, amely görbe mentén amp\-li\-tú\-dó\-he\-lyes szintézis érhető el.

Ezután az alapvetően a spektrális tartományban adott explicit megoldás nagyfrekvenciás, tértartománybeli közelítése kerül bemutatásra, amely megoldás a bemutatott hangtérszintézis módszerrel megegyező alakra hozható. 
Ez tehát bizonyítja a két módszer nagyfrekvenciás, aszimptotikus ekvivalenciáját általános szintézis problémákra.
A dolgozat ezen felül egy egyszerű aszimptotikus átlapolódásgátló-szűrési stratégiát is bemutat, amely alapján a hangszórósor diszkrét jellege miatt jelenlévő másodlagos, térbeli-átlapolódó hullámfrontok hatékonyan elnyomhatók.

Végezetül összetett példaként a leírtak alkalmazására a mozgó források által keltett hangterek szintézisének kérdései kerülnek tárgyalásra.
A dolgozatban bemutatott módszerek egyszerűen kiterjeszthetőek mozgó források szintézisére:
hangtérszintézis és explicit vezérlőfüggvények kerülnek bevezetésre erre a dinamikus, idővariáns esetre is, az átlapolódásgátló-szűrési stratégia kiterjesztésével együtt.