	%\section{The problem formulation}
\begin{figure}[b!]
	\centering
	\begin{overpic}[width = .8\columnwidth]{Figures/SFS_theory/general_sfs.png}
	\small
	\put(0,26){virtual source}
	\put(44.5,0.5){$\mathbf{0}$}
	\put(71,31){$\vx$}
	\put(43,15){$\vxo$}
	\begin{turn}{27}
	\put(57,-3){$|\vx - \vxo|$}
	\end{turn}
	\put(50,35){$\Omega$}
	\put(80,20.5){$\dO$}
	\end{overpic}
	\caption{Geometry for the general sound field synthesis problem}
	\label{Fig:Theory:general_sfs_geometry}
\end{figure}

In the followings the general sound field synthesis (SFS) problem is formulated. 
Consider a source-free volume $\Omega \subset \mathbb{R}^3$, bounded by a continuous set of acoustic sources forming the boundary surface $\dO$.
The enclosing source ensemble is termed the \emph{secondary source distribution (SSD)}.
%Obviously, in the aspect of practical applications only 3D problems are of importance ($n=3$).%, however for the sake of computational simplicity several simulations 
The general geometry is depicted in Figure \ref{Fig:Theory:general_sfs_geometry}.
For the sake of simplicity it is assumed that the boundary is acoustically transparent and the secondary sources are acoustic point sources described by the free field Green's function. 
Unless it is denoted otherwise, $G(\vx,\omega)$ refers to the 3D Green's function in the followings.
Since dynamic loudspeakers can be modeled as 3D monopoles in the low-frequency region, this choice of SSD elements is reasonable. 

With these assumptions the synthesized pressure at any receiver position $\vx \in \Omega$ is given by the superposition of the fields of individual SSD elements, written as a single layer potential \cite{Ahrens2012,Ahrens2010phd,Wierstorf2014,Schultz2014:Comparing_approaches}:
\begin{equation}
P(\vx,\omega) = \oint_{\dO} D(\vxo,\omega) G(\vx - \vxo , \omega ) \td \dO ( \vxo ).
\label{Eq:Theory:3D_SFS}
\end{equation}
The weighting factor $D(\vxo,\omega)$ is termed the \emph{driving function} for the given SSD. 
The sound field synthesis problem can be formulated as the following:
Given a \emph{target sound field} or the sound field of a \emph{virtual source} $P(\vx,\omega)$, the aim is to solve the integral equation for $D(\vxo,\omega)$, so that the superposition of the SSD's sound field---the \emph{synthesized field}---equals to the target sound field. 
The problem is therefore an inverse problem with a unique solution for general enclosures.

Comparing the general SFS formulation \eqref{Eq:Theory:3D_SFS} with the Kirchhoff-Helmholtz integral \eqref{Eq:Theory:Kirchhoff-Helmholtz} it becomes clear that SFS with a single layer SSD is not able to ensure identically zero sound field outside the enclosure.
Practically, the dipole sources that would cancel the field of the monopoles outside the volume are removed from the surface.
In the present thesis free-field conditions are assumed: the exterior sound field satisfies the Sommerfeld radiation condition, thus the effects of the listening environment in practical applications (i.e. wall reflections) are not considered. 
For the inclusion of room effects to the SFS problem refer to \cite{Spors2005}.

\vspace{3mm}
The above discussed general 3D SFS setup requires an enclosing surface of 3D point sources, making practical implementations hardly realizable.
In practice it is often sufficient to restrict the reproduction to the $z=0$ plane, containing the 2D contour of secondary sources.
This reproduction scenario is termed \emph{2.5D synthesis}, referring to the fact that although the problem dimensionality is reduced to $n=2$, still the 2D SSD contour consists of 3D point source elements.
In this geometry the general 2.5D synthesis problem is formulated as
\begin{equation}
P(\vx,\omega) = \oint_{C} D(x_0,y_0,0,\omega) G(x - x_0, y-y_0, 0 , \omega ) \td s( x_0, y_0 ),
\label{Eq:Theory:25D_SFS}
\end{equation}
where $C(x_0,y_0)$ is the SSD contour and $\td s$ is the arc length.
Obviously, neither 2D nor 3D sound fields can be perfectly synthesized in this geometry due to dimensionality inconsistency between the target field and the SSD.
Overcoming the artifacts of this \emph{dimensionality mismatch} is the central question of practical sound field synthesis and is the main topic of the present chapter.

In the followings this chapter presents the approaches to solve the 3D and 2.5D SFS problem including physically based implicit and particularly mathematical explicit solutions. 

\section{Implicit solution: Wave Field Synthesis}
% TODO: Implicit solution with HF approx -> local solution

\subsection{3D Wave Field Synthesis}

The implicit solution for the SFS problem aims at the derivation of an appropriate single layer potential representation of the target sound field, containing the required SSD driving functions implicitly.
In case of a 3D SFS problem obtaining the implicit solution is straightforward, based on the boundary integral representations discussed in the previous chapters.

Assume a general enclosing 3D SSD surface consisting of 3D point sources.
Comparing the Kirchhoff approximation of the Kirchhoff-Helmholtz integral \eqref{Eq:SFS_theory:Kirchhoff_appr} or \eqref{Eq:HF_appr:Kirchhoff_approximation} with the general SFS equation \eqref{Eq:Theory:3D_SFS} reveals, that the Kirchhoff approximation implicitly contains the driving functions for a general enclosing SSD. 
The driving function is given by
\begin{equation}
D(\vxo,\omega) = - 2w(\vxo)\frac{\partial P(\vxo,\omega)}{\partial \vni}, 
\label{Eq:Theory:2D_3D_WFS_driv_fun}
\end{equation}
or making use of the high frequency gradient approximation
\begin{equation}
D(\vxo,\omega) = 2 w(\vxo) \ti k_{\mathrm{n}}^P(\vxo) P(\vxo,\omega)
\label{Eq:Theory:2D_3D_WFS_driv_fun_2}
\end{equation}
with $k_{\mathrm{n}}^P(\vxo)$ being the normal component of the target field's local wavenumber vector taken on the SSD and $w(\vxo)$ being the window function, as introduced in the previous chapter:
\begin{equation}
w(\vxo) = \begin{cases}
                        1, \hspace{3mm} \forall \hspace{3mm} \langle \mathbf{k}^P(\vxo) \cdot \mathbf{n}_{\text{in}}(\vxo) \rangle > 0 \\
                        0  \hspace{3mm} \text{elsewhere}.
                    \end{cases}
\end{equation}
The driving function \eqref{Eq:Theory:2D_3D_WFS_driv_fun_2} is a common generalization of the 3D WFS driving function given by \cite[Eq. 20.]{Zotter2013:uniqueness} for a virtual point source.
Both formulations are valid in the high frequency region for convex SSDs within the validity of the Kirchhoff approximation: in the far-field of the virtual source distribution generating the virtual sound field---i.e. where the local plane wave approximation of the virtual field holds.
In the context of WFS the windowing is termed \emph{secondary source selection} \cite{Spors2007, Spors2007:DAGA:SS_selection_criterion}, selecting the \emph{active secondary sources} contributing to the synthesized field. 

\vspace{3mm}
In the special case of an infinite planar SSD surface located along the plane $y = y_0$, the Kirchhoff-Helmholtz integral degenerates to the Rayleigh I integral representing the field of any source distribution located at $y<y_0$ in terms of a single layer potential, as discussed in \ref{Section:Theory:Rayleigh}.
Therefore in this geometry driving function \eqref{Eq:Theory:2D_3D_WFS_driv_fun} is capable of the perfect synthesis of an arbitrary virtual sound field in the listening half-space $y>y_0$ without any approximations involved.
In this case no windowing is required, i.e. $w(\vxo) \equiv 1$ and the normal derivative is simply given by the $y$-derivative of the target/virtual sound field.

As it was discussed in \ref{Sec:HF_approx:1D_Greens} the frequency response of the 1D Green's function---representing an infinite planar distribution of 3D point sources---is proportional to $\sim \frac{1}{\ti k} = \frac{c}{\ti \omega}$ expressing an infinite impulse response of a step function, which performs the integration of the source time history. 
In case of a non-homogeneous spatial distribution a directivity factor is also present expressed locally as $\sim \frac{c}{\ti \omega \hat{k}^P_y(\vxo)}$ with $\vk^P$ being the normalized local wavenumber vector, indicating that a surface of point sources generates larger pressure field into lateral directions than into the normal direction (see \eqref{eq:HF_approx:Greens_2D_Spectrum} and \eqref{Eq:HF_approx:H_det_Rayleigh} for its local approximation).
The factor $\ti k_{\mathrm{n}}^P(\vxo)$ in \eqref{Eq:Theory:2D_3D_WFS_driv_fun_2} and in the Rayleigh/Kirchhoff-Helmholtz integral may be therefore interpreted as a correction term, ensuring flat frequency response for the SSD surface by inverse filtering (taking the time derivative) of the excitation signal, and compensating for its directivity.

\subsection*{Application example: 3D synthesis of a virtual point source}

\begin{figure}  
\small
  \begin{minipage}[c]{0.64\textwidth}
	\begin{overpic}[width = 1\columnwidth ]{Figures/SFS_theory/3D_WFS_general.png}
	\small
	\put(2,53){(a)}
	\put(2,1){(b)}
	\end{overpic}   \end{minipage}\hfill
	\begin{minipage}[c]{0.35\textwidth}
    \caption{3D synthesis of a 3D point source located at $\vxs = \posvec{3}{0.4}{2.5}{0}$, radiating at $\omega_0 = 2\pi \cdot 1.5 \mathrm{krad}/s$.
    The SSD surface is chosen to be independent of the $z$-coordinate as illustrated in \ref{fig:SFS_theory:WFS_geometry}.
	Obviously, instead of a vertically infinite SSD, for the numerical calculation the SSD was truncated along the vertical dimension by choosing parameters, so that the diffraction effects due the truncation are negligible in the simulation results.
    Figure (a) depicts the real part of the synthesized field and part (b) presents the absolute error of synthesis (the discrepancy between the synthesized and the target sound field) in a logarithmic scale measured in the horizontal plane, containing the virtual point source.
	The active arc of the SSD is denoted by solid black line, and the inactive part with dotted by black line.
    }
\label{fig:SFS_theory:3D_WFS_general}  \end{minipage}
\end{figure}

As a simple example the 3D WFS of a virtual point source is discussed.
Assume a 3D point source located at $\vxs = \posvec{3}{x_s}{y_s}{z_s}$.
Substituting the Green's function into \eqref{Eq:Theory:2D_3D_WFS_driv_fun_2} yields the point source specific 3D WFS driving function
\begin{equation}
\label{Eq:SFS_theory:3D_WFS_ps_driv_fun}
D(\vxo,\omega) = w(\vxo)  \frac{\ti k }{2\pi} \frac{\left< \vxo-\vxs \cdot \vni(\vxo) \right> }{|\vxo-\vxs|} \frac{\te^{-\ti k |\vxo-\vxs|}}{|\vxo-\vxs|},
\end{equation}
begin equivalent to \cite[Eq. 20.]{Zotter2013:uniqueness} and \cite[Eq. 19.]{Spors2008:WFSrevisited}.

The result of synthesis is depicted in Figure \ref{fig:SFS_theory:3D_WFS_general} for the special case of an SSD surface, being invariant to translation along the $z$-axis, as illustrated by Figure \ref{fig:SFS_theory:WFS_geometry} in the following section.
The driving functions ensure amplitude correct synthesis within the validity of the Kirchhoff-approximation:
amplitude errors arise 
\begin{itemize}
\item in the proximity of SSD elements with large local curvature of the SSD surface, due to the local failure of the tangent plane approximation
\item in the proximity of SSD elements where the normal component of the local wavenumber vector is small---i.e. parts of the SSD, nearly parallel to the local virtual field propagation direction--- since at these positions the high frequency gradient approximation fails, with also the lack of diffractional waves causing amplitude errors.
\item at space regimes for which the above described SSD elements serve as a stationary position as discussed in section \ref{Sec:HS_approx:SPA_for_Rayleigh}.
\end{itemize}

Driving function \eqref{Eq:SFS_theory:3D_WFS_ps_driv_fun} expresses the loudspeaker driving signals for a virtual point source radiating at a single frequency component.
Assuming a wideband source excitation with the time history given by $s(t)$ and its frequency content $S(\omega)$ the time domain WFS driving functions are given by the inverse temporal Fourier transform of \eqref{Eq:SFS_theory:3D_WFS_ps_driv_fun} weighted by the excitation spectrum.
\begin{equation}
d(\vxo,t) = \frac{w(\vxo)}{2\pi} \int_{\infty}^{\infty} \frac{\left< \vxo-\vxs \cdot \vni(\vxo) \right> }{|\vxo-\vxs|} \frac{\ti \omega }{2\pi c} S(\omega) \frac{\te^{ \ti \omega ( t - \frac{|\vxo-\vxs|}{c} ) }}{|\vxo-\vxs|} 
\td \omega.
\end{equation}
By realizing that $\ti \omega S(\omega)$ describes the temporal derivative of the source time history and exploiting the Fourier transform shift theorem, the time domain 3D WFS driving functions are obtained for a virtual point source as
\begin{equation}
\label{Eq:SFS_theory:3D_WFS_ps_driv_fun_td}
d(\vxo,t) = \frac{\left< \vxo-\vxs \cdot \vni(\vxo) \right> }{|\vxo-\vxs|} \frac{w(\vxo)}{2\pi c} \frac{s'_t(t - \frac{|\vxo-\vxs|}{c} )}{|\vxo-\vxs|},
\end{equation}
where differentiation with respect to time compensates for the the effects of surface integrating the Green's function, as discussed above.

\subsection{The 2.5D Kirchhoff approximation}

Before getting involved with the questions of 2.5D Wave Field Synthesis a further simplification of the Kirchhoff approximation is introduced.
This simplification reduces the 3D Kirchhoff integral into a 2D contour integral representing a 3D sound field as the superposition of 3D Green's functions.
The approximation is therefore referred to as the \emph{2.5D Kirchhoff integral}, frequently occurring in the field of seismic migration and inversion problems.
The dimensionality reduction is performed by applying the stationary phase approximation to the Kirchhoff integral along the vertical dimension.

Assume a 3D interior radiation problem, with the sound field under consideration described by the Kirchhoff integral \eqref{Eq:HF_appr:Kirchhoff_approximation} written on a surface, being translation invariant along the $z$-axis.
The problem geometry is depicted in Figure \ref{fig:SFS_theory:WFS_geometry}.
In the followings the receiver position is assumed to be at $z=0$ inside the enclosure at $\vx = \posvec{3}{x}{y}{0} \in \Omega$.
%
\begin{figure}  
\begin{minipage}[c]{0.6\textwidth}
  \hspace{0cm}
	\begin{overpic}[width = 1\columnwidth ]{Figures/SFS_theory/WFS_geometry.png}
	\small
	\put(82,51){$x$}
	\put(91.5,33){$y$}
	\put(95,65.5){$z$}
	\put(48,35.5){$\vx$}
	\put(65,45.5){$\vxo$}
	\put(7,22){plane of interest}
	\put(30,8){$\dO$: 3D surface}
	\put(48,24.5){$C$: 2.5D contour}
	\end{overpic}  \end{minipage}\hfill
	\begin{minipage}[c]{0.37\textwidth}
    \caption{
    Geometry for the derivation of 2.5D Kirchhoff integral.
The enclosing surface $\dO(x_0,y_0)$ is chosen to be independent of the $z$-coordinate in order to be able to evaluate the Kirchhoff integral with respect to $z_0$ using the SPA. 
If the sound field to be described is a 2D one propagating in the direction parallel to the listening plane, then the surface can be interpreted as a continuous set of infinite vertical line sources along $C$, capable of the perfect description of a 2D field inside the enclosure by a 2D countour integral.}
\label{fig:SFS_theory:WFS_geometry}  
\end{minipage}
\end{figure}
%
In this special geometry the integral variables are separable and the Kirchhoff integral can be written as
\begin{equation}
P(\vx,\omega) = 
\oint_{C} \int_{-\infty}^{\infty} 
2 w(\vxo) \ti k_{\mathrm{n}}^P(\vxo) 	
P(\vxo,\omega) G(\vx-\vxo,\omega) \, \td z_0 \, \td s(x_0, y_0),
\label{Eq:SFS_theory:Kirchhoff_spec_geom}
\end{equation}
with the integral variable $\td s$ being the arc length along the contour $C = \dO(x_0,y_0,0)$.

The integral along $z_0$ is approximated applying the stationary phase approximation.
Since the contour of integration is chosen to lie at the $z=0$ plane, therefore the vertical stationary position has to be found at $z_0^* = 0$.
Based on the foregoing this requirement can be formulated as
\begin{equation}
k_z^P(x_0,y_0,0) = k_z^G(x-x_0,y-y_0,0) = 0,
\end{equation}
stating the trivial fact, that a sound field can be described by a 2.5 dimensional contour integral only in the plane, where all the sound sources are located, and which plane the emerging waves propagate parallel with.
In the plane of investigation $k_z^P(x,y,0) \equiv 0$ holds, being valid for 3D sources located at the plane of investigation and for 2D sources being invariant along the vertical dimension.
Throughout the present thesis when dealing with 2.5D synthesis problems exclusively this types of virtual fields are considered.

Having the vertical stationary position fixed to $z_0^* = 0$ the vertical integral can be approximated by the SPA.
Application of the 1D SPA formulation \eqref{Eq:SPAResult} requires the sign of the phase function's second derivative at the stationary position. 
An important property of wave fields under consideration is that the vertical curvature described by $\phi^{P''}_{zz}(x_0,y_0,0,\omega)$ is one of their principal curvatures itself, denoted by $\Kv^P$.
In the stationary position the principal curvatures are additive, and the second phase derivative is the negative sum of the principal curvatures of sound field $P$ and the Green's function, as discussed in appendix \ref{App:Hessian} in details:
\begin{equation}
\small
\phi''_{zz}(x_0,y_0,0,\omega) = \phi^{P''}_{zz}(x_0,y_0,0,\omega) +\phi^{G''}_{zz}(x-x_0,y-y_0,0,\omega) = -k\left( \Kv^P(\vxo) + \Kv^G(\vx-\vxo) \right).
\label{Eq:SFS_theory:Curvatures_Addition}
\end{equation}
By definition, for an arbitrary diverging sound field the principal curvatures are positive.
For a converging wavefront the signature of the resultant curvature depends on the receiver position $\vx$: in regions of the receiver plane where sound field $P$ locally converges the resultant curvature is negative, while in regions where the sound field diverges, e.g. after passing a focal point the resultant curvature is positive.
In the present thesis only locally diverging wave fields are discussed, ensuring that $\mathrm{sgn} \left( \phi''_{zz}(x_0,y_0,0,\omega) \right) = -1$ holds.

With these considerations application of the SPA to \eqref{Eq:SFS_theory:Kirchhoff_spec_geom} results in the \emph{2.5D Kirchhoff integral}, reading as
\begin{equation}
\small
P(\vx,\omega) = 
\oint_{C}
2 w(\vxo)
\sqrt{\frac{2 \pi}{\ti |\phi^{P''}_{zz}(\vxo,\omega) +\phi^{G''}_{zz}(\vx-\vxo,\omega)|}}
\underbrace{\ti k_{\mathrm{n}}^P(\vxo) 	P(\vxo,\omega) }_{ -\frac{\partial P(\vxo,\omega)}{\partial \vn_{\mathrm{in}}}}
G(\vx-\vxo,\omega) \td s(x_0,y_0), 
\label{Eq:SFS_thery:25_KI}
\end{equation}
with both $\vx = \posvec{3}{x}{y}{0}$ and $\vxo = \posvec{3}{x_0}{y_0}{0}$ now denoting in-plane positions.

\subsection{2.5D Wave Field Synthesis}

The 2.5D Kirchhoff integral implicitly contains the 2.5D WFS driving functions for a convex continuous contour of 3D point sources at the $z = 0$ plane.
%Comparing the expression for the synthesized field \eqref{Eq:Theory:25D_SFS} with \eqref{Eq:SFS_thery:25_KI} implicates that the required driving functions can be extracted from the 2.5D Kirchhoff integral.
The resulting driving functions are however still dependent on the listener position through the argument of $\phi^{G''}_{zz}(\vx-\vxo,\omega)$, which dependency may be avoided by fixing the listener position.
This strategy would only allow the synthesis of the virtual field, optimized to a single, fixed receiver position termed the \emph{reference point}, while in other points in the listening plane amplitude errors would be present.
In the followings it is presented how these driving functions can be further manipulated in order to ensure correct synthesis along an arbitrary receiver curve termed the \emph{reference curve} within the validity of the stationary phase approximation resulting in the \emph{unified 2.5D WFS theory}.

\vspace{3mm}
As it was stated in section \ref{Sec:HS_approx:SPA_for_Rayleigh} for any receiver position $\vx$ the Kirchhoff integral is dominated by that stationary contour element $\vxo^*(\vx)$, from which the emerging spherical wavefronts locally coincide with the target field wavefront, i.e. where $\vk^P(\vxo^*(\vx)) = \vk^G(\vx - \vxo^*(\vx))$ is satisfied.
As a consequence, the 2.5D Kirchhoff integral may be further approximated by expressing the amplitude factor with its value at the stationary position as
\begin{equation}
P(\vx,\omega) = 
\oint_{C}
\! 2 w(\vxo) 
\sqrt{\frac{2 \pi}{\ti |\phi^{P''}_{zz}(\vxo^*(\vx)) +\phi^{G''}_{zz}(\vx-\vxo^*(\vx))|}}
\ti k_{\mathrm{n}}^P(\vxo) 	P(\vxo,\omega)
G(\vx-\vxo,\omega) \td s,
\label{Eq:SFS_theory:25D_KI_appr}
\end{equation}
where $\vxo^*(\vx)$ is defined by the implicit relation above.

The statement can be expressed by reversing causality, forming the main idea of 2.5D WFS theory: 
every point $\vxo$ on the secondary distribution contributes to the total sound field at the set of positions $\vx(\vxo)$ where the local propagation direction of a point source positioned at $\vxo$ coincides with that of the target field, i.e. where their local wavenumber vectors coincide.
Hence $\vx$ and $\vxo$ are \emph{stationary point pairs}, mutually determining each other.
By reversing the causality choosing $\vxo$ as an independent parameter the 2.5D WFS driving functions can be extracted from \eqref{Eq:SFS_theory:25D_KI_appr} resulting in the \emph{unified 2.5D Wave Field Synthesis driving functions}
\begin{equation}
D(\vxo, \omega) = w(\vxo) 
\sqrt{\frac{8\pi}{\ti k}}\sqrt{\dref(\vxo)}
\ti k_{\mathrm{n}}^P(\vxo) 	P(\vxo,\omega),
\label{Eq:SFS_theory:25D_WFS_driv_fun}
\end{equation}
with the term $\dref(\vxo)$ denoting the \emph{referencing function}, defined as
\begin{equation}
\label{Eq:SFS_theory:Referencing function}
\dref(\vxo) = \frac{k}{|\phi^{P''}_{zz}(\vxo) +\phi^{G''}_{zz}(\vxref(\vxo)-\vxo)|} = \frac{1}{\Kv^P(\vxo)+ \Kv^G(\vxref(\vx)-\vxo)},
\end{equation}
where $\Kv^P$ and $\Kv^G$ are the vertical principal curvatures of the involved sound fields.
Position $\vxref(\vxo)$ is a point on a pre-defined \emph{reference curve} $C_{\mathrm{ref}}$, for which $\vxo$ is a stationary position on the SSD defined by
\begin{equation}
\vk^P(\vxo) = \vk^G(\vxref(\vxo) - \vxo),
\label{Eq:SFS_theory:WFS_General_Stat_pos}
\end{equation}
with $\vxref(\vxo) \in C_{\mathrm{ref}}$.
The reference curve must be a smooth convex curve inside the listening region, ensuring that each reference point has a unique stationary point pair.
Once the reference position $\vxref(\vxo)$ is known for each SSD element the WFS driving functions \eqref{Eq:SFS_theory:25D_WFS_driv_fun} can be evaluated.
Referencing the WFS driving function is therefore done by prescribing a unique reference point for each SSD element with the aid of the SPA, so that the set of these reference points form the continuous reference curve.
The resulting driving functions will result in amplitude correct synthesis over the reference curve within the validity of the integral formulation \eqref{Eq:SFS_theory:25D_KI_appr}.

%
\begin{figure}
	\centering
	\begin{overpic}[width = .75\columnwidth]{Figures/SFS_theory/WFS_ref_point.png}
	\small
	\put(31,32){$\vxo$}
	\put(48,25){$\vxref^*(\vxo)$}
	\begin{turn}{-12.5}
	\put(28,33){$\vk^P(\vxo)$}
	\end{turn}
	\begin{turn}{18}
	\put(52,1){reference curve}
	\end{turn}
	\end{overpic}
    \caption{
    Location of the reference position for an SSD element located in $\vxo$.
    Due to the phase characteristics of the Green's function the reference position $\vxref^*(\vxo)$ for an arbitrary SSD element can be found at the intersection of the reference curve and the line emerging from $\vxo$ pointing into the local wavenumber vector of the virtual field $\vk^P(\vxo)$.
	The location of the arbitrarily chosen reference curve is denoted by dashed black line with solid line indicating the positions, for which a stationary SSD position can be found.
	Amplitude correct synthesis may be only achieved along this part of the reference curve.
   }
\label{fig:SFS_theory:WFS_ref_point}  
\end{figure}
%
By substituting the explicit expression for the Green's function's wavenumber ($\vk^G(\vx) = k\frac{\vx}{|\vx|}$) into \eqref{Eq:SFS_theory:WFS_General_Stat_pos} the set of positions for which a given $\vxo$ serves as stationary point reads as
\begin{equation}
\vx = \vxo + \hat{\vk}^P(\vxo) |\vx-\vxo|.
\end{equation} 
The equation describes straight lines passing through $\vxo$ into the direction of the local wavenumber vector of the target sound field $\vk^P(\vxo)$.
Each SSD element therefore dominates the synthesized field towards the direction of the virtual field's local propagation direction at the SSD position.
Along this straight line inside the SSD the virtual field wavefront matches the actual SSD element's wavefront, and the reference position for the actual SSD element is found at the intersection of this straight line and the reference curve.
The location of the reference position for a given SSD element is illustrated in Figure \ref{fig:SFS_theory:WFS_ref_point} for the case of a virtual point source. 
Once the reference position is expressed for each SSD element, the driving functions can be evaluated.

In order to gain a physical interpretation on the structure of the resulting driving functions the referencing function can be expressed based on \eqref{Eq:SFS_theory:Curvatures_Addition} in terms of the vertical principal radii of the virtual field and the Green's function, yielding
\begin{equation}
\small
\label{Eq:SFS_theory:25D_WFS_driv_fun_ver_2}
D(\vxo, \omega) = 
\underbrace{\sqrt{\frac{2\pi \Rv^G(\vxref(\vxo)-\vxo) }{\ti k}}}_{{\substack{\text{SSD}\\\text{compensation}}}}
\underbrace{\sqrt{ \frac{\Rv^P(\vxo) }{\Rv^P(\vxo) +  \Rv^G(\vxref(\vxo)-\vxo) } }}_{{\substack{\text{virtual source}\\\text{compensation}}} 
 \hspace{1mm} = \hspace{1mm}\sqrt{\frac{\Rv^P(\vxo)}{\Rv^P(\vxref(\vxo))}}
}
\underbrace{2 w(\vxo)  \ti k_{\mathrm{n}}^P(\vxo) 	P(\vxo,\omega)}_{\substack{\text{2D}\\\text{driving function}}},
\end{equation}
where $\rho_v^P$ and $\rho_v^G$ are the principal radii of the virtual field and the Green's function along the vertical direction normalized by $k$, with the absolute value operation omitted due to their positive sign for diverging virtual fields.
The Green's function's principal radius is given simply as $\Rv^G(\vxref(\vxo)-\vxo) = |\vxref(\vxo)-\vxo|$ and the virtual field's principal radius $\Rv^P(\vxref(\vxo))$ is expressed by applying \eqref{eq:app:propagated_radii}.

The terms in the driving function can be identified as compensation factors for the \emph{dimensionality mismatch}, emerging in the 2.5D Kirchhoff integral.
Expressing the Kirchhoff approximation \eqref{Eq:SFS_theory:Kirchhoff_appr} for an entirely 2D problem, an arbitrary 2D sound field may be described in the area of investigation by a contour integral.
The enclosing boundary can be interpreted as the continuous distribution of two dimensional secondary point sources described by the 2D Green's function, representing infinite vertical line sources in three dimensions.
The 2D Green's function is weighted by the normal derivative of the sound field, taken on the SSD contour.
%
Application of the 2.5D WFS driving functions aims to describe a 3D sound field in terms of a 2D contour integral with the kernel being the 3D Green's function,
weighted by the normal derivative of the 3D sound field.
This results in a dimensionality mismatch for both the virtual field and the secondary source elements.
The interpretation of the compensation factors in the driving function is then the following:
\begin{itemize}
\item Term $\sqrt{\frac{ 2\pi |\vxref(\vxo)-\vxo| }{ \ti k }}$ is the compensation factor for the \emph{secondary source dimensionality mismatch}.
	Comparison with \eqref{eq:HF_approx:2D_vs_3D_GF} indicates, that the compensation factor approximates the frequency response and attenuation factor of the 2D Green's function in terms of the 3D Green's function.
	Obviously, the attenuation factors can be matched only at a particular distance from a given SSD element $\vxo$, chosen to be at the reference position $\vxref(\vxo)$.
	Alternatively, the frequency response compensation term ensures the flat frequency response of the SSD: a 2D contour of point sources exhibits the frequency response of $\sim \frac{1}{\sqrt{\ti k}}$, which along with the normal derivative term would result in a transfer function of $\sim \sqrt{\ti k}$, that has to be compensated for.
%
\item The virtual source compensation factor resolves the \emph{virtual source dimensionality mismatch}, correcting the virtual source attenuation factor from a 2D to a 3D one.
It is assumed, that the general relationship between a 2D and a 3D sound field, generated by the same planar source distribution at $z = 0$ reads as
\begin{equation}
P_{3\text{D}}(\vx,\omega) = \sqrt{\frac{\ti k}{2\pi}}
\frac{P_{2\text{D}}(\vx,\omega)}{\sqrt{\Rv^P(\vx)}},
\label{Eq:SFS_thory:2D_3D_relation}
\end{equation}
at $x = \posvec{3}{x}{y}{0}$.
This is a straightforward generalization of \eqref{eq:HF_approx:2D_vs_3D_GF} towards general sound fields.
Expressing a 2D sound field at $\vxref(\vxo)$ in terms of the 2D Kirchhoff integral and rewriting in terms of the the corresponding 3D sound fields by applying \eqref{Eq:SFS_thory:2D_3D_relation} leads to the virtual source correction factor under discussion.
A detailed explanation for the virtual source dimensionality compensation is given for the special case of a virtual point source in \cite{Voelk2012}.
\end{itemize}
Referencing the synthesis therefore can be interpreted physically as adjusting both attenuation correction factors for each SSD element to be amplitude correct on the reference curve, performed by prescribing a frequency independent curvature correction factor.

% TODO: TD WFS
%In practical applications the driving functions are implemented in the temporal domain.
%Taking the inverse Fourier transform of \eqref{Eq:SFS_theory:25D_WFS_driv_fun} yields the 2.5D WFS driving functions in the time domain, reading
%\begin{equation}
%d(\vxo, t) = w(\vxo) 
%\sqrt{ 8\pi \dref(\vxo)}
%\sqrt{\frac{\ti \omega}{c} }\hat{k}_{\mathrm{n}}^P(\vxo) 	P(\vxo,\omega),
%\end{equation}

%
\begin{figure}  
\small
  \begin{minipage}[c]{0.64\textwidth}
	\begin{overpic}[width = 1\columnwidth ]{Figures/SFS_theory/25D_WFS_general.png}
	\small
	\put(2,53){(a)}
	\put(2,1){(b)}
	\end{overpic}   \end{minipage}\hfill
	\begin{minipage}[c]{0.35\textwidth}
    \caption{2.5D synthesis of a 3D point source located at $\vxs = \posvec{3}{0.4}{2.5}{0}$, radiating at $\omega_0 = 2\pi \cdot 1.5 \mathrm{krad}/s$.
    Figure (a) depicts the real part of the synthesized field, (b) presents the absolute error of synthesis in a logarithmic scale.
	The reference curve was defined by simple rescaling of the SSD contour.
	The active arc of the SSD is denoted by solid black line, and the inactive part with dotted by black line.
	The reference position on the reference curve for each active SSD element is evaluated numerically.
	Obviously in the present geometry there exist SSD elements, for which no unique reference position can be found.
	In order to ensure smooth driving functions and avoid truncation artifacts for these SSD positions the referencing function is extrapolated.
    }
\label{fig:SFS_theory:25D_WFS_generals}   \end{minipage}
\end{figure}
\vspace{3mm}
The introduced driving functions are capable of the synthesis of arbitrary sound fields applying arbitrary shaped convex SSDs, referencing the synthesis to an arbitrary reference curve. 
The result of such a general 2.5D WFS scenario is presented in Figure \ref{fig:SFS_theory:25D_WFS_generals}.
As the image depicting the synthesis error indicates: on those part of the reference curve for which a stationary point pair can be found on the SSD amplitude correct synthesis is ensured, as the error exhibits a minimum.
%

If a parametrization of the SSD contour and the reference curve along with an analytical virtual source model is known, the referencing function can be expressed analytically, resulting in closed form driving functions specific to the SSD and the referencing contour. 
The following two examples are presented in order to demonstrate the analytical application of the presented driving functions.

% TODO: OUT TO INTRODUCTION (to the entire thesis or for this chapter)
%The presented referencing approach not only allows the derivation of referencing functions for arbitrary SSD-reference curve geometries, but also the analysis of former WFS approaches in the aspect of the positions of amplitude correct synthesis.
%These former approaches include traditional WFS, which optimizes the synthesis of a point source to a reference line as discussed via the following example \cite{Berkhout1993:Acoustic_control_by_WFS,  Start1997:phd, Verheijen1997:phd}, and revisited WFS formulation, which applies a target field independent constant referencing function without taking the virtual source dimensionality mismatch into consideration \cite{Spors2008:WFSrevisited}.
%The further analysis of the latter approach is not included in the present thesis, a thorough discussion of the topic can be found in \cite{Firtha2016}.

\subsection*{Application example: Synthesis of a 3D point source applying a linear SSD}

As a first example assume an infinite linear SSD located at $\vxo = \posvec{3}{x_0}{0}{0}$.
The reference contour is set to be an infinite line parallel to the SSD, located at $\vxref = \posvec{3}{x}{y_{\mathrm{ref}}}{0}$.
This geometry has a distinctive role in the field of sound field synthesis, being the arrangement for which traditional WFS was first formulated \cite{Berkhout1988, Berkhout1993:Acoustic_control_by_WFS,  Start1997:phd, Verheijen1997:phd}.
In this arrangement the driving functions may be derived directly by applying the SPA to the Rayleigh integral describing a sound field perfectly in terms of a planar single layer potential.
Therefore applying of a linear SSD involves the least approximations, avoiding errors due to the application of the Kirchhoff approximation.
Choosing a reference line parallel to the SSD also ensures the existence of a unique reference position for each SSD element, therefore amplitude correct synthesis may be ensured on the entire reference line.
Furthermore, explicit solution can be found directly for this special geometry as described in the following section.

For the case of a virtual point source the referencing function \eqref{Eq:SFS_theory:Referencing function} takes the form
\begin{equation}
\dref(\vxo) = \frac{|\vxo-\vxs| \cdot |\vxref(\vxo)-\vxo|}{|\vxo - \vxs| + |\vxref(\vxo)-\vxo|},
\label{Eq:SFS_theory:PS_ref_dist}
\end{equation}
since the principal radii are simply given as Euclidean distances.
Expressing the referencing function requires the determination of the distance between the reference position on the reference line and the corresponding SSD elements by solving equation
\begin{equation}
\vxref(\vxo) = \vxo + \hat{\vk}^P(\vxo) |\vxref(\vxo)-\vxo|
\label{eq:sfs_theory:ref_pos}
\end{equation}
for $|\vxref(\vxo)-\vxo|$, termed the \emph{reference distance}.
The terminology indicates that it denotes the distance measured from the individual SSD elements at which the synthesis is optimized.

In the present geometry both $\vxo$ and $\vxref$ are lying along infinite parallel lines, with the $y$-coordinates of both curves fixed to constant, thus for the second coordinates of equation \eqref{eq:sfs_theory:ref_pos} 
\begin{equation}
y_{\mathrm{ref}} = y_0 + \hat{k}_y^P(\vxo) |\vxref(\vxo)-\vxo|
\end{equation}
must hold.
With $y_0=0$ the above equation yields the reference distance in the present geometry
\begin{equation}
|\vxref(\vxo)-\vxo| = \frac{y_{\mathrm{ref}}}{\hat{k}_y^P(\vxo)} = -\frac{k}{\phi^{''G}_{zz}(\vxref(\vxo) - \vxo,\omega)},
\end{equation}
hence the general 2.5D WFS driving function, ensuring amplitude correct synthesis of an arbitrary sound field on a reference line reads as
\begin{equation}
D(\vxo, \omega) = 
\sqrt{\frac{8\pi}{\ti k}}\sqrt{\frac{k}{\left|\phi^{P''}_{zz}(\vxo,\omega) - \frac{k_y^P(\vxo)}{y_{\mathrm{ref}}}\right|}}
\ti k_y^P(\vxo) P(\vxo,\omega).
\end{equation}

\begin{figure}
\centering
	\begin{overpic}[width = 1\columnwidth ]{Figures/SFS_theory/25D_WFS_linear_SSD.png}
	\small
	\put(0, 0){(a)}
	\put(47,0){(b)}	
	\end{overpic}   
    \caption{2.5D synthesis of a 3D point source located at $\vxs = \posvec{3}{0}{-2}{0}$, radiating at $\omega_0 = 2\pi \cdot 1 \mathrm{krad}/s$ with the reference line set at $y_{\mathrm{ref}} = 1.5~\mathrm{m}$.
    Figure (a) depicts the real part of the synthesized field, (b) shows the error of synthesis.
    Based on the equivalent scattering interpretation of the synthesis the discrepancy between the synthesized field and the virtual field at $y<0$ can be interpreted as the field of a point source reflected from a planar scatterer surface. 
    Due to the problem symmetry the scattered field is given amplitude correctly along $y = - y_{\mathrm{ref}}$.
    }
\label{fig:SFS_theory:25D_WFS_linear_ssd}  
\end{figure}

Substituting the reference distance into the referencing function \eqref{Eq:SFS_theory:PS_ref_dist} and expressing the local wavenumber vector for a point source ($k_y^P(\vxo) = k \frac{y_0-y_s}{|\vxo-\vxs|}$) results in the virtual source-SSD shape-receiver shape specific driving function
\begin{equation}
D(\vxo, \omega) =  -\frac{1}{4\pi}
\sqrt{ \frac{8\pi}{\ti k} }
\sqrt{ \frac{y_{\mathrm{ref}}}{y_{\mathrm{ref}} -y_s } }
\ti k y_s \frac{\te^{-\ti k |\vxo-\vxs|}}{|\vxo-\vxs|^{\frac{3}{2}}}.
\label{eq:SFS_theory:WFS_point_source}
\end{equation}
This result is precisely equivalent with the traditional WFS driving function \cite[(2.27)]{Verheijen1997:phd}, \cite[(3.16)\&(3.17)]{Start1997:phd} of a point source, and furthermore identical to the farfield/high frequency approximated explicit solution presented in the next section \cite[(25)]{Spors10ahrens:analysis}, \cite[Ch. 2.3]{Schultz2016}. 
The result of synthesis is depicted in Figure \ref{fig:SFS_theory:25D_WFS_linear_ssd} confirming, that by applying the derived driving functions amplitude correct synthesis is ensured along the reference line.

Taking the temporal inverse Fourier transform of the driving functions weighted by $S(\omega)$ yields the temporal driving functions for a virtual point source with the source time history $s(t)$.
\begin{equation}
d(\vxo, t) =  -\frac{1}{2\pi} \int_{-\infty}^{\infty}
\sqrt{\frac{\ti k}{2\pi}}
\sqrt{\frac{y_{\mathrm{ref}}}{y_{\mathrm{ref}} -y_s } } y_s \frac{\te^{\ti \omega ( t - \frac{|\vxo-\vxs|}{c} )}}{|\vxo-\vxs|^{\frac{3}{2}}} \td \omega.
\end{equation}
Exploiting the shift theorem and the associativity of convolution yields the temporal driving function
\begin{equation}
d(\vxo, t) =  -
\sqrt{\frac{1}{2\pi c}}
\sqrt{\frac{y_{\mathrm{ref}}}{y_{\mathrm{ref}} -y_s } } y_s \frac{ s^{\nshortmid}_t( t - \frac{|\vxo-\vxs|}{c} )}{|\vxo-\vxs|^{\frac{3}{2}}},
\end{equation}
where $  s^{\nshortmid}_t( t ) = h(t) \ast_t s( t  )$, with $\ast_t$ denoting convolution in the time dimension.
The source time history is pre-equalized with a filter, exhibiting the frequency response of $H(\omega) = \sqrt{\ti \omega}$ being a half-differentiator. 
The filter compensates the frequency response of a 2D SSD contour as the part of the secondary source compensation factor, as discussed above.
The impulse response of the SSD compensation filter can be expressed by differentiating the half-integrator's impulse response as given in \cite{Deregowski1983}
\begin{equation}
h(t) = \frac{\delta(t)}{\sqrt{\pi t}} - \frac{1}{2} \frac{\theta(t)}{t^{3/2}},
\end{equation}
where $\theta(t)$ is the Heaviside step function.
Practical implementation of this prefilter---being present in the 2.5D time domain driving function for an arbitrary sound field---is discussed in details in \cite{Schultz2013:IIR_prefilters} applying IIR filters, while in \cite[Sec. 2.5]{Schultz2016} the ideal FIR filter coefficients are given analytically.

\subsection*{Application example: Synthesis of a plane wave applying a circular SSD}

As a second example the synthesis of a plane wave applying a circular SSD with the radius of $R_{\mathrm{SSD}}$ is presented.
The synthesis is referenced to a concentric circle inside the SSD with the radius of $R_{\mathrm{ref}}$.
For this geometry the explicit driving functions are also available \cite{Ahrens2008:Analytical_Circ_Spherical_SFS, Ahrens2009:circularSSD_mismatch, Ahrens2009:circular25D_SFR}, which are however not discussed in details in the present thesis.

\begin{figure}
\centering
	\begin{overpic}[width = 1\columnwidth ]{Figures/SFS_theory/25D_WFS_circular_SSD.png}
	\put(0,1){(a)}
	\put(48,1){(b)}
	\end{overpic}   
    \caption{2.5D synthesis of a 2D plane wave with the angular frequency $\omega_0 = 2\pi \cdot 1 \mathrm{krad}/s$ propagating into the direction $\vk^{\mathrm{PW}} = \posvec{3}{k_x^{\mathrm{PW}} }{0}{0}$.
    The SSD is a circular one, with the radius $R_{\mathrm{SSD}} = 2~\mathrm{m}$.
    The reference curve is a circle with the radius $R_{\mathrm{ref}} = 1.5~\mathrm{m}$.
    Figure (a) depicts the real part of the synthesized field, (b) shows the error of synthesis.
    }
\label{fig:SFS_theory:25D_WFS_circular_ssd}  
\end{figure}

Again, the system of equations describing the reference distance for each SSD element is given by
\begin{align}
\vxref(\vxo) &= \vxo + \hat{\vk}^P(\vxo) |\vxref(\vxo)-\vxo|
\\
|\vxref(\vxo)| &= R_{\mathrm{ref}}.
\end{align}
Expressing the reference distance leads to a second order equation.
By exploiting that $|\vxo| = R_{\mathrm{SSD}}$, $|\hat{\vk}^P(\vxo)| = 1$ and taking only the smaller root into consideration---corresponding to the closer arc of the reference circle to the actual SSD position---yields the reference distance
\begin{equation}
|\vxref(\vxo)-\vxo| = R_{\mathrm{SSD}} \left( \hat{k}^P_r(\vxo) + \sqrt{ \hat{k}^P_r(\vxo)^2 + \left( \frac{R_{\mathrm{ref}}}{R_{\mathrm{SSD}}} \right)^2 - 1 } \right),
\label{eq:SFS_theory:pw_circ_ref}
\end{equation}
with $\hat{k}^P_r(\vxo)$ denoting the radial component of the normalized wavenumber vector.
Applying the reference distance to the general 2.5D WFS driving functions \eqref{Eq:SFS_theory:25D_WFS_driv_fun} allows the synthesis of an arbitrary sound field referenced on a reference circle inside the SSD.

Assume the special case of a virtual 2D plane wave, propagating parallel to the synthesis plane described by the wavenumber vector $\vk^{\mathrm{PW}} = \posvec{3}{k_x^{\mathrm{PW}}}{k_y^{\mathrm{PW}}}{0}$.
For a 2D sound field invariant along the vertical dimension the vertical curvature is zero ($\phi^{''P}_{zz}(\vxo) = 0$) and the referencing function is the reference distance itself.
The driving function synthesizing a 2D plane wave is then given as
\begin{equation}
\label{eq:SFS_theory:WFS_plane_wave}
D(\vxo, \omega) = -w(\vxo) 
\sqrt{\frac{8\pi}{\ti k}}\sqrt{|\vxref(\vxo)-\vxo|}
\ti k_r^{\mathrm{PW}}(\vxo) 	\te^{-\ti \left< \vk^{\mathrm{PW}} \cdot \vxo \right> },
\end{equation}
with the reference distance given by \eqref{eq:SFS_theory:pw_circ_ref} in order to optimize the synthesis onto the reference circle.
For the application of these driving functions a simple example is depicted in Figure \ref{fig:SFS_theory:25D_WFS_circular_ssd} for the synthesis of a harmonic plane wave.

In order to find the driving signal for the synthesis of a plane wave carrying a broadband excitation time history $s(t)$, \eqref{eq:SFS_theory:WFS_plane_wave} is inverse Fourier transformed weighted by the excitation spectrum
\begin{equation}
d(\vxo, t) = -w(\vxo) 
\sqrt{\frac{8 \pi }{c	}|\vxref(\vxo)-\vxo|}  \,
\hat{k}_r^{\mathrm{PW}}(\vxo)  h(t) \ast_t s( t - \frac{1}{c}\left< \hat{\vk}^{\mathrm{PW}} \cdot \vxo \right>),
\end{equation}
where $h(t) = \mathcal{F}^{-1}_t\left\{ \sqrt{\ti \omega} \right\}$ is the SSD compensation filter, performing a half-derivation on the time history and $\hat{\vk}^{\mathrm{PW}} = \posvec{3}{\hat{k^{\mathrm{PW}}_x}}{k^{\mathrm{PW}}_y}{0}$ being a unit vector pointing into the plane wave propagation direction.

\section{Explicit solution: Spectral Division Method}

The explicit solution for the general sound field synthesis problem aims at the direct solution of the inverse problems, described by integral equations \eqref{Eq:Theory:3D_SFS} and \eqref{Eq:Theory:25D_SFS}.

Generally speaking the explicit methodology utilizes compact operator theory by exploiting that integral \eqref{Eq:Theory:3D_SFS} constitutes a compact Fredholm operator with the kernel being the Green's function \cite{Ahrens2012,MorseFeshbach1953}.
Such an operator and the involved acoustic fields can by expanded into the series of orthogonal eigenfunctions of the wave equation on a control surface, that form a complete basis of the solution.
The inverse problem can be straightforwardly solved for the driving function expansion coefficients by a comparison of the corresponding eigenvalues, as long as none of the expansion coefficients of the operator kernel is zero.
Otherwise the problem is termed \emph{ill-conditioned}.
Finally the explicit analytical solution is found for the driving function as an infinite sum of the weighted basis functions.
The method is often referred to as \emph{mode-matching} solutions, since the eigenfunctions of a given geometry are termed the \emph{modes}.
This solution is unique for general enclosures and also for the---strictly speaking---non-enclosing planar case as shown in \cite{Zotter2013:uniqueness} and \cite{Fazi2010} respectively \footnote{In contrary sound field control utilizing the Kirchhoff-Helmholtz formulation would be non-unique on the eigenfrequencies of the enclosure due to resonance phenomena.}.

The determination of the appropriate eigenfunctions for a general geometry is a tough challenge.
For spherical and circular geometries spherical and circular harmonics form the demanded basis functions. For a rigorous treatment for mode-matching SFS using spherical and circular SSDs see \cite{Ahrens2010phd,Zotter2009phd,Ahrens2012,Ahrens2009:circularSSD_mismatch,Ahrens2009:circular25D_SFR,Ahrens2008:Analytical_Circ_Spherical_SFS,Schultz2014:Comparing_approaches} and \cite{Koyama2014} for the cylindrical solution.
In the present thesis only the planar and linear geometries are investigated in details.
However, under the validity of the tangent plane approximation---i.e. under high-frequency assumptions with small SSD curvatures---the SSD surface/contour can be considered locally planar/linear and the spatial form of the explicit driving functions are valid for arbitrary shaped SSDs.

%\begin{figure}
%	\centering
%	\begin{overpic}[width = .85\columnwidth]{Figures/SFS_theory/planar_linear_geometry.png}
%	\footnotesize
%	\put(0, 0){(a)}
%	\put(45,0){(b)}
%	\end{overpic}
%\caption{Secondary source distribution geometry .}
%	\label{Fig:Theory:planar_linear_geometry}
%\end{figure}

\subsection{3D Spectral Division Method}

Assume an infinite planar SSD along---for the sake of simplicity---the $y = 0$ plane, degenerated from the geometry introduced for the Rayleigh integrals in the previous chapter, shown in Figure \ref{Fig:Theory:Rayleigh_geometry}.
The half-space of synthesis is chosen to be at $y>0$, therefore all the virtual sources are assumed to be located at $y<0$.
The synthesized field in this geometry is given by a Fredholm-integral of the first kind 
\begin{equation}
P(\vx,\omega) = \iint_{-\infty}^{\infty} D(x_0,z_0,\omega) G(x-x_0,y,z-z_0, \omega) \td x_0 \td z_0 = D(x,z,\omega)\ast_{x,z} G(x,y,z,\omega),
\end{equation}
describing a continuous convolution along the SSD plane.
Here $G(x,y,z,\omega)$ denotes the sound field of a secondary source element placed at the origin and $\ast_{x,z}$ denotes convolution along the $x$ and $z$ dimensions.

For this geometry the orthogonal basis is given by the continuous set of exponentials, and the decomposition of the involved quantities is given by a double Fourier transform \cite{Ahrens2012, Arfken2005,Schultz2014:Comparing_approaches}, with the physical interpretation of a plane wave decomposition as discussed in section \ref{Sec:thoery:angular_Spectrum}.
Applying the convolution theorem to the angular spectrum representation the convolution may be transformed into a multiplication \cite{Girod2001}:
\begin{equation}
\tilde{P}(k_x,y,k_z, \omega) = \tilde{D}(k_x,k_z, \omega) \cdot \tilde{G}(k_x,y,k_z, \omega).
\end{equation}
%
The expansion coefficient are therefore obtained by a comparison of spectral coefficients and the driving function takes the form:
\begin{equation}
\tilde{D}(k_x,k_z,\omega) = \frac{\tilde{P}(k_x,y,k_z, \omega)}{ \tilde{G}(k_x,y,k_z, \omega)} = 
\frac{\mathcal{F}\left\{ P(\vx,\omega) \right\} }
{  \mathcal{F}\left\{ G(\vx,\omega) \right\} },
\label{Eq:Theory:Dkxkz}
\end{equation}
\begin{equation}
D(x_0,z_0,\omega) = \frac{1}{4\pi^2} \iint_{-\infty}^{\infty} \tilde{D}(k_x,k_z, \omega) \te^{-\ti (k_x x_0 + k_z z_0)} \td k_x \td k_z.
\label{Eq:Theory:Dkx_inverse_Fourier}
\end{equation}
Since the driving function spectrum is yielded by a division in the spectral domain the approach is termed the \emph{Spectral Division Method} \cite{Ahrens2010a, Ahrens2012:Ambisonics_for_planar_linear, Ahrens2011:icassp, Ahrens2010:Ambisonics_w_planar_linear}.

Substituting the $k_x-k_z$ representation of the 3D Green's function given by \eqref{eq:HF_approx:Greens_2D_Spectrum} the driving function \eqref{Eq:Theory:Dkx_inverse_Fourier} reads as
\begin{equation}
D(x_0,z_0,\omega) = \frac{1}{4\pi^2} \iint_{-\infty}^{\infty} 2\ti k_y \frac{\tilde{P}(k_x,y,k_z, \omega)}{ \te^{ -\ti k_y  | y |  } } \te^{-\ti (k_x x_0 + k_z z_0)} \td k_x \td k_z.
\end{equation}
with $k_y$ defined by \eqref{eq:theory:k_y_definition}.	
Expressing the target field spectrum by extrapolating from the plane $y = 0$ according to \eqref{Eq:Theory:Wave_field_extrapolation}---i.e. as 
$
\tilde{P}(k_x,y,k_z, \omega) = \tilde{P}(k_x,0,k_z, \omega)  \te^{ -\ti k_y  y }
$---
the exponential pressure propagators cancel out and the driving function becomes independent from the $y$-coordinate. 
The driving function in the wavenumber domain therefore reads as
\begin{equation}
\tilde{D}(k_x,k_z,\omega) = 2\ti k_y \tilde{P}(k_x,0,k_z,\omega) = -2 \left. \frac{\partial}{\partial y} \tilde{P}(k_x,y,k_z,\omega) \right|_{y = 0},
\label{Eq:Theory:Planar_explicit_driv_fun}
\end{equation}
for which it was exploited that according to \eqref{eq:Theory:Fourier_diff} multiplication by $k_y$ represents differentiation along the $y$-dimension.

Straightforwardly, the explicit expression of the driving function in the spatial domain is obtained by the corresponding inverse Fourier transform according to \eqref{Eq:Theory:Dkx_inverse_Fourier}:
\begin{equation}
D(x_0,z_0,\omega) = -2 \left. \frac{\partial}{\partial y} P(\vx,\omega) \right|_{y = 0}.
\label{Eq:Theory:Planar_explicit_driv_fun_spatial}
\end{equation}

The planar explicit driving functions are therefore completely equivalent to the implicit solution, in a planar geometry provided by the Rayleigh integral.
The coincidence of the explicit and implicit driving functions is a consequence of the uniqueness of the problem in the present geometry.
It is also indirectly proven, that the wavefield extrapolation equations are the spectral domain representations of the Rayleigh integrals.

However, an important difference between the implicit and explicit solution exist: until \eqref{Eq:Theory:Dkx_inverse_Fourier} the present method does not pose any constraints on the actual form of the Green's function. 
Theoretically an arbitrary transfer function may be assigned for the SSD elements. 
As long the problem is well-conditioned---i.e. the spectrum of the transfer function does not exhibit zeros---unique driving functions may be derived applying the explicit methodology.

If the SSD elements are 3D point sources the following physical interpretation can be assigned to the explicit solution for: 
as it was stated in section \ref{Sec:HF_approx:1D_Greens} a planar distribution of point sources with a harmonic spatial distribution described by $k_x, k_z$ radiate plane waves with the same wavenumber components and a wavenumber/direction dependent amplitude factor $\frac{1}{2 \ti k_y}$ (c.f. \eqref{eq:HF_approx:Greens_2D_Spectrum}, degenerating at $k_x = k_z = 0$ to the 1D Green's function).
The driving function \eqref{Eq:Theory:Planar_explicit_driv_fun} thus compensates the planar SSD's response for the synthesis of a single plane wave component.
Finally, the explicit driving function for an arbitrary virtual field is found as the sum of the individual plane wave driving functions weighted by the virtual field wavenumber content.


\subsection*{Application example: Synthesis of a 3D point source using a planar SSD}
\fscom{well done, I've tried it for the Acta 2014 article, but could not go any further, have to check your solution in detail. However, it might be worth that this issue from the acta in the appendix is solved here!}
\begin{figure}
	\centering
	\begin{overpic}[width = 1\columnwidth]{Figures/SFS_theory/Planar_SDM.png}
	\small
	\put(0, 0){(a)}
	\put(47,0){(b)}
	\end{overpic}
\caption{
Synthesis of a virtual point source using a planar SSD applying the SDM driving functions.
The SSD is located at $\vxo = [x_0,\ 0,\ z_0]^{\mathrm{T}}$ denoted by solid black line. 
The virtual source is located at $\vxs = [0,\ -2,\ 0]^{\mathrm{T}}$ oscillating at $\omega_0 = 2\pi \cdot 1 ~\mathrm{krad/sec}$.
The figures depict the real part of the synthesized field (a) and the deviation from the target sound field (b) measured at $z=0$.}
	\label{Fig:Theory:monopole_synthesis_by_planar_SDM}
\end{figure}

The application of the planar explicit solution is presented via the synthesis of a 3D virtual point source at $\vxs = \posvec{3}{x_s}{y_s}{z_s}$ behind the SSD plane, located at $\vxo = \posvec{3}{x_0}{0}{z_0}$.
The wavenumber domain representation of the driving function is obtained by substituting the angular spectrum of a 3D point source into \eqref{Eq:Theory:Dkxkz} with applying the Fourier transform shift theorem
\begin{equation}
\tilde{D}(k_x,k_z,\omega) =  \frac{-\frac{\ti}{2} \frac{ \te^{-\ti k_y | y - y_s|} }{ k_y} \te^{\ti (k_x x_s +k_z z_s)} }{-\frac{\ti}{2} \te^{-\ti k_y | y |} / k_y   } = \te^{-\ti k_y |y_s|}\te^{\ti (k_x x_s +k_z z_s)}.
\label{Eq:Theory:Monopole_SDM_planar_driv_fun}
\end{equation}
The double inverse Fourier transform can be carried out analytically by taking the $y$-derivative of the Weyl's integral representation of the Green's function (See \cite{Lalor1969} or \cite[(2.65)]{Williams1999}):
\begin{equation}
\frac{\partial}{\partial y} G(\vxo - \vxs,\omega ) = 
\frac{1}{4\pi^2} \iint_{-\infty}^{\infty} -\frac{1}{2} \te^{ -\ti k_y  | y - y_s |  }
\te^{\ti (k_x x_s + k_z z_s)} \te^{-\ti (k_x x_0 + k_z z_0)} \td k_x \td k_z,
\label{Eq:Theory:Weyls_derivative}
\end{equation}
Comparing \eqref{Eq:Theory:Monopole_SDM_planar_driv_fun} and \eqref{Eq:Theory:Weyls_derivative} it is revealed that the driving function in the spatial domain is given by
\begin{equation}
D(x_0,z_0,\omega) = -2 \frac{\partial}{\partial y} \left. G(\vxo - \vxs,\omega )\right|_{y = 0} = -\frac{y_s}{2\pi} \left( \frac{1}{|\vxo-\vxs|} + \ti k\right) \frac{\te^{-\ti k |\vxo-\vxs|}}{|\vxo-\vxs|^2},
\end{equation}
which is in agreement with equation \eqref{Eq:Theory:Planar_explicit_driv_fun_spatial}.

The result of synthesizing the steady state field of a point source is illustrated in Figure \ref{Fig:Theory:monopole_synthesis_by_planar_SDM}. 
In the target half space $y>0$ perfect synthesis is achieved, as it is indicated in Figure \ref{Fig:Theory:monopole_synthesis_by_planar_SDM} (b), depicting the discrepancy between the synthesized and the target sound field. 
Obviously, the figure also presents the result of 3D WFS of a spherical wave without applying the high frequency gradient approximation.


\subsection{2.5D Spectral Division Method}
\label{Sec:25D_SDM}

As the geometry for the derivation of 2.5D explicit driving functions assume that the SSD is an infinite linear distribution of point sources, located at $\vxo = \posvec{3}{x_0}{0}{0}$.
The synthesized field in this arrangement reads as
\begin{equation}
P(x,y,z,\omega) = \int_{-\infty}^{\infty} D(x_0,\omega) G(x-x_0,y,z,\omega) \td x_0.
\end{equation}
Similarly to the planar case the basis functions for a linear SSD are given by exponentials:
by realizing that the above equation is a convolution along the $x$-axis, the convolution can be transformed into a multiplication by means of a spatial forward Fourier transform
\begin{equation}
\tilde{P}(k_x,y,z, \omega) = \tilde{D}(k_x,\omega) \cdot \tilde{G}(k_x,y,z, \omega).
\end{equation}
The driving function spectra is then obtained as a spectral ratio
\begin{equation}
\tilde{D}(k_x,\omega) = \frac{\tilde{P}(k_x,y,z, \omega)}{\tilde{G}(k_x,y,z, \omega)} = \frac{\mathcal{F}_x\left\{ P(\vx,\omega) \right\}}{\mathcal{F}_x\left\{ G(\vx,\omega) \right\}},
\label{Eq:SFS_Theory:LinearSDM_spectral}
\end{equation}
and the frequency domain driving function is obtained as the spatial inverse Fourier transform with respect to $k_x$
\begin{equation}
D(x_0,\omega) = \frac{1}{2\pi} \int_{-\infty}^{\infty} \frac{\tilde{P}(k_x,y,z, \omega) }{\tilde{G}(k_x,y,z, \omega)} \te^{-\ti k_x x_0} \td k_x.
\label{Eq:Theory:LinearSDM1}
\end{equation}
Again, theoretically the transfer function may describe the field of an arbitrary sound source, as long as it does not exhibit zeros in order to keep the problem well-conditioned.

\vspace{3mm}
Unlike the planar case the present driving function contains both $y$ and $z$ positions, thus the driving function depends on the listener position: Equation \eqref{Eq:Theory:LinearSDM1} may be solved only for positions on the surface of a cylinder with fixed radius $d = \sqrt{y^2 + z^2}$ \cite[p.~60.]{Ahrens2010phd}.
This is a direct consequence of the fact, that the pressure of an arbitrary 3D sound field measured on the SSD does not determine completely the pressure on the reference line---and vice versa---.
Furthermore an infinite line source---i.e. the SSD---can only radiate wavefronts with cylindrical symmetry as it was discussed in details in \ref{sec:greens_function_spectrum}.
Phase correct synthesis therefore may be assured only in a plane containing the SSD in which the radial wavenumber of the synthesized field and the target field coincide. 
Amplitude correct synthesis is ensured in this plane at a distance $\dref = \sqrt{y^2 + z^2}$, for which the driving functions are calculated.

For practical applications the plane of synthesis if chosen to be the horizontal plane $z=0$ and reference the driving functions to a \emph{reference line} by setting $y = \yref$, requiring that for the virtual field $k_z(x,y,0) = 0$ holds.
The driving function thus reads as
\begin{equation}
D(x_0,\omega) = \frac{1}{2\pi} \int_{-\infty}^{\infty} \frac{\tilde{P}(k_x,\yref,0, \omega) }{\tilde{G}(k_x,\yref,0, \omega)} \te^{-\ti k_x x_0} \td k_x.
\label{Eq:Theory:Linear_SDM}
\end{equation}
In this geometry amplitude correct synthesis is restricted to the reference line.

Similarly to the 3D case the following physical interpretation can be assigned to the 2.5D explicit solution:
given an infinite line source with a harmonic spatial distribution described by $k_x$. 
The radiated sound field is given by
\begin{multline}
\int_{-\infty}^{\infty} G(x - x_0,y,z) \te^{\ti k_x x_0} \td x_0 = \hat{G}(k_x,y,z,\omega) \te^{-\ti k_x x} = \\
=  -\frac{\ti}{4} H_0^{(2)}\left( \sqrt{\left( \frac{\omega}{c} \right)^2-k_x^2} \sqrt{y^2+z^2} \right)  \te^{-\ti k_x x},
\end{multline}
at $k_x=0$ resulting in the 2D Green's function as discussed in section \ref{sec:greens_function_spectrum}.
Such a source radiates cylindrical symmetric sound fields with conical wavefronts as depicted in Figure \eqref{Fig:Theory:greens_stat_pos} (a). 
Along a fixed reference line at $z=0$ the SSD reproduces a harmonic spatial distribution $\te^{-\ti k_x x}$ attenuated approximately by $\frac{1}{\sqrt{k_y}|y_{\mathrm{ref}}|}$, corresponding attenuating plane waves with $k_z=0$.
Therefore the wavenumber domain driving function ensures the compensation of the linear SSD response for the synthesis of a single plane wave component propagating in the plane of synthesis.
Obviously, for sound fields that can be expanded into the series of plane waves with $k_z=0$ the driving function is the weighted sum of the plane wave driving function, resulting in \ref{Eq:Theory:Linear_SDM}.

It is worth noting that the analytic Fourier transform coefficients of the target sound field are available only for limited simple virtual source models. 
Even in these cases the inverse transform of the driving functions can rarely be evaluated analytically, therefore numerical transforms are needed.
For a practical and optimized implementation of the SDM for an arbitrary target sound field refer to \cite{ahrens2013a:efficientSDM}.

\subsection*{Application example: Synthesis of a 3D point source using a linear SSD}

As an example for 2.5D explicit synthesis the reproduction of a 3D point source is presented.
The virtual source is located at $\vxs = \posvec{3}{x_s}{y_s}{0}$, with $y_s<0$. 
The SSD is an linear linear set of 3D point sources, located along $\vxo = \posvec{3}{x_0}{0}{0}$.
The explicit driving function for a linear SSD is given by \eqref{Eq:Theory:Linear_SDM}. 
Substituting the 1D spectra of the virtual and the secondary point sources along with applying the Fourier shift theorem the driving function is given as
\begin{equation}
\hat{D}(k_x,\omega) = 
\frac{ -\frac{\ti}{4} H_0^{(2)} \left( \sqrt{ \left(\frac{\omega}{c}\right)^2 - k_x^2} |\yref - y_s| \right)  \te^{\ti k_x x_s} }
     { -\frac{\ti}{4} H_0^{(2)} \left( \sqrt{ \left(\frac{\omega}{c}\right)^2 - k_x^2} |\yref| \right)  }
,
\end{equation}
and in the spatial domain the inverse Fourier transform yields
\begin{equation}
\label{Eq:Theory:SDM_point_source}
D(x_0,\omega) = \frac{1}{2\pi} \int_{-\infty}^{\infty} 
\frac{  H_0^{(2)} \left( \sqrt{ \left(\frac{\omega}{c}\right)^2 - k_x^2} |\yref - y_s| \right)  }
     {  H_0^{(2)} \left( \sqrt{ \left(\frac{\omega}{c}\right)^2 - k_x^2} |\yref|       \right)  }
\te^{- \ti k_x (x_0 - x_s)}
\td k_x.
\end{equation}
The synthesized field using this driving function is depicted in \ref{Fig:Theory:monopole_synthesis_by_linear_SDM} (a). 
As it can be seen from Figure (b) displaying the discrepancy between the synthesized field and the target field application of the explicit driving function ensures perfect synthesis on the reference line. 
In other parts of the space amplitude errors are present.

\begin{figure}
	\centering
	\begin{overpic}[width = 1\columnwidth]{Figures/SFS_theory/Linear_SDM.png}
	\footnotesize
	\put(0, 0){(a)}
	\put(45,0){(b)}
	\end{overpic}
\caption{Synthesis of a virtual point source employing a linear SSD applying the 2.5D SDM driving functions.
The SSD is located at $\vxo = [x_0,\ 0,\ 0]^{\mathrm{T}}$, denoted by a solid black line. 
The virtual source is located at $\vxs = [0,\ -2,\ 0]^{\mathrm{T}}$ oscillating at $\omega_0 = 2\pi \cdot 1 ~\mathrm{krad/sec}$. 
The reference line was set to $\yref = 1.5~\mathrm{m}$.
The figure depicts the synthesized field at the synthesis plane ($z = 0$) with (a) depicting the real part of the synthesized field, (b) depicting the error of synthesis.}
	\label{Fig:Theory:monopole_synthesis_by_linear_SDM}
\end{figure}
%
As discussed in \cite{Spors10ahrens:analysis} the derived driving function spectrum can be simplified by applying the large-argument/asymptotic approximation of the Hankel function, given by \eqref{Eq:HF_approx:Hankel_asymptotic_form}.
The asymptotic form gives a fair approximation for \eqref{Eq:Theory:SDM_point_source} if $k_y |\yref| \gg 1$ holds, valid in the far field of the SSD in front of the virtual source, where $k_y \gg k_x$ dominates the inverse transform.
Applying the Hankel's function approximation the inverse transform can be carried out analytically, resulting in
\begin{equation}
D(x_0,\omega) \approx \frac{1}{2} \sqrt{\frac{\yref}{\yref-y_s}} \ti \frac{\omega}{c} \frac{y_s}{|\vxo-\vxs|} H_1^{(2)}\left( \frac{\omega}{c} |\vxo-\vxs| \right)
\end{equation}
as given by \cite[(24)]{Spors10ahrens:analysis}.
A further large-argument approximation of the Hankel function returns the 2.5D WFS driving function for a 3D point source referencing the synthesis on a reference line, given by \eqref{eq:SFS_theory:WFS_point_source}, indicating the the implicit solution constitutes a high frequency approximation for the explicit solution in case of a virtual point source.
The equivalence of the SDM and 2.5D WFS referencing the synthesis of a virtual plane wave on a reference line was further discussed in \cite{Firtha2016, Schultz2016:DAGA,Schultz2016}.
The general relation of the explicit solution and 2.5D WFS is investigated in the following section.

\subsection{Explicit solution in the spatial domain}

The determination of a single spectral coefficient for the explicit solution requires the knowledge of the entire target field over the boundary surface in order to perform the spectral decomposition. 
The explicit solution is therefore often termed as a \emph{global solution}.
In contrary, the implicit solution requires the value of the local field variables only at the actual SSD position at which the driving functions is to be expressed.
Implicit solution is thus referred to as a \emph{local solution}.
In the followings it is presented how the global solution can be transformed into a local one by the application of the stationary phase method.

As it was discussed in \ref{Sec:SPA_for_Fourier} the stationary phase approximation allows the evaluation of forward and inverse Fourier integrals around stationary positions in the spatial and spectral domain.
The SPA therefore may employed in order to give an approximate formulation for the 2.5D explicit driving functions purely in the spatial domain.
The derivation consists of two main steps
\begin{itemize}
%
\item First the spectral driving functions are expressed in an asymptotic form. 
The calculus can be done by assuming, that the involved spectra are obtained via the SPA of the corresponding forward Fourier transforms. 
This step links the spectral coefficients to stationary positions in the spatial domain.
%
\item It is followed by the inverse Fourier transform of the asymptotic spectral driving functions.
The evaluation of the inverse transform with the SPA relates the forward transform stationary positions to positions along the SSD.
\end{itemize}

The derivation starts from the 2.5D explicit driving functions in the wavenumber domain given by \eqref{Eq:SFS_Theory:LinearSDM_spectral}
\begin{equation}
\tilde{D}(k_x,\omega) = \frac{\tilde{P}(k_x,y,0, \omega)}{\tilde{G}(k_x,y,0, \omega)},
\end{equation}
ensuring perfect synthesis along a line with fixed $y$-coordinate.
In the followings for the sake of brevity and transparency $\omega$ and $z$ dependency is suppressed, the latter since the driving functions are defined at $z=0$.
By definition the wavenumber content of the involved quantities are obtained via a forward Fourier transform, with the involved sound fields expressed by their polar form
\begin{align} 
\tilde{P}(k_x,y) = \int\limits_{-\infty}^{\infty} A^P(x,y) \, \te^{\ti \phi^P(x,y)} \, \te^{\ti k_x x} \td x, \\
\tilde{G}(k_x,y) = \int\limits_{-\infty}^{\infty} A^G(x,y) \, \te^{\ti \phi^G(x,y)} \, \te^{\ti k_x x} \td x.
\end{align}
The involved spectra are can be approximated by using the SPA:
under high frequency assumptions the Fourier integrals may be approximated by evaluation around their stationary point $x^*_P(k_x)$ and $x^*_G(k_x)$ where their phase derivative vanishes. 
The stationary positions are defined as
\begin{align}	
\label{eq:xP_xG_in_spatial_domain}
\left. \frac{\partial}{\partial x} \left(  \phi^P(x,y) + k_x x \right )\right|_{x = x^*_P(k_x)} = 0
\hspace{3mm} &\rightarrow \hspace{3mm}
k_x^P(x^*_P(k_x),y) = k_x \nonumber
\\ 
\left. \frac{\partial}{\partial x} \left( \phi^G(x,y) + k_x x \right )\right|_{x = x^*_G(k_x)} = 0
\hspace{3mm} &\rightarrow \hspace{3mm}
k_x^G(x^*_G(k_x),y) = k_x.
\end{align}
Since it is assumed, that $k_z^P(x,y) = k_z^G(x,y) = k_z = 0$ holds in the plane of investigation, therefore the stationary positions are found along a given $y$, where the local propagation direction of the virtual field and the Green's function matches to that of the spectral plane wave, defined by $k_x$.
The properties of the involved wave fields at these positions will dominate the corresponding Fourier integrals.
Hence, the forward transform defines two particular positions in the space, linked together via the actual spectral wavenumber.

Having defined the stationary positions, the forward Fourier transforms can be evaluated by the SPA. 
With accounting for the negative second phase-derivatives---since both the virtual sound field and the Green's function are diverging--- their spectra can be approximated as \cite[Ch. 5]{Tracy2014}
\begin{align}
\tilde{P}(k_x,y) \approx& \sqrt{\frac{2\pi}{\ti \, |\phiPxx(x^*_P(k_x),y)|}} A^P(x^*_P(k_x),y) \, \te^{\ti \phi^P(x^*_P(k_x),y)} \, \te^{\ti k_x \cdot x^*_P(k_x)},\\
\tilde{G}(k_x,y) \approx& \sqrt{\frac{2\pi}{\ti \, |\phiGxx(x^*_G(k_x),y)|}} A^G(x^*_G(k_x),y) \, \te^{\ti \phi^G(x^*_G(k_x),y)} \, \te^{\ti k_x \cdot x^*_G(k_x)},
\end{align}
and the asymptotic approximation of the explicit driving function on a given spectral component yields
\begin{equation}
\label{eq:hfapproxspectra}
\tilde{D}(k_x,y)
\approx  
\sqrt{\frac{\phi_{G,xx}''(x^*_G(k_x),y)}{\phi_{P,xx}''(x^*_P(k_x),y)}}
\, \frac{P(x^*_P(k_x),y)}{G(x^*_G(k_x),y)}
\, \te^{\ti k_x \cdot \left( x^*_P(k_x) - x^*_G(k_x)\right)}.
\end{equation}
Hence, the spectrum can be be expressed by evaluating the target pressure and the Green's function at evaluation points, where the local propagation direction of the involved fields coincides with that of the actual spectral plane wave.

In order to express the driving functions in the spatial domain the inverse Fourier transform of the asymptotic spectrum \eqref{eq:hfapproxspectra}, reading
%
\begin{equation}
\label{Eq:inverse_transform_def}
D(x_0,y)
=  \frac{1}{2\pi} \int\limits_{-\infty}^{\infty}
\sqrt{\frac{\phi_{G,xx}''(x^*_G(k_x),y)}{\phi_{P,xx}''(x^*_P(k_x),y)}} 
\frac{P(x^*_P(k_x),y)}{G(x^*_G(k_x),y)}
\,\te^{\ti k_x \cdot\left( x^*_P(k_x) - x^*_G(k_x)\right)}
\,\te^{-\ti k_x x_0} \td k_x
\end{equation}
is approximated by the stationary phase method, with the phase function under investigation given by
\begin{equation}
\label{Eq:inverse_transform_phase_function}
{\Phi}(k_x) = \phi^P[x^*_P(k_x),y] -  \phi^G[x^*_G(k_x),y] +  k_x \, x^*_P(k_x) - k_x\, x^*_G(k_x) -  k_x x_0.
\end{equation}
%
As it was discussed in \ref{Sec:SPA_for_Fourier}, generally speaking in the spatial inverse Fourier transform of an arbitrary wave field spectrum $\tilde{P}$ each wavenumber component $k_x$ will dominate one spatial position $x_0$, where the actual wavenumber component $k_x(x_0)$ coincides with the local wavenumber of the sound field $k_x^P(x_0)$.
For the present case this wavenumber is found as the stationary phase wavenumber $k_x^*(x_0)$ of the integral \eqref{Eq:inverse_transform_def} \cite{Tracy2014}.

The derivative of the spectral phase function \eqref{Eq:inverse_transform_phase_function} can be evaluated by applying the chain rule, resulting in
\begin{multline}
\label{eq:spectral_phase_first_derivative}
\frac{\partial}{\partial k_x}\Phi(k_x) =  
  x^{*'}_{P,k_x}(k_x) \underbrace{ \left( \phi^{P'}_x[x^*_P(k_x),y]  + k_x \right)}_{ = 0} - \\	
- x^{*'}_{G,k_x}(k_x) \underbrace{ \left( \phi^{G'}_x[x^*_G(k_x),y]  + k_x \right) }_{ = 0} 
+x^*_P(k_x)- x^*_G(k_x) -  x_0,
\end{multline}
where $x^{*'}_{k_x}(k_x)$ is the rate of change of the forward transform stationary positions with respect to the change of the spectral wavenumber.
The bracketed terms cancel out according to the definition of the stationary points for the forward transform \eqref{eq:xP_xG_in_spatial_domain}.
The stationary wavenumber $k_x^*(x_0)$ is then found where
\begin{equation}
\label{eq:xP_xG_in_spectral_domain}
\left. \frac{\partial}{\partial k_x}\Phi(k_x) \right|_{k_x=k_x^*(x_0)} = x^*_P(k_x^*(x_0))- x^*_G(k_x^*(x_0)) -  x_0 = 0
\end{equation}
holds.
This definition relates the evaluation points $x^*_P$ and $x^*_G$ directly to the actual SSD coordinate $x_0$, therefore the intermediate stationary wavenumber ($k_x^*$) dependency may be omitted i.e. $x^*_P(k_x^*(x_0)) \rightarrow x^*_P(x_0)$ and $x^*_G(k_x^*(x_0)) \rightarrow x^*_G(x_0)$ may be written. 

\begin{figure}[t!]
\small
  \begin{minipage}[c]{0.58\textwidth}
%  \hspace{1cm}
	\small
%	\centering
%	\hspace{-30mm}
	\begin{overpic}[width = \textwidth ]{Figures/SFS_theory/explicit_sol_stationary_point.png}
	\put(96,30){$x$}
	\put(15,80){$y$}
	\put(75.5,29.5){$x^*_P(x_0)$}
	\put(60,29.5){$x_0$}
	\put(62,69){$\vk^P(x^*_P,y)$}
	\end{overpic}  \end{minipage}\hfill
	\begin{minipage}[c]{0.4\textwidth}
    \caption{
       Illustration of the evaluation position $x^*_P(x_0)$ (and $x^*_G(x_0)$) as the function of $x_0$. 
	   For a given SSD position $x_0$ the stationary positions is found on a given reference line $y=\text{const}$, where the virtual field propagation direction coincides with that of the Green's function translated into $x_0$. 
	   Furthermore, at $x^*_P(x_0)$ the local curvature of the translated Green's function is always greater, than that of the virtual field.
       } 
       \label{fig:SFS_theroy:explicit_sol_stationary_points}
  \end{minipage}
\end{figure}
%
The definitions for the forward and inverse transform stationary points completely define the evaluation points $x^*_P$,$x^*_G$ for a given SSD position $x_0$ independently of the spectral wavenumber:
by expressing \eqref{eq:xP_xG_in_spatial_domain} applying \eqref{eq:xP_xG_in_spectral_domain} for an arbitrary $x_0$ position the evaluation points along a fixed $y$ is found, where
%
\begin{align} 
\label{Eq:stationary_evaluation_points}
k_{x}^P(x^*_P(x_0),y) = k_{x}^G(x^*_P(x_0) - x_0,y)
\end{align}
is satisfied.
This results states, that \emph{for a given SSD coordinate $x_0$ the evaluation point $x^*_P$ is found on a reference line, where the local propagation direction of the target field $P$ coincides with that of a point source positioned at $\posvec{3}{x_0}{0}{0}$}. 
For an illustration refer to Figure \ref{fig:SFS_theroy:explicit_sol_stationary_points}.
%

Having found the stationary position for \eqref{Eq:inverse_transform_def} in order to apply the SPA one still needs the phase function's second derivative around the stationary position and its sign.
The second derivative is obtained by a further differentiation of \eqref{eq:spectral_phase_first_derivative} with respect to $k_x$ leads to
\begin{multline}
\small 
\frac{\partial^2}{\partial k_x^2}\Phi(k_x) = \\ =
  x^{*''}_{P,k_x k_x}(k_x) \underbrace{ \left( \phi^{P'}_x[x^*_P(k_x),y]  + k_x \right)}_{ = 0} + 
  x^{*'}_{P,k_x}(k_x) \left( x^{*'}_{P,k_x}(k_x) \phi^{P''}_{xx}[x^*_P(k_x),y]  +2  \right)  -\\
- x^{*''}_{G,k_x k_x}(k_x) \underbrace{ \left( \phi^{G'}_x[x^*_G(k_x),y]  + k_x \right) }_{ = 0} 
- x^{*'}_{G,k_x}(k_x) \left( x^{*'}_{G,k_x}(k_x)  \phi^{G''}_{xx}[x^*_G(k_x),y]  + 2\right).
\end{multline}
The required rate of change of the stationary positions ($x^{*'}_P(k_x)$ and $x^{*'}_G(k_x)$) can be obtained by differentiating their definition \eqref{eq:spectral_phase_first_derivative} with respect to $k_x$, resulting in
\begin{equation}
x^{*'}_{P,k_x}(k_x) = -\frac{1}{\phi^{P''}_{xx}[x^*_P(k_x),y]}, \hspace{5mm} x^{*'}_{G,k_x}(k_x) = -\frac{1}{\phi^{G''}_{xx}[x^*_G(k_x),y]}.
\end{equation}
Thus the second derivative is given by
\begin{equation}
\label{eq:SFS_theory:second_Derivative_2}
\small
\frac{\partial^2}{\partial k_x^2}\Phi(k_x) = 
\frac{\phi^{P''}_{xx}[x^*_P(k_x),y] - \phi^{G''}_{xx}[x^*_G(k_x),y]}{\phi^{P''}_{xx}[x^*_P(k_x),y] \, \phi^{G''}_{xx}[x^*_G(k_x),y]} = 
\frac{1}{k\hat{k}^{P 2}_y} \left( \Rh^P(x^*_P(k_x),y) - \Rh^G(x^*_G(k_x),y) \right) 
,
\end{equation}
with $\Rh^P$ and $\Rh^G$ being the horizontal principal radii. of the target field and the Green's function, expressed by applying \eqref{Eq:App:Hessian_inplane}.
Obviously, for any source distribution behind the SSD its horizontal principal radius is larger at position at $y>0$, than that of a point source and the sign of \eqref{eq:SFS_theory:second_Derivative_2} is positive. 

These results now may be substituted back into the SPA \eqref{Eq:SPAResult} of the inverse transform \eqref{Eq:inverse_transform_def}.
For the sake of brevity in the followings the evaluation point is denoted by $x^*_P \rightarrow x^*$. 
Denoting the stationary position by $\vxref(\vxo) = \posvec{3}{x^*(x_0)}{y}{0}$ the resulting driving function is formulated as
\begin{equation}
\label{eq:SFS_theory:spatial_sdm}
D(x_0,\omega) =
\sqrt{\frac{ \left| \phiGxx(\vxref(\vxo)-\vxo,\omega )\right|^2}{\left| \phiPxx(\vxref(\vxo,\omega)) - \phiGxx(\vxref(\vxo)-\vxo,\omega)\right|}}
\sqrt{\frac{\ti}{2\pi}} 
\frac{P(\vxref(\vxo,\omega))}{G(\vxref(\vxo)-\vxo,\omega)},
\end{equation}
where $k_x^P(\vxref(\vxo)) = k_x^G(\vxref(\vxo)-\vxo)$ holds.
This result states, that an arbitrary sound field may be synthesized by finding the positions along the reference line, where the propagation direction/wavefront of the target field matches the field of the actual SSD elements.
In this stationary position the driving functions are obtained by the ratio of the target field and the actual SSD element, corrected by the factor, containing the wavefront curvatures at the same position.
Therefore the explicit, global solution can be approximated by simple local wavefront matching.

Expressing the second derivatives in terms of the principal radii according to \eqref{Eq:App:Hessian_inplane} the driving function can be cast into the form
\begin{equation}
\label{eq:SFS_theory:spatial_sdm_2}
D(x_0,\omega) =
\footnotesize
\underbrace{\sqrt{\frac{\Rh^P(\vxref(\vxo))}{\Rh^P(\vxref(\vxo))-\Rh^G(\vxref(\vxo)-\vxo) }}}_{{\substack{\text{virtual source}\\\text{compensation}}}}
\underbrace{ \sqrt{\frac{\ti k}{2\pi \Rh^G(\vxref(\vxo)-\vxo)}} }_{{\substack{\text{SSD}\\\text{compensation}}}} 
\underbrace{\frac{\hat{k}_y^{P}(\vxo))P(\vxref(\vxo),\omega)}{G(\vxref(\vxo)-\vxo,\omega)}}_{\substack{\text{2D explicit}\\\text{driving function}}}.
\end{equation}
The formulation implies the fact, that similarly to the implicit solution, the explicit driving functions also requires the derivative of the target field, measured on the reference position.
The driving function explicitly contains the virtual source compensation factor presented in \eqref{Eq:SFS_theory:25D_WFS_driv_fun_ver_2}, compensating the dimensionality mismatch for the virtual source.
Furthermore, term 
\begin{equation}
\small
\sqrt{\frac{\ti k}{2\pi \Rh^G(\vxref(\vxo)-\vxo)}} \frac{\hat{k}_y^{P}(\vxref(x_0))}{G(\vxref(\vxo)-\vxo,\omega)} = 
2 \sqrt{\frac{2 \pi |\vxref(\vxo)-\vxo|}{\ti k}} \frac{\ti k_y^{G}(\vxref(\vxo)-\vxo)  }{\te^{-\ti k |\vxref(\vxo)-\vxo|}} 
\end{equation}
compensates for the transfer function of the SSD contour---or more specifically the stationary SSD element---both regarding its frequency response being a half-integrator and its attenuation factor.

One important fact is pointed out here: although having derived the above driving functions in terms of a forward an inverse spatial Fourier transform along a straight line, there is no restriction on the $y$-coordinate of the stationary point in \eqref{eq:SFS_theory:spatial_sdm} due to the local approximations involved: the $y$-coordinate might be $x_0$-dependent.
This means, that an arbitrary referencing curve may be defined as $\vxref(x_0)$, and the driving functions can be calculated by finding the stationary positions 
satisfying $k_x^P(\vxref(x_0)) = k_x^G(\vxref(x_0) - \vx_0)$ along this curve.
Evaluating the driving functions in the stationary positions will result in amplitude correct synthesis along the reference curve. 
This means, that the presented driving functions are equivalent to the 2.5D WFS driving functions with the important difference that here the target field needs to be evaluated on the reference curve.
Furthermore, within the validity of the Kirchhoff-approximation also the SSD has to be not necessarily linear: the spatial explicit driving functions can be applied using an arbitrary shaped SSD contour.
Obviously in that case \eqref{eq:SFS_theory:spatial_sdm_2} has to be evaluated with $\hat{k}_y^{P}(\vxref(x_0)) \rightarrow \hat{k}_{\mathrm{n}}^{P}(\vxref(x_0))$, i.e. with the wavenumber component along the normal direction of the actual SSD element.

\subsection*{Application example: Synthesis of a 3D point source using a linear SSD}

In the followings a simple example is presented in order to demonstrate the validity of the spatial SDM driving functions with the example of the synthesis of a virtual 3D point source.
For the synthesis a linear secondary source distribution is applied, located at $\vxo = \posvec{3}{x_0}{0}{0}$.
The virtual source is positioned at $\vxs = [x_s,\ y_s,\ 0]^{\mathrm{T}}$ with $y_s < 0$, referencing the synthesis to a circle around the virtual point source with the radius of $R_{\mathrm{ref}}$.
Along with the equation describing the reference curve $\vxref(x_0) = \posvec{3}{x^*(x_0)}{y^*(x_0)}{0}$ the stationary points satisfy the following equations
\begin{align}
\vk^G(\vxref(\vxo)-\vxs) &= \vk^G(\vxref(\vxo)-\vxo), \\
|\vxref - \vxs|    &= R_{\mathrm{ref}}.
\end{align}
The~solution for the equations is given by
\begin{align}
\label{Eq:SFS_theory:spatial_SDM_circle_ref_points}
\vxref(\vxo) = \vxs + R_{\mathrm{ref}}\frac{\vxo-\vxs}{|\vxo-\vxs|}.
\end{align}
Substituting the spherical virtual field into \eqref{eq:SFS_theory:spatial_sdm_2}---with the principal radii given by simple distances from the point sources--- yields the explicit driving function in the spatial domain for a virtual point source.
\begin{equation}
D(x_0) =
\sqrt{\frac{|\vxref-\vxs|}{|\vxref-\vxs|-|\vxref-\vxo| }}
\sqrt{\frac{\ti k}{2\pi |\vxref-\vxo|}} 
\hat{k}_y^G(\vxref(\vxo)-\vxs)
\frac{G(\vxref(\vxo)-\vxs)}{G(\vxref(\vxo)-\vxo)}
\end{equation}
Substituting the reference position coordinates along the reference circle \eqref{Eq:SFS_theory:spatial_SDM_circle_ref_points} specifies the driving functions, optimizing the synthesis on the reference circle
\begin{equation}
\label{Eq:SFS_theory:linear_SSD_ref_circle}
D(x_0) =-y_s
\sqrt{\frac{R_{\mathrm{ref}}-|\vxo-\vxs|}{R_{\mathrm{ref}}}}
\sqrt{\frac{\ti k }{2\pi}} 
\frac{\te^{-\ti k |\vxo-\vxs|}}
{ |\vxo-\vxs|^{\frac{3}{2}} }.
\end{equation}
%
\begin{figure}
\centering
	\begin{overpic}[width = 1\columnwidth ]{Figures/SFS_theory/25D_spatial_SDM_linear_SSD.png}
	\end{overpic}   
    \caption{2.5D synthesis of a 3D point source located at $\vxs = \posvec{3}{0}{-2}{0}$, radiating at $\omega_0 = 2\pi \cdot 1 \mathrm{krad}/s$.
	The synthesis is referenced on a circle around the virtual source, with a radius of $R_{\mathrm{ref}} = 4 ~ \mathrm{m}$.
    Figure (a) depicts the real part of the synthesized field, (b) shows the error of synthesis.
    }
\label{fig:SFS_theory:25D_spatial_SDM_linear_ssd}  
\end{figure}
Investigating Figure \ref{fig:SFS_theory:25D_spatial_SDM_linear_ssd} verifies, that the synthesis is optimized on the prescribed reference curve.

Although having derived the above driving function from the pressure, measured along the reference curve, \eqref{Eq:SFS_theory:linear_SSD_ref_circle} is already written merely in terms of the target field, measured on the actual SSD element, equivalently to the WFS solution.
In the followings this relation is generalized by expressing the explicit driving functions for an arbitrary target sound field is written in terms of pressure, measured along the SSD, revealing the general connection between the implicit and explicit solutions.

\section{Relation of implicit and explicit solutions}

As given in the previous section the asymptotic formulation of the explicit driving function written in terms of the horizontal principal radii of the involved wave fields reads as
\begin{equation}
\label{eq:SFS_theory:spatial_sdm_3}
D(x_0,\omega) =
\footnotesize
\sqrt{\frac{\Rh^P(\vxref(\vxo))}{\Rh^P(\vxref(\vxo))-\Rh^G(\vxref(\vxo)-\vxo) }}
\sqrt{\frac{\ti k}{2\pi \Rh^G(\vxref(\vxo)-\vxo)}} 
\frac{\hat{k}_y^{P}(\vxref(x_0))P(\vxref(\vxo),\omega)}{G(\vxref(\vxo)-\vxo,\omega)},
\end{equation}
with the reference position and the actual SSD position related by satisfying 
\begin{equation}
\label{Eq:stationary_evaluation_points}
\vk^P(\vxref(\vxo)) = \vk^G(\vxref(\vxo)-\vxo).
\end{equation}
In order to express the driving functions merely in terms of the involved quantities measured on the SSD, all the principal radius, the local wavenumber vector and the pressure of the target field has to be defined along the $y =0$ line.
For the principal radii this connection was given by \eqref{eq:app:propagated_radii}:
\begin{equation}
\label{Eq:principal_radii_addition}
\Rh^P(\vx)=  \Rh^P(\vxo^*(\vx)) + \Rh^G(\vx-\vxo^*(\vx)),
\end{equation}
stating that the principal radii of an arbitrary field increase linearly along the path of propagation.
Furthermore, assuming isotropic medium, the propagation direction---and the local wavenumber vector---does not change along the propagation path, and $\vk^P(\vxref(\vxo)) = \vk^P(\vxo)$ holds.
With further manipulation the driving function results in
\begin{equation}
\label{eq:SFS_theory:spatial_sdm_3}
D(x_0,\omega) =
\small
\sqrt{\frac{\Rh^P(\vxo) + \Rh^G(\vxref(\vxo)-\vxo)}{\Rh^P(\vxo) \cdot \Rh^G(\vxref(\vxo)-\vxo) }}
\sqrt{\frac{1}{2\pi \ti k}} 
\frac{\ti k_y^{P}(\vxo) \, P(\vxref(\vxo),\omega)}{G(\vxref(\vxo)-\vxo,\omega)}.
\end{equation}

The relation between $P(\vxref(\vxo))$ and $P(\vxo)$---i.e. the target field measured at an arbitrary reference position and measured along the SSD---can be established by the asymptotic evaluation of the Rayleigh integral, as it was discussed in details in section \ref{Sec:HF:RayleighSPA}, given for a general wave field by \eqref{eq:HF_approx:asymptotic_rayleigh}.
In the present case the principal curvatures are given by the horizontal and vertical curvature components.
Specializing the formulation for the present problem geometry, expressing in terms of the principal radii yields
\begin{equation}
\label{Eq:Asymptotic_Rayleigh_integral}
\footnotesize
P(\vx,\omega) = 4\pi
\sqrt{\frac{\Rh^P(\vxo^*(\vx)) \cdot \Rh^G(\vx-\vxo^*(\vx))}{\Rh^P(\vxo^*(\vx)) + \Rh^G(\vx-\vxo^*(\vx))}}
\sqrt{\frac{\Rv^P(\vxo^*(\vx)) \cdot \Rv^G(\vx-\vxo^*(\vx))}{\Rv^P(\vxo^*(\vx)) + \Rv^G(\vx-\vxo^*(\vx))}}
P(\vxo^*(\vx),\omega) G(\vx-\vxo^*(\vx),\omega)
,
\end{equation}
where positions $\vx$ and $\vxo^*(\vx)$ are related through the implicit relation
\begin{equation}
\label{Eq:SFS_theory:Rayleigh25D_horizontal_stat_point}
\vk^P(\vxo^*(\vx)) = \vk^G( \vx - \vxo^*(\vx)).
\end{equation}

Comparing the definition of the stationary points for the Rayleigh integral \eqref{Eq:SFS_theory:Rayleigh25D_horizontal_stat_point} and the stationary SDM evaluation points \eqref{Eq:stationary_evaluation_points}, it is revealed, that they describe stationary point pairs.
Therefore, without the loss of generality the target pressure $P(\vxref(\vxo),\omega)$ can be formulated by the asymptotic Rayleigh integral \eqref{Eq:Asymptotic_Rayleigh_integral} with choosing the SSD position $\vxo$ as an independent variable, therefore \eqref{eq:SFS_theory:spatial_sdm_3} can be expressed as
\begin{equation}
D(x_0,\omega) =
\small
2\sqrt{\frac{2\pi  \Rv^G(\vxref(\vxo)-\vxo)}{\ti k}} 
\sqrt{\frac{\Rv^P(\vxo) }{\Rv^P(\vxo) + \Rv^G(\vxref(\vxo)-\vxo)}}
\ti k_y^{P}(\vxo)
P(\vxo,\omega),
\end{equation}
or expressed in terms of the second phase derivatives 
\begin{equation}
D(x_0,\omega) =
\small
2\sqrt{\frac{2\pi }{\ti }} 
\frac{ 1 }{\sqrt{\phi^{''P}_{zz}(\vxo) + \phi^{''G}_{zz}(\vxref(\vxo)-\vxo)}}
\ti k_y^{P}(\vxo)
P(\vxo,\omega).
\end{equation}

Comparison with \eqref{Eq:SFS_theory:25D_WFS_driv_fun_ver_2} and \eqref{Eq:SFS_theory:25D_WFS_driv_fun} reveals that the asymptotic SDM driving functions exactly coincide the 2.5D WFS driving function when applied for a linear---and within the validity of the Kirchhoff approximation for an arbitrary shaped---SSD.
It is therefore verified that under high frequency assumptions WFS is the asymptotic or local approximation of the global explicit solution.

An important difference is however that the WFS driving functions were obtained from the 2.5D Neumann Rayleigh integral in an intuitive manner, by introducing the reference curve concept with interchanging the role of the receiver position and its stationary SSD position. 
On the other hand the explicit driving function \eqref{eq:SFS_theory:spatial_sdm} inherently contains the horizontal SPA and the reference curve concept.

\subsection*{Application example: Synthesis applying directive secondary sources}
\label{Sec:Dir_SSD}