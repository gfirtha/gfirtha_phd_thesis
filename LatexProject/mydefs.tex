%
\renewcommand{\floatpagefraction}{.99}

\newcount\posveccount
\newcommand*\posvec[1]{
        \global\posveccount#1
        [
        \posvecnext
}
\def\posvecnext#1{
        #1
        \global\advance\posveccount-1
        \ifnum\posveccount>0
                ,\
                \expandafter\posvecnext
        \else
                ]^{\mathrm{T}}
        \fi
}

\newcount\colveccount
\newcommand*\colvec[1]{
        \global\colveccount#1
        \begin{bmatrix}
        \colvecnext
}
\def\colvecnext#1{
        #1
        \global\advance\colveccount-1
        \ifnum\colveccount>0
                \\[5pt]
                \expandafter\colvecnext
        \else
                \end{bmatrix}
        \fi
}

\definecolor{maincolor}{rgb}{0, 0.447, 0.741}

\newtcolorbox{oddnote}[1][]{
    width=\marginparwidth,
    left=2pt,
    right=0pt,
    top=0pt,
    bottom=0pt,
    empty,
    notitle,
    sharp corners,
    borderline west={1pt}{0pt}{maincolor},
    halign=left,
    #1
}

\newtcolorbox{evennote}[1][]{
    width=\marginparwidth,
    left=0pt,
    right=2pt,
    top=0pt,
    bottom=0pt,
    empty,
    notitle,
    sharp corners,
    borderline east={1pt}{0pt}{maincolor},
    halign=right,
    #1
}

\newcommand{\SideNote}[2][]{%
    \begin{tikzpicture}[remember picture,overlay]
    \node[inner sep=0pt,anchor=west] at ([xshift=0pt]current page marginpar area.west|-{pic cs:})
{\ifthispageodd
{\begin{oddnote}{\color{maincolor} \textit{ {\footnotesize \noindent#2 }}}\end{oddnote}}
{\begin{evennote}{\color{maincolor} \textit{ {\footnotesize \noindent#2 }}}\end{evennote}}};
    \end{tikzpicture}%
}
%%
\newcommand{\importanteq}[2]{\begin{equation}\SideNote{#1} #2 \end{equation}}
\newcommand{\importantalign}[2]{\begin{align}\SideNote{#1} #2 \end{align}}
\newcommand{\importantmline}[2]{\begin{multline}\SideNote{#1} #2 \end{multline}}
\newcommand{\importanteqnarray}[2]{\begin{eqnarray}\SideNote{#1} #2 \end{eqnarray}}


%%\setcounter{secnumdepth}{2}
%  \renewcommand\not[1]{#1\xnot}
%  \renewcommand{\notin}{\not\in}
  
  
\newcommand{\dint}{\int\!\!\!\!\!\int}
\newcommand{\tint}{\int\!\!\!\!\int\!\!\!\!\int}
\newcommand{\qint}{\int\!\!\!\!\int\!\!\!\!\int\!\!\!\!\int}
\newcommand{\td}{\mathrm{d}}
\newcommand{\te}{\mathrm{e}}
\newcommand{\ti}{\mathrm{j}}
\newcommand{\sinfi}{\sin\varphi}
\newcommand{\cosfi}{\cos\varphi}
\newcommand{\sinteta}{\sin\theta}
\newcommand{\costeta}{\cos\theta}
\newcommand{\yref}{y_{\mathrm{ref}}}
\newcommand{\dref}{d_{\mathrm{ref}}}
\newcommand{\vx}{\mathbf{x}}
\newcommand{\vn}{\mathbf{n}}
\newcommand{\vxo}{\mathbf{x}_0}
\newcommand{\vni}{\mathbf{n}_{\mathrm{in}}}
\newcommand{\vno}{ \mathbf{n}_{\mathrm{out}} }
\newcommand{\vxs}{\mathbf{x}_{\mathrm{s}}}
\newcommand{\vxref}{\mathbf{x}_{\mathrm{ref}}}
\newcommand{\vk}{\mathbf{k}}
\newcommand{\vhk}{\hat{\mathbf{k}}}
\newcommand{\kn}{k_\mathrm{n}}
\newcommand{\kt}{k_\mathrm{t}}
\newcommand{\Oi}{\Omega_{\mathrm{i}}}
\newcommand{\Oe}{\Omega_{\mathrm{e}}}
\newcommand{\dO}{\partial \Omega}
\newcommand{\Div}{\mathrm{div}}
\newcommand{\Dx}{\nabla_{\!\!\vx}\,}
\newcommand{\Dxo}{\nabla_{\!\!\vxo}\,}
\newcommand{\Lx}{\nabla^2_{\!\!\vx}}
\newcommand{\vv}{\mathbf{v}}
\newcommand{\vvs}{\mathbf{v}_{\mathrm{s}}}
\newcommand{\mH}{\mathbf{H}}

\newcommand{\Kv}{\kappa_\mathrm{v}}
\newcommand{\Kh}{\kappa_\mathrm{h}}
\newcommand{\Rv}{\rho_\mathrm{v}}
\newcommand{\Rh}{\rho_\mathrm{h}}

\newcommand{\fgcom}[1]{{\color{red}#1}}

\newcommand{\phix}{\phi'_{x}}
\newcommand{\phixx}{\phi''_{xx}}

\newcommand{\phiy}{\phi'_{y}}
\newcommand{\phiyy}{\phi''_{yy}}

\newcommand{\phiz}{\phi'_{z}}
\newcommand{\phizz}{\phi''_{zz}}

\newcommand{\phiPxx}{\phi^{P''}_{xx}}
\newcommand{\phiGxx}{\phi^{G''}_{xx}}

\newcommand{\phiPyy}{\phi^{P''}_{yy}}
\newcommand{\phiGyy}{\phi^{G''}_{yy}}

\newcommand{\phiPzz}{\phi^{P''}_{zz}}
\newcommand{\phiGzz}{\phi^{G''}_{zz}}

\newcommand{\Phikk}{\Phi''_{k_x k_x}}

\newcommand{\tE}{t_{\mathrm{e}}}
\newcommand{\Tret}{{\scriptstyle \frac{|\vx-\vxo|}{c} }}
\newcommand{\TretS}{{\scriptstyle \frac{|\vx-\vxo^*(\vx,t)|}{c} }}

\newcommand{\FT}[2][]{
\mathcal{F}_{#1} \left\{ #2 \right\}
}
\newcommand{\IFT}[2][]{
\mathcal{F}^{-1}_{#1} \left\{ #2 \right\}
}

\renewcommand{\arraystretch}{1}

\newtheoremstyle{thesisgroupstyle}%
	{1em}% 				% Space above
	{5mm}% 				% Space below
	{\slshape}%			% Font type of body
	{0mm}%				% Indentation			
	{\bfseries}%		% Font of title
	{\newline}%			% Punctuation between head and body
	{1em}%				% Space after theorem head
	{}% 				% Manually specify head
\makeatother
\theoremstyle{thesisgroupstyle}
\newtheorem{thesisgroup}{Thesis group}
\renewcommand{\thethesisgroup}{\Roman{thesisgroup}}
\makeatletter
\newtheoremstyle{indented}
  {5pt}% space before
  {5pt}% space after
  {\addtolength{\@totalleftmargin}{3.5em}
   \addtolength{\linewidth}{-3.5em}
   \parshape 1 3.5em \linewidth}% body font
  {}% indent
  {\bfseries}% header font
  {.}% punctuation
  {.75em}% after theorem header
  {}% header specification (empty for default)
\makeatother
\theoremstyle{indented}
\newtheorem{thesis}{Thesis}[thesisgroup]